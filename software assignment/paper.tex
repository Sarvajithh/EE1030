\documentclass{article}
\usepackage[preprint]{neurips_2024}
\usepackage[utf8]{inputenc} % allow utf-8 input
\usepackage[T1]{fontenc}    % use 8-bit T1 fonts
\usepackage{hyperref}       % hyperlinks
\usepackage{url}            % simple URL typesetting
\usepackage{booktabs}       % professional-quality tables
\usepackage{amsfonts}       % blackboard math symbols
\usepackage{nicefrac}       % compact symbols for 1/2, etc.
\usepackage{microtype}      % microtypography
\usepackage{xcolor}         % colors
\usepackage{listings}
\usepackage{amsmath}
\lstset{
    language=C,
    basicstyle=\ttfamily\small, % Code font size
    keywordstyle=\color{purple}\bfseries, % Color for keywords
    stringstyle=\color{red}, % Color for strings
    commentstyle=\color{green!50!black}, % Color for comments
    numbers=left, % Line numbers on the left
    numberstyle=\tiny\color{gray}, % Line number styling
    frame=single, % Frame around the code
    breaklines=true, % Enable line breaking
    captionpos=b, % Caption position (bottom)
    tabsize=4 % Tabs as 4 spaces
}


%\renewenvironment{abstract}{\begin{center}\bfseries\large\end{center}\vspace{1em}}{}

\title{Eigen Value Decomposition}

\author{%
G Sarvajith\\
\texttt{ai24btech11008@iith.ac.in}\\
}
\begin{document}
\maketitle
\section{Eigen Value Decomposition}
Eigenvalue decomposition (EVD) is a method of decomposing a square matrix into its eigenvalues and eigenvectors. It is one of the cornerstones of linear algebra, used extensively in mathematical modeling, numerical computation, and machine learning.\\
\textbf{Eigenvalues}$(\lambda)$ are scalars associated with the transformation described by matrix.\\
\textbf{Eigenvectors(v)} are vectors that remain in the same direction(or opposite) after the transformation of matrix.\\
The relationship between a matrix(A), its eigenvalues and eigenvectors is$$Av=\lambda v$$
The goal of Eigenvalue Decomposition is to express matrix A as:$$A =V\Lambda V^{-1}$$\\ 
Where
\begin{itemize}
 \item A: Original $n\times n$ matrix.
 \item V: A square $n\times n$ matrix whose columns are the eigenvectors of A.
 \item $\Lambda$: A diagonal $n\times n$ matirx where each diagonal entry is an eigenvalue of A. 
 \item $V^{-1}$: The inverse of V.
\end{itemize}
Conditions for Eigenvalue Decomposition\\
\begin{itemize}
    \item A matrix \textbf{A} should be diagonalizable(i.e it has a full set of n linearly independent eigenvectors).
    \item If A doesn't have n linearly independent eigenvectors, it cannot be decomposed.
\end{itemize}
Eigenvalue decomposition provides a way to understand how the matrix A scales and rotates vectors. Eigenvectors determine the directions in which the transformation occurs.
Eigenvalues determine the scaling factors along those directions.\\
\section{QR algorithm }
The QR algorithm for eigenvalue decomposition relies on repeatedly factoring a matrix into an orthogonal matrix \textbf{Q} and an upper triangular matrix \textbf{R}.

QR decomposition is the process of decomposing a matrix A into:$$ A = QR$$\\
\begin{itemize}
    \item Q is an orthogonal matrix, and columns of Q are orthonormal vectors.
    \item R is an upper triangular matrix.
\end{itemize}
There are many ways to compute the matrices Q, R. \textbf{Gram-Schmidt orthogonalization} process can be used to orthogonal matrix Q and Upper triangular matrix R.
\section{Gram-Schmidt orthogonlization}
Gram-Schmidt orthogonalization is a process to convert a set of linearly independent vectors into an orthonormal set of vectors, meaning the vectors are:
\begin{itemize}
    \item Orthogonal: Each pair of vectors is perpendicular (their dot product is zero).
    \item Normalized: Each vector has unit length (magnitude of 1).
\end{itemize}
\subsection{Upper Triangular matrix(R) and Orthogonal Matrix(Q)}
A matrix $A$ with columns as $a_1, a_2, a_3,....., a_n$ is given as an input.\\
\begin{enumerate}
\item In the first step the first column of A($a_1$) is normalized to give the vector $q_1$.$$q_1 =\frac{a_1}{||a_1||}$$
\item For the other columns, it follows the following process:\\
      for j=2,3,...,n:

$$    q_j =\frac{u_j}{||u_j||}$$
where
$$u_j = a_j - \sum_{i=1}^{j-1}(q_i^T a_j)q_i $$

\item $$R_{ij}=\begin{cases}q_i^T a_j & \text{if } i \leq j, \\  0 & \text{if } i > j.\end{cases}$$
\item The matrix with columns \( q_1, q_2, \dots, q_n \) is \( Q \).
\end{enumerate}
\section{Iterative QR algorithm for Eigenvalues}
Using QR decomposition iteratively transforms the matrix A into a nearly diagonal matrix, revealing its Eigenvalues.
\begin{itemize}
    \item We will start with a matrix $A_0$ = A.
    \item QR Decomposition :$$A_k = Q_kR_k$$
    \item Update $A_{k+1}:$ $$A_{k+1} = R_kQ_k$$.
    \item Repeat these steps until $A_k$ converges to a diagonal matrix $\Lambda$, whose diagonal entries are the eigen values of A.
\end{itemize}
The matrix $A_k$(updated after each iteration) converges to an upper triangle matrix whose diagonal elements are the Eigenvalues of the original matrix A. This matrix is referred as \textbf{Schur form} when A is asymmetric and \textbf{diagonal matrix} when A is symmetric.
\subsection{Details of convergence}
\begin{enumerate}
    \item If  A is symmetric, then it always converges to a diagonal matrix $A_k$.
    \item If A is not symmetric, then the QR algorithm converges to an upper triangular matrix. The diagonal elements of this matrix are Eigenvalues.
\end{enumerate}
The iterative step in the QR algorithm stops when the matrix $A_k$ satisfies convergence criterion. This occurs when the off-diagonal elements of $A_k$ become sufficiently small or within chosen tolerance.
\\Stopping criteria in the QR algorithm:
\begin{itemize}
    \item The algorithm stops when the off-diagonal elements of $A_k$ are smaller than a predefined threshold $\epsilon$
    \item The algorithm can also stop when the difference between successive iterates becomes negligible:$$||A_{k+1}-A_k||<\epsilon$$
    \item To avoid infinite loops in cases of slow convergence or divergence a maximum number of iterations $N_{max}$ is often imposed.
\end{itemize}
\section{C-Code}
\subsection{Helper Functions}
\begin{lstlisting}
    double** allocateMatrix(int n) {
    double** matrix = (double**)malloc(n * sizeof(double*));
    for (int i = 0; i < n; i++) {
        matrix[i] = (double*)malloc(n * sizeof(double));
    }
    return matrix;
}
\end{lstlisting}
\begin{itemize}
\item This function allocates a square matrix with double data type values, where the dimension of the to be created matrix (n) is taken as input.
\end{itemize}
\begin{lstlisting}
    void freeMatrix(double** matrix, int n) {
    for (int i = 0; i < n; i++) {
        free(matrix[i]);
    }
    free(matrix);
}
\end{lstlisting}
\begin{itemize}
\item This function frees the dynamically allocated matrix, by the previous function. This is done to increase the memory efficiency.
\end{itemize}
\begin{lstlisting}
    void multiplyMatrices(double** A, double** B, double** C, int n) {
    for (int i = 0; i < n; i++) {
        for (int j = 0; j < n; j++) {
            C[i][j] = 0;
            for (int k = 0; k < n; k++) {
                C[i][j] += A[i][k] * B[k][j];
            }
        }
    }
}
\end{lstlisting}
As the function name suggests, it performs matrix multiplication.
\begin{lstlisting}
    void transposeMatrix(double** A, double** B, int n) {
    for (int i = 0; i < n; i++) {
        for (int j = 0; j < n; j++) {
            B[j][i] = A[i][j];
        }
    }
}
\end{lstlisting}
\begin{itemize}
\item This function performs the transpose of a matrix.
\item It takes a matrix as an input and stores it in another matrix.
\end{itemize}
\subsection{QR-Algorithm's Functions}
\begin{lstlisting}
    void qrDecomposition(double** A, double** Q, double** R, int n) {
    for (int i = 0; i < n; i++) {
        for (int j = 0; j < n; j++) {
            Q[i][j] = A[i][j];
        }
    }


    for (int j = 0; j < n; j++) {
        R[j][j] = 0;
        for (int i = 0; i < n; i++) {
            R[j][j] += Q[i][j] * Q[i][j];
        }
        R[j][j] = sqrt(R[j][j]);
        
        for (int i = 0; i < n; i++) {
            Q[i][j] /= R[j][j];
        }
        
        for (int k = j + 1; k < n; k++) {
            R[j][k] = 0;
            for (int i = 0; i < n; i++) {
                R[j][k] += Q[i][j] * Q[i][k];
            }
            for (int i = 0; i < n; i++) {
                Q[i][k] -= R[j][k] * Q[i][j];
            }
        }
    }
}
\end{lstlisting}
\begin{itemize}
\item This block of code implements the Gram-Schmidt process for QR decomposition.
\item It takes an input matrix A and decomposes it into an orthogonal matrix Q and an upper triangular matrix R.
\item Initially all the elements in A are copied into Q, and then the matrix Q gets modified.
\item The values in Q gets modified into orthonormal vetcors.
\item \textbf{Note:} This is not the iterativ process.
\item After the completion of loop, Q becomes a matrix of orthogonal unit vectors.
\item  R accumulates the norms and projection coefficients and forms an upper triangular matrix.
\end{itemize}
\begin{lstlisting}
    void qrAlgorithm(double** A, int n) {
    double** Q = allocateMatrix(n);
    double** R = allocateMatrix(n);
    double** temp = allocateMatrix(n);

    for (int iter = 0; iter < MAX_ITER; iter++) {
        qrDecomposition(A, Q, R, n);  // QR decomposition of A
        multiplyMatrices(R, Q, temp, n);  // A = R * Q
        
        // Check convergence by looking at off-diagonal elements
        int converged = 1;
        for (int i = 0; i < n; i++) {
            for (int j = 0; j < n; j++) {
                if (i != j && fabs(A[i][j]) > EPSILON) {
                    converged = 0;
                    break;
                }
            }
            if (!converged) break;
        }

        if (converged) break;

        // Update A with the new values
        for (int i = 0; i < n; i++) {
            for (int j = 0; j < n; j++) {
                A[i][j] = temp[i][j];
            }
        }
    }

    printf("Eigenvalues:\n");
    for (int i = 0; i < n; i++) {
        printf("%lf ", A[i][i]);  // Eigenvalues are on the diagonal
    }
    printf("\n");

    freeMatrix(Q, n);
    freeMatrix(R, n);
    freeMatrix(temp, n);
}
\end{lstlisting}
\begin{itemize}
\item The actual QR algorithm is performed by this code. 
\item It takes the targeted matrix A as its input, and starts with allocating matrices for Q,R.
\item Then starts the iterative process for QR Decomposition.
\item The loop goes on until it reaches Max number of iterations or it reaches the convergence criterion. 
\end{itemize}
\end{document}