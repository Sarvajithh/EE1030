\let\negmedspace\undefined
\let\negthickspace\undefined
\documentclass[journal]{IEEEtran}
\usepackage[a5paper, margin=10mm, onecolumn]{geometry}
\usepackage{lmodern} % Ensure lmodern is loaded for pdflatex
 % Include tfrupee package
\setlength{\headheight}{1cm} % Set the height of the header box
\setlength{\headsep}{0mm}     % Set the distance between the header box and the top of the text
\usepackage{enumitem}
\usepackage{gvv-book}
\usepackage{gvv}
\usepackage{cite}
\usepackage{amsmath,amssymb,amsfonts,amsthm}
\usepackage{algorithmic}
\usepackage{graphicx}
\usepackage{textcomp}
\usepackage{xcolor}
\usepackage{txfonts}
\usepackage{listings}
\usepackage{enumitem}
\usepackage{mathtools}
\usepackage{gensymb}
\usepackage{graphicx}
\usepackage{wrapfig}
\usepackage{comment}
\usepackage[breaklinks=true]{hyperref}
\usepackage{tkz-euclide} 
\usepackage{listings}
% \usepackage{gvv}                                        
\def\inputGnumericTable{}                                 
\usepackage[latin1]{inputenc}                                
\usepackage{color}                                            
\usepackage{array}                                            
\usepackage{longtable}                                       
\usepackage{calc}                                             
\usepackage{multirow}                                         
\usepackage{hhline}                                           
\usepackage{ifthen}                                           
\usepackage{lscape}
\begin{document}

\bibliographystyle{IEEEtran}
\vspace{3cm}


\author{AI24BTECH11008- Sarvajith
}
\title{Assignment 11}
% \maketitle
% \newpage
% \bigskip
{\let\newpage\relax\maketitle}
\title{2011, MA}
\renewcommand{\thefigure}{\theenumi}
\renewcommand{\thetable}{\theenumi}
\setlength{\intextsep}{10pt} % Space between text and floats
\numberwithin{equation}{enumi}
\numberwithin{figure}{enumi}
\renewcommand{\thetable}{\theenumi}
\begin{enumerate}
    \item[14.] The linear Programming Problem:\\Maximize $z=x_1+x_2$\\subject to \hfill (2011)
    \begin{align*}
     x_1 + 2x_2 \leq 20\\
     x_1 + x_2 \leq 15\\
     x_2\leq 6\\
     x_1, x_2 \geq 0
    \end{align*}
    \begin{enumerate}[label=(\Alph*)]
        \item has exactly one optimum solution
        \item has more than one optimum solution
        \item has unbounded solution
        \item has no solution
    \end{enumerate}
    \item[15.] Consider the Primal Linear Programming Problem:\\
    P: $\begin{cases} \text{Maximize} z = c_1x_1+c_2x_2+\ldots+c_nx_n\\\text{subject to}\\a_{11}x_1+a_{12}x_2+\ldots+a_{1n}x_n \leq b_1\\a_{21}x_1+a_{22}x_2+\ldots+a_{2n}x_n \leq b_2\\.      .      .\\.      .      .\\.      .      .\\a_{m1}x_1+a_{m2}x_2+\ldots+a_{mn}x_n \leq b_m\\ x_j \geq 0, j=1,\ldots,n\end{cases}$
    \\The Dual of P is\\
    D: $\begin{cases} \text{Minimize} z' = b_1w_1+b_2w_2+\ldots+b_mw_n\\\text{subject to}\\a_{11}w_1+a_{21}w_2+\ldots+a_{m1}w_m \geq c_1\\a_{12}w_1+a_{22}w_2+\ldots+a_{m2}w_m \geq c_2\\.      .      .\\.      .      .\\.      .      .\\a_{1n}w_1+a_{2n}w_2+\ldots+a_{mn}w_m \geq c_n\\ w_i \geq 0, i=1,\ldots,m\end{cases}$
    \hfill (2011)
    \begin{enumerate}[label=(\Alph*)]
        \item $N_g<<N_e$
        \item $N_g>>N_e$
        \item $N_g\approx N_e\approx \frac{N}{2}$
        \item $N_g-N_e\approx \frac{N}{2}$
    \end{enumerate}
    \item[16.] The number of irreducible quadratic polynomials over the field of 2 elements $F_2$ is  \hfill (2011)
    \begin{enumerate}[label=(\Alph*)]
        \item 0
        \item 1
        \item 2
        \item 3
    \end{enumerate}
    \item[17.] The number of elements in the conjugacy class of the 3-cycle$\brak{2 3 4}$ in symmetric group $S_6$ is  \hfill (2011)
    \begin{enumerate}[label=(\Alph*)]
        \item 20
        \item 40 
        \item 120
        \item 216
    \end{enumerate}
    \item[18.] The initial value Problem
    $$x\frac{dy}{dx} = y + x^2, x>0; y\brak{0} = 0,$$ has  \hfill (2011)
    \begin{enumerate}[label=(\Alph*)]
        \item infinitely many solutions
        \item exactly two solutions
        \item a unique solution
        \item no solution
    \end{enumerate}
    \item[19.] The subspace $P = \{\brak{x,y,z}\in R^3 : z = x^2 + y^2 + 1\}$ is  \hfill (2011)
    \begin{enumerate}[label=(\Alph*)]
        \item comapct and connected
        \item compact but not connected
        \item not compact but connected
        \item neither compact nor connected
    \end{enumerate}
    \item[20.] Let $P = \brak{0,1}; Q=[0,1); U = (0,1], T = R \text{and} A = \{P,Q,U,S,T\}$. The equivalence relation 'homeomorphism' induces which one of the following as the partition of A?   \hfill (2011)
    \begin{enumerate}[label=(\Alph*)]
        \item \{P,Q,U,S\},\{T\}
        \item \{P,T\},\{Q\},\{U\},\{S\}
        \item \{P,T\},\{Q,U,S\}
        \item \{P,T\},\{Q,U\},\{S\}
    \end{enumerate}
    \item[21.] Let $x =\brak{x_1, x_2,\ldots}\in l^4, x\neq 0.$ For which one of the following values of p, the series $\sum_{i=1}^{\infty}x_iy_i$ converges for every $y=\brak{y_1,y_2,\ldots}\in l^p$?  \hfill (2011)
    \begin{enumerate}[label=(\Alph*)]
        \item 1
        \item 2
        \item 3
        \item 4
    \end{enumerate} 
    \item[22.] Let $H$ be a complex Hilbert space and $H^*$ be its dual. The mapping $\phi:H\rightarrow H^*$ defined by $\phi\brak{y}f_y \text{where} f_y\brak{x} = \langle x,y \rangle$ is   \hfill (2011)
    \begin{enumerate}[label=(\Alph*)]
        \item not linear but onto
        \item both linear and onto
        \item linear but not onto
        \item neither linear nor onto
    \end{enumerate}
    \item[23.] A horizontal lever is in static equilibrium under the application of vertical forces $F_1$ at a distance $l_1$ from the fulcrum and $F_2$ at a distance $l_2$ from the fulcrum. The equilibrium for the above quantities can be obtained if  \hfill (2011)
     \begin{enumerate}[label=(\Alph*)]
        \item $F_1L_1 = 2F_2L_2$
        \item $2F_1L_1 = F_2L_2$
        \item $F_1L_1 = F_2L_2$
        \item $F_1L_1 < F_2L_2$
     \end{enumerate}
    \item[24.] Assume $F$ to be a twice continuously differentiable function. Let $J\brak{y}$ be a functional of the form $$\int_{0}^{1}F\brak{x,y'}dx, 0\leq x\leq 1$$ defined on the set of all continuously differentiable functions y on [0,1] satisfying $y\brak{0}=a,y\brak{1}=b.$ For some arbitrary constant c, a necessary condition for y to be an extremum of J is   \hfill (2011)
    \begin{enumerate}[label=(\Alph*)]
        \item $\frac{\partial F}{\partial x}=c$
        \item $\frac{\partial F}{\partial y'}=c$
        \item $\frac{\partial F}{\partial y}=c$
        \item $\frac{\partial F}{\partial x}=0$
    \end{enumerate}
    \item[25.] The eigenvalue $\lambda$ of the following Fredholm integral equation $$y\brak{x}= \lambda\int_{0}^{1}x^2ty\brak{t}dt$$ is  \hfill (2011)
    \begin{enumerate}[label=(\Alph*)]
        \item -2
        \item 2
        \item 4
        \item -4
    \end{enumerate}
    \textbf{Q.26-Q.55 carry two marks each.}
    \item[26.] The application of Gram-Schmidt process of orthonormalization to $$u_1 = \brak{1,1,0}, u_2=\brak{1,0,0}, u_3 = \brak{1,1,1}$$ yields  \hfill (2011)
    \begin{enumerate}[label=(\Alph*)]
        \item $\frac{1}{\sqrt{2}}\brak{1,1,0},\brak{1,0,0},\brak{0,0,1}$
        \item $\frac{1}{\sqrt{2}}\brak{1,1,0},\frac{1}{\sqrt{2}}\brak{1,-1,0},\frac{1}{\sqrt{2}}\brak{1,1,1}$ 
        \item $\brak{0,1,0},\brak{1,0,0},\brak{0,0,1}$
        \item $\frac{1}{\sqrt{2}}\brak{1,1,0},\frac{1}{\sqrt{2}}\brak{1,-1,0},\brak{0,0,1}$ 
    \end{enumerate} 
\end{enumerate}
\end{document}