\let\negmedspace\undefined
\let\negthickspace\undefined
\documentclass[journal]{IEEEtran}
\usepackage[a5paper, margin=10mm, onecolumn]{geometry}
\usepackage{lmodern} % Ensure lmodern is loaded for pdflatex
 % Include tfrupee package
\setlength{\headheight}{1cm} % Set the height of the header box
\setlength{\headsep}{0mm}     % Set the distance between the header box and the top of the text
\usepackage{enumitem}
\usepackage{gvv-book}
\usepackage{gvv}
\usepackage{cite}
\usepackage{amsmath,amssymb,amsfonts,amsthm}
\usepackage{algorithmic}
\usepackage{graphicx}
\usepackage{textcomp}
\usepackage{xcolor}
\usepackage{txfonts}
\usepackage{listings}
\usepackage{enumitem}
\usepackage{mathtools}
\usepackage{gensymb}
\usepackage{graphicx}
\usepackage{wrapfig}
\usepackage{comment}
\usepackage[breaklinks=true]{hyperref}
\usepackage{tkz-euclide} 
\usepackage{listings}
% \usepackage{gvv}                                        
\def\inputGnumericTable{}                                 
\usepackage[latin1]{inputenc}                                
\usepackage{color}                                            
\usepackage{array}                                            
\usepackage{longtable}                                       
\usepackage{calc}                                             
\usepackage{multirow}                                         
\usepackage{hhline}                                           
\usepackage{ifthen}                                           
\usepackage{lscape}
\begin{document}

\bibliographystyle{IEEEtran}
\vspace{3cm}


\author{AI24BTECH11008- Sarvajith
}
\title{Assignment 7}
% \maketitle
% \newpage
% \bigskip
{\let\newpage\relax\maketitle}
\title{2017, EE}
\renewcommand{\thefigure}{\theenumi}
\renewcommand{\thetable}{\theenumi}
\setlength{\intextsep}{10pt} % Space between text and floats
\numberwithin{equation}{enumi}
\numberwithin{figure}{enumi}
\renewcommand{\thetable}{\theenumi}
\begin{enumerate}
    \item[14.] For the circuit shown in the figure below, assume that diodes $D_1, D_2$ and $D_3$ are ideal.\hfill (2017)
    \begin{figure}[!ht]
        \centering
        \resizebox{0.4\textwidth}{!}{%
    \begin{circuitikz}
    \tikzstyle{every node}=[font=\normalsize]
    \draw (5.5,10) to[D] (6.75,10);
    \draw (6.75,9.25) to[D] (5.5,9.25);
    \draw [short] (6.75,10) -- (6.75,9.25);
    \draw [short] (5.5,10) -- (5.5,9.25);
    \draw [short] (6.75,9.75) -- (8.75,9.75);
    \draw (8.75,9.75) to[R] (8.75,6.75);
    \draw (5.5,9.75) to[R] (3.25,9.75);
    \draw (3.25,9.75) to[sinusoidal voltage source, sources/symbol/rotate=auto] (3.25,6.75);
    \draw (7.75,6.75) to[D,l={ \normalsize $D_3$}] (7.75,9.75);
    \draw [short] (3.25,6.75) -- (8.75,6.75);
    \node [font=\LARGE] at (9.25,8.75) {+};
    \node [font=\LARGE] at (9.25,7.75) {-};
    \node [font=\normalsize] at (9.25,8.25) {$v_2$};
    \node [font=\normalsize] at (4.25,9.25) {$v_1$};
    \node [font=\Large] at (3.5,9.5) {+};
    \node [font=\Large] at (5,9.5) {-};
    \node [font=\normalsize] at (6,10.5) {$D_1$};
    \node [font=\normalsize] at (6.25,9) {$D_2$};
    \node [font=\normalsize] at (1.5,8.25) {$v(t) = \pi sin(100\pi t) V$};
    \end{circuitikz}
    }% % Specify the path to your TikZ file
        \caption{1}
        %\label{fig2}
    \end{figure}
    The DC componoents of voltages $v_1$ and $v_2$ respectively are
    \begin{enumerate}[label=(\Alph*)]
        \item 0 V and 1 V
        \item -0.5 V and 0.5 V
        \item 0 V and 0.5 V
        \item 1 V and 1 V
    \end{enumerate}
    \item[15.] For the power semiconductor devices IGBT, MOSFET, Diode and thyristor, which one of the following statements are true?\hfill (2017)
    \begin{enumerate}[label=(\Alph*)]
        \item All the four are majority carrier devices.
        \item All the four are minority carrier devices.
        \item IGBT and MOSFET are majority carrier devices, whereas Diode and Thyristor are minority carrier devices.
        \item MOSFET is majority carrier device, whereas IGBT, Diode, Thyristor are minority carrier devices.
    \end{enumerate}
    \item[16.] Consider $g\brak{t}=\begin{cases}t-\lfloor t \rfloor, & t\geq 0\\ t-\lceil t \rceil, & otherwise \end{cases}$ where $t \in R$. Here $\lfloor t \rfloor$ represents the largest integer less than or equal to t and  $\lceil t \rceil$ denotes the smallest integer greater than or equal to t. The coefficient of the second harmonic component of the Fourier series representing $g\brak{t}$ is?\hfill (2017)
    \item[17.] Let I= $c\iint_R xy^2 dxdy$, where R is the region shown in the figure and $c = 6\times 10^{-4}$. The value of I is equal to?(Give the answer upto 2 decimal places.)\hfill (2017)
    \begin{figure}[!ht]
        \centering
        \resizebox{0.2\textwidth}{!}{%
    \begin{circuitikz}
    \tikzstyle{every node}=[font=\normalsize]
    
    \draw [->, >=Stealth] (4.25,7.75) -- (4.25,12.75);
    \draw [->, >=Stealth] (4.25,7.75) -- (9.25,7.75);
    \draw [short] (5.25,7.75) -- (5.25,8.75);
    \draw [short] (5.25,8.75) -- (7.5,11.75);
    \draw [short] (7.5,11.75) -- (7.5,7.75);
    \draw [dashed] (4.25,8.75) -- (5.25,8.75);
    \draw [dashed] (4.25,11.75) -- (7.5,11.75);
    \node [font=\normalsize] at (4,11.75) {10};
    \node [font=\normalsize] at (4,8.75) {2};
    \node [font=\normalsize] at (6.25,9.25) {R};
    \node [font=\normalsize] at (5.25,7.5) {1};
    \node [font=\normalsize] at (7.5,7.5) {5};
    \end{circuitikz}
    }%  % Specify the path to your TikZ file
        \caption{2}
        %\label{fig2}
    \end{figure}
    \newpage
    \item[18.] The power supplied by the 25V source in the figure shown below is ...........W?\hfill (2017)
    \begin{figure}[!ht]
        \centering
        \resizebox{0.3\textwidth}{!}{%
    \begin{circuitikz}
    \tikzstyle{every node}=[font=\normalsize]
    \draw (3.25,10.5) to[R] (7,10.5);
    \draw (6.75,10.5) to[R] (6.75,8.5);
    \draw [short] (6.75,10.5) -- (8,10.5);
    \draw [short] (8,10.5) -- (8,9.75);
    \draw  (8.25,9.375) -- (8,9) -- (7.75,9.375) -- (8,9.75) -- cycle;
    \draw [short] (8,9) -- (8,8.5);
    \draw [short] (8,8.5) -- (3.25,8.5);
    \draw  (3.25,9.75) circle (0.25cm);
    \draw [short] (3.25,10.5) -- (3.25,10);
    \draw [short] (3.25,8.5) -- (3.25,9.5);
    \draw [->, >=Stealth] (8,9) -- (8,9.75);
    \node [font=\normalsize] at (8.5,9.5) {0.4I};
    \node [font=\normalsize] at (7,9.75) {$R_2$};
    \node [font=\normalsize] at (6,9.25) {14A};
    \node [font=\normalsize] at (5,10) {17V};
    \node [font=\normalsize] at (5,11) {$R_1$};
    \node [font=\normalsize] at (3.75,10.75) {I};
    \node [font=\normalsize] at (3.5,10) {+};
    \node [font=\normalsize] at (2.5,9.75) {25V};
    \node [font=\normalsize] at (3.5,9.5) {-};
    \node [font=\normalsize] at (4.5,10.25) {+};
    \node [font=\normalsize] at (5.75,10.25) {-};
    \node [font=\normalsize] at (6.5,10) {+};
    \node [font=\normalsize] at (6.5,9) {-};
    \draw [->, >=Stealth] (5.5,9.5) -- (5.5,9);
    \draw [short] (4,10.5) -- (3.75,10.75);
    \draw [short] (4,10.5) -- (3.75,10.25);
    \end{circuitikz}
    }%  % Specify the path to your TikZ file
        \caption{3}
        %\label{fig2}
    \end{figure}
    \item[19.] The equivalent resistance between the terminals A and B is .............$\Omega$.\hfill (2017)
    \begin{figure}[!ht]
        \centering
        \resizebox{0.3\textwidth}{!}{%
    \begin{circuitikz}
    \tikzstyle{every node}=[font=\normalsize]
    \draw (2.5,9.5) to[R] (4.5,9.5);
    \draw (4.5,9.5) to[R] (7.75,9.5);
    \draw (4.75,9.5) to[R] (4.75,6.25);
    \draw (4.75,9.5) to[R] (6.5,7.75);
    \draw (4.75,6.25) to[R] (6,7.5);
    \draw (7.75,9.5) to[R] (7.75,6.25);
    \draw (7.75,9.5) to[R] (11,9.5);
    \draw (11,9.5) to[R] (11,6.25);
    \draw (4.75,6.25) to[R] (2.5,6.25);
    \draw [short] (6,7.5) .. controls (6.5,7.5) and (6.5,7.5) .. (6.5,8);
    \draw [short] (6.5,8) -- (7.75,9.5);
    \draw [short] (4.75,6.25) -- (6.75,6.25);
    \draw [short] (4.75,6.25) -- (7.75,6.25);
    \draw [short] (6.5,7.75) -- (7.75,6.25);
    \draw [short] (7.75,6.25) -- (11,6.25);
    \draw (2.5,9.5) to[short, -o] (1.75,9.5) ;
    \draw (2.5,6.25) to[short, -o] (1.75,6.25) ;
    \node [font=\normalsize] at (3.5,10) {1$\Omega$};
    \node [font=\normalsize] at (6,9.75) {2$\Omega$};
    \node [font=\normalsize] at (9.25,9.75) {1$\Omega$};
    \node [font=\normalsize] at (10.5,8) {1$\Omega$};
    \node [font=\normalsize] at (8.25,8) {6$\Omega$};
    \node [font=\normalsize] at (6,9) {6$\Omega$};
    \node [font=\normalsize] at (5.75,6.5) {3$\Omega$};
    \node [font=\normalsize] at (4.25,8) {3$\Omega$};
    \node [font=\normalsize] at (3.5,5.75) {0.8$\Omega$};
    \end{circuitikz}
    }% % Specify the path to your TikZ file
        \caption{4}
        %\label{fig2}
    \end{figure}
    \item[20.] A three-phase, 50Hz, star connected cylindrical-rotor synchronus machine is running as a motor. The machine is operated from a 6.6kV grid and draws current at unity power factor(UPF). The synchronus reactance of the motor is $30\Omega$ per phase. The load angle is $30^{\circ}$. The power delivered to the motor in kW is ............(Give the answer up to one decimal place).\hfill (2017)
    \item[21.] A bus power system consists of four generator buses indexed as G1, G2, G3, G4 and six load buses indexed as L1, L2, L3,L4, L5 and L6. The generator-bus G1 is considered as slack bus, and the load buses are L3 and L4 are voltage controlled buses. The generator at bus G2 cannot supply the required reactive power demand, and hence it is operating at its maximum reactive power limit. The number of non-linear equations required for solving the load flow problem using Newton-Raphson method in polar form is ...........\hfill (2017)
    \item[22.] Consider the unity feedback control system shown. The value of K that results in a phase margin of the system to be $30^{\circ}$............(give the answer up to 2 decimal places.)\hfill (2017)
    \begin{figure}[!ht]
        \centering
        \resizebox{0.4\textwidth}{!}{%
    \begin{circuitikz}
    \tikzstyle{every node}=[font=\normalsize]
    \draw  (5,9.75) rectangle (7.5,8.25);
    \draw [->, >=Stealth] (7.5,9) -- (9.75,9);
    \draw [->, >=Stealth] (3,9) -- (5,9);
    \draw  (2.75,9) circle (0.25cm);
    \draw [->, >=Stealth] (1.25,9) -- (2.5,9);
    \draw [short] (2.5,9.25) -- (3,8.75);
    \draw [short] (2.5,8.75) -- (3,9.25);
    \draw [->, >=Stealth] (2.75,6.75) -- (2.75,8.75);
    \draw [short] (2.75,6.75) -- (8.75,6.75);
    \draw [short] (8.75,6.75) -- (8.75,9);
    \node [font=\normalsize] at (6.25,9) {$\frac{Ke^{-s}}{s}$};
    \node [font=\normalsize] at (1.75,9.5) {$U(s)$};
    \node [font=\normalsize] at (2.5,9.5) {$+$};
    \node [font=\normalsize] at (2.5,8.75) {$-$};
    \node [font=\normalsize] at (9,9.5) {$Y(s)$};
    \end{circuitikz}
    }% % Specify the path to your TikZ file
        \caption{5}
        %\label{fig2}
    \end{figure}
    \item[23.] The following measurements are obtained on a single phase load: $V = 220 V\pm 1\%, I = 5.0A\pm 1\%$ and $W = 555W \pm 2\%$. If the power factor is calculates using these measurements, the worst case error in the calculated power factor in percent is ..........(Give answer up to one decimal place.) \hfill (2017)
    \item[24.] In the converter circuit shown below, the switches are controlled such that the load voltage $v_0\brak{t}$ is a 400 Hz square wave.
    \begin{figure}[!ht]
        \centering
        \resizebox{0.4\textwidth}{!}{%
    \begin{circuitikz}
    \tikzstyle{every node}=[font=\normalsize]
    \draw [short] (3.5,10.25) -- (9.25,10.25);
    \draw [short] (3.5,10.25) -- (3.5,8.5);
    \draw  (3.5,8.25) circle (0.25cm);
    \draw [short] (3.5,8) -- (3.5,6.25);
    \draw [short] (3.5,6.25) -- (5.25,6.25);
    \draw [short] (5.25,10.25) -- (5.25,9.5);
    \draw [short] (5.25,8.75) -- (5.25,7);
    \draw [short] (5.25,8.25) -- (6.25,8.25);
    \draw [short] (8,8.25) -- (9.25,8.25);
    \draw [short] (9.25,10.25) -- (9.25,9.5);
    \draw [short] (9.25,8.5) -- (9.25,7.25);
    \draw [short] (5.25,6.25) -- (5.25,6.5);
    \draw [short] (5.25,6.25) -- (9.25,6.25);
    \draw [short] (9.25,6.25) -- (9.25,6.5);
    \draw  (6.25,8.5) rectangle (8,8);
    \draw [->, >=Stealth] (5.25,9.5) -- (5.75,9);
    \draw [->, >=Stealth] (9.25,9.5) -- (9.75,9);
    \draw [->, >=Stealth] (9.25,7.25) -- (9.75,6.75);
    \draw [->, >=Stealth] (5.25,7) -- (5.75,6.5);
    \node [font=\normalsize] at (7,8.25) {$Load$};
    \node [font=\normalsize] at (5,9.25) {$S_1$};
    \node [font=\normalsize] at (9,7) {$S_2$};
    \node [font=\normalsize] at (9,9.25) {$S_3$};
    \node [font=\normalsize] at (5,6.75) {$S_4$};
    \node [font=\normalsize] at (7,7.75) {$v_0(t)$};
    \node [font=\normalsize] at (6,7.75) {$+$};
    \node [font=\normalsize] at (8.25,7.75) {$-$};
    \node [font=\normalsize] at (3.75,8.75) {$+$};
    \node [font=\normalsize] at (3.75,7.75) {$-$};
    \node [font=\normalsize] at (2.75,8.25) {$220V$};
    \end{circuitikz}
    }%  % Specify the path to your TikZ file
        \caption{6}
        %\label{fig2}
    \end{figure}
    The RMS value of the fundamental component of $v_0\brak{t}$ in volts is ................
    \item[25.] A 3-phase voltage source inverter is supplied from a 600V DC source as shown in the figure below. For a star connected resistive load of $20\Omega$ per phase, the load power for $120^{\circ}$ device connection. in kW is ................\hfill (2017)
    \begin{figure}[!ht]
        \centering
        \resizebox{0.4\textwidth}{!}{%
        \begin{circuitikz}
        \tikzstyle{every node}=[font=\normalsize]
        
        
        \draw [short] (1.75,10.25) -- (7.75,10.25);
        \draw [short] (1.75,10.25) -- (1.75,8.75);
        \draw  (1.75,8.5) circle (0.25cm);
        \draw [short] (1.75,8.25) -- (1.75,5.75);
        \draw [short] (1.75,5.75) -- (4.5,5.75);
        \draw (4,8.25) to[Tnpn, transistors/scale=1.19] (4,10.25);
        \draw (4,5.75) to[Tnpn, transistors/scale=1.19] (4,8.25);
        \draw (5.75,8) to[Tnpn, transistors/scale=1.19] (5.75,10.25);
        \draw (7.75,8.25) to[Tnpn, transistors/scale=1.19] (7.75,10.25);
        \draw (7.75,5.75) to[Tnpn, transistors/scale=1.19] (7.75,8.25);
        \draw (5.75,5.75) to[Tnpn, transistors/scale=1.19] (5.75,8);
        \draw (4.25,8.5) to[D] (4.25,9.75);
        \draw (4.25,6.5) to[D] (4.25,7.75);
        \draw (6,8.5) to[D] (6,9.75);
        \draw (6,6.25) to[D] (6,7.5);
        \draw (8,8.75) to[D] (8,10);
        \draw (8,6.5) to[D] (8,7.5);
        \draw [short] (4,9.75) -- (4.25,9.75);
        \draw [short] (4,8.5) -- (4.25,8.5);
        \draw [short] (4,7.75) -- (4.25,7.75);
        \draw [short] (4,6.5) -- (4.25,6.5);
        \draw [short] (5.75,9.75) -- (6,9.75);
        \draw [short] (5.75,8.5) -- (6,8.5);
        \draw [short] (5.75,7.5) -- (6,7.5);
        \draw [short] (5.75,6.25) -- (6,6.25);
        \draw [short] (7.75,6.5) -- (8,6.5);
        \draw [short] (7.75,7.5) -- (8,7.5);
        \draw [short] (7.75,8.75) -- (8,8.75);
        \draw [short] (7.75,10) -- (8,10);
        \draw [short] (4.5,5.75) -- (7.75,5.75);
        \draw (12,8.25) to[R] (12,10);
        \draw (12,8.25) to[R,l={ \normalsize 20$\Omega$}] (10.75,7);
        \draw (12,8.25) to[R,l={ \normalsize 20$\Omega$}] (13.25,7);
        \draw [short] (12,10) -- (10.5,10);
        \draw [short] (10.5,10) -- (10.5,8.25);
        \draw [short] (4,8) -- (5.5,8);
        \draw [short] (6,8) -- (7.5,8);
        \draw [short] (8,8) -- (10.5,8.25);
        \draw [short] (7.5,8) .. controls (7,8.5) and (8.25,8.5) .. (8,8);
        \draw [short] (5.5,8) .. controls (5,8.25) and (6.25,8.5) .. (6,8);
        \draw [short] (5.75,7.75) -- (7.5,7.75);
        \draw [short] (7.5,7.75) .. controls (7,8.25) and (8.5,8.25) .. (8,7.75);
        \draw [short] (8,7.75) -- (10.5,7.75);
        \draw [short] (10.5,7.75) -- (10.75,7);
        \draw [short] (7.75,7.75) -- (10.25,6.25);
        \draw [short] (10.25,6.25) -- (13.25,7);
        \node [font=\normalsize] at (12.5,9.25) {20$\Omega$};
        \node [font=\normalsize] at (1.5,8.75) {+};
        \node [font=\normalsize] at (1.5,8.25) {-};
        \node [font=\normalsize] at (1,8.5) {600V};
        \end{circuitikz}
        }%  % Specify the path to your TikZ file
        \caption{7}
        %\label{fig2}
    \end{figure}
    \item[26.] A function $f\brak{x}$ is defined as $f\brak{x} = \begin{cases}e^x, & x<1\\ \ln x+ax^2+bx &x\geq 1\end{cases}$ where $x \in R$. Which one of the following statements is TRUE?\hfill (2017)
    \begin{enumerate}[label = (\Alph*)]
        \item $f\brak{x}$ is NOT differentiable at x = 1 for any values of a and b.
        \item $f\brak{x}$ is differentiable at x = 1 for unique values of a and b.
        \item $f\brak{x}$ is differentiable at x = 1 for all values of a and b such that a+b = e.
        \item $f\brak{x}$ is differentiable at x = 1 for all values of a and b.
    \end{enumerate}
\end{enumerate}
\end{document}