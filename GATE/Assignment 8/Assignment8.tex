\let\negmedspace\undefined
\let\negthickspace\undefined
\documentclass[journal]{IEEEtran}
\usepackage[a5paper, margin=10mm, onecolumn]{geometry}
\usepackage{lmodern} % Ensure lmodern is loaded for pdflatex
 % Include tfrupee package
\setlength{\headheight}{1cm} % Set the height of the header box
\setlength{\headsep}{0mm}     % Set the distance between the header box and the top of the text
\usepackage{enumitem}
\usepackage{gvv-book}
\usepackage{gvv}
\usepackage{cite}
\usepackage{amsmath,amssymb,amsfonts,amsthm}
\usepackage{algorithmic}
\usepackage{graphicx}
\usepackage{textcomp}
\usepackage{xcolor}
\usepackage{wrapfig}
\usepackage{txfonts}
\usepackage{listings}
\usepackage{enumitem}
\usepackage{mathtools}
\usepackage{gensymb}
\usepackage{graphicx}
\usepackage{wrapfig}
\usepackage{comment}
\usepackage[breaklinks=true]{hyperref}
\usepackage{tkz-euclide} 
\usepackage{listings}
\usepackage{gvv}                                        
\def\inputGnumericTable{}                                 
\usepackage[latin1]{inputenc}                                
\usepackage{color}                                            
\usepackage{array}                                            
\usepackage{longtable}                                       
\usepackage{calc}   
\usepackage{multicol}                                          
\usepackage{multirow}                                         
\usepackage{hhline}                                           
\usepackage{ifthen}                                           
\usepackage{lscape}
\begin{document}

\bibliographystyle{IEEEtran}
\vspace{3cm}


\author{AI24BTECH11008- Sarvajith
}
\title{Assignment 8}
 \maketitle
 %\newpage
 %\bigskip
{\let\newpage\relax\maketitle}
\title{2008, PH}
\renewcommand{\thefigure}{\theenumi}
\renewcommand{\thetable}{\theenumi}
\setlength{\intextsep}{10pt} % Space between text and floats
\numberwithin{equation}{enumi}
\numberwithin{figure}{enumi}
\renewcommand{\thetable}{\theenumi}
\begin{enumerate}
    \item[69.] The following circuit where(where $R_L>>>R$) performs the operation of \hfill (2008)
    \begin{figure}[!ht]
    \centering
    
    \resizebox{0.3\textwidth}{!}{%
    \begin{circuitikz}
    \tikzstyle{every node}=[font=\normalsize]
    \draw [short] (3.75,10.25) -- (3.75,7.75);
    \draw [short] (3.75,7.75) -- (7.25,7.75);
    \draw [short] (4,8.75) -- (7,8.75);
    \node [font=\normalsize] at (7.25,7.5) {B};
    \node [font=\normalsize] at (3.25,10) {$\rho_{xy}$};
    \end{circuitikz}
    }%
    
    % Specify the path to your TikZ file
    \caption{1}
    \label{fig1}
     \end{figure}
    \begin{enumerate} [label=(\Alph*)]
        \item OR gate for a negative logic system
        \item NAND gate for a negative logic system 
        \item AND gate for a positive logic system
        \item AND gate for a negative logic system
    \end{enumerate}
    \item[70.]  In the T type master-slave $J_K$ flip flop is shown along with the clock and input waveforms. The $Q_n$ output of flip flop was zero initially. Identify the correct output waveform. \hfill (2008)
     
     \begin{figure}[!ht]
      \centering
      
    \resizebox{0.3\textwidth}{!}{%
    \begin{circuitikz}
    \tikzstyle{every node}=[font=\normalsize]
    
    \draw (-5.5,13.25) to[R] (-3.5,13.25);
    \draw (-2,13.75) node[op amp,scale=1, yscale=-1 ] (opamp2) {};
    \draw (opamp2.+) to[short] (-3.5,14.25);
    \draw  (opamp2.-) to[short] (-3.5,13.25);
    \draw (-0.8,13.75) to[short](-0.5,13.75);
    \draw (-3.5,14.25) to (-4,14.25) node[ground]{};
    \draw [short] (-3,13.25) -- (-3,12.25);
    \draw [short] (-1,13.75) -- (-1,12.25);
    \draw (-3,12.25) to[empty Schottky diode] (-1,12.25);
    \node [font=\normalsize] at (-2,11.75) {5.1V};
    \end{circuitikz}
    }%
    
     % Specify the path to your TikZ file
      \caption{ 2}
      \label{fig2}
     \end{figure}
    
     \begin{figure}[!ht]
      \centering
      
    \resizebox{0.4\textwidth}{!}{%
    \begin{circuitikz}
    \tikzstyle{every node}=[font=\normalsize]
    
    \draw (6.5,12.25) node[ieeestd not port, anchor=in](port){} (port.out) to[short] (8.25,12.25);
    \draw (port.in) to[short] (6.25,12.25);
    \draw (6.5,11) node[ieeestd not port, anchor=in](port){} (port.out) to[short] (8.25,11);
    \draw (port.in) to[short] (6.25,11);
    \draw (6.5,9.75) node[ieeestd not port, anchor=in](port){} (port.out) to[short] (8.25,9.75);
    \draw (port.in) to[short] (6.25,9.75);
    \draw [short] (8.25,12.25) -- (8.75,12);
    \draw [short] (8.25,11) -- (8.75,11.5);
    \draw [short] (8,9.75) -- (11.5,9.75);
    \draw [short] (10.75,11.75) -- (11.5,11.75);
    \draw [short] (11.5,11.75) -- (13,10.75);
    \draw [short] (11.5,9.75) -- (13,10.25);
    \draw [short] (15,10.5) -- (16,10.5);
    \node [font=\normalsize] at (6,12.25) {P};
    \node [font=\normalsize] at (6,11) {R};
    \node [font=\normalsize] at (6,9.75) {Q};
    \node [font=\normalsize] at (16.5,10.5) {Y};
    \draw (8.75,12) to[short] (9,12);
    \draw (8.75,11.5) to[short] (9,11.5);
    \draw (9,12) node[ieeestd and port, anchor=in 1, scale=0.89](port){} (port.out) to[short] (10.75,11.75);
    \draw (13,10.75) to[short] (13.25,10.75);
    \draw (13,10.25) to[short] (13.25,10.25);
    \draw (13.25,10.75) node[ieeestd or port, anchor=in 1, scale=0.89](port){} (port.out) to[short] (15,10.5);
    \end{circuitikz}
    }%
    
    % Specify the path to your TikZ file
      %\caption{ 2}
      %\label{fig2}
     \end{figure}
        \begin{figure}[!ht]
          \centering
          
    \resizebox{0.3\textwidth}{!}{%
    \begin{circuitikz}
    \tikzstyle{every node}=[font=\normalsize]
    \draw (2.5,9.5) to[R] (4.5,9.5);
    \draw (4.5,9.5) to[R] (7.75,9.5);
    \draw (4.75,9.5) to[R] (4.75,6.25);
    \draw (4.75,9.5) to[R] (6.5,7.75);
    \draw (4.75,6.25) to[R] (6,7.5);
    \draw (7.75,9.5) to[R] (7.75,6.25);
    \draw (7.75,9.5) to[R] (11,9.5);
    \draw (11,9.5) to[R] (11,6.25);
    \draw (4.75,6.25) to[R] (2.5,6.25);
    \draw [short] (6,7.5) .. controls (6.5,7.5) and (6.5,7.5) .. (6.5,8);
    \draw [short] (6.5,8) -- (7.75,9.5);
    \draw [short] (4.75,6.25) -- (6.75,6.25);
    \draw [short] (4.75,6.25) -- (7.75,6.25);
    \draw [short] (6.5,7.75) -- (7.75,6.25);
    \draw [short] (7.75,6.25) -- (11,6.25);
    \draw (2.5,9.5) to[short, -o] (1.75,9.5) ;
    \draw (2.5,6.25) to[short, -o] (1.75,6.25) ;
    \node [font=\normalsize] at (3.5,10) {1$\Omega$};
    \node [font=\normalsize] at (6,9.75) {2$\Omega$};
    \node [font=\normalsize] at (9.25,9.75) {1$\Omega$};
    \node [font=\normalsize] at (10.5,8) {1$\Omega$};
    \node [font=\normalsize] at (8.25,8) {6$\Omega$};
    \node [font=\normalsize] at (6,9) {6$\Omega$};
    \node [font=\normalsize] at (5.75,6.5) {3$\Omega$};
    \node [font=\normalsize] at (4.25,8) {3$\Omega$};
    \node [font=\normalsize] at (3.5,5.75) {0.8$\Omega$};
    \end{circuitikz}
    }%
   % Specify the path to your TikZ file
       %   \caption{ 4}
         % \label{fig 4}
            \end{figure}
            
      \begin{figure}[!ht]
        \centering
        
    \resizebox{0.3\textwidth}{!}{%
    \begin{circuitikz}
    \tikzstyle{every node}=[font=\normalsize]
    
   
    \draw [short] (9.5,9.5) -- (12.5,9.5);
    
    \draw [short] (12.5,9.5) -- (12.5,10.25);
    \draw [short] (12.5,10.25) -- (13,10.25);
    \draw [short] (13,10.25) -- (13,9.5);
    \draw [short] (13,9.5) -- (14.5,9.5);
    \draw [short] (14.5,9.5) -- (14.75,9.5);
    \end{circuitikz}
    }%
    
    % Specify the path to your TikZ file
        %\caption{ 5}
        %\label{fig 5}
    \end{figure}
    \begin{figure}[!ht]
      \centering
      
    \resizebox{0.3\textwidth}{!}{%
    \begin{circuitikz}
    \tikzstyle{every node}=[font=\normalsize]
    \draw [short] (3.75,10.25) -- (3.75,7.75);
    \draw [short] (3.75,7.75) -- (7.25,7.75);
    \node [font=\normalsize] at (3.75,10.5) {$\frac{C}{T}$};
    \node [font=\normalsize] at (7.5,7.75) {$T^2$};
    \draw [short] (3.75,8.25) -- (6.5,10);
    \end{circuitikz}
    }%
     % Specify the path to your TikZ file
      %\caption{ 6}
      %\label{fig 6}
       \end{figure}
      
   
    \section{Common Data Questions}
    \textbf{Common data for questions 71, 72 and 73}: A beam of identical particles of mass m and energy E is incident from left on a potential barrier of width L (between $0 < x < L$) and height $V_0$ as shown in the figure $\brak{E < V_0}$
    \begin{figure}[!ht]
      \centering
      
    \resizebox{0.4\textwidth}{!}{%
    \begin{circuitikz}
    \tikzstyle{every node}=[font=\normalsize]
    \draw [->, >=Stealth] (10.5,9.25) -- (18.25,9.25);
    \draw [->, >=Stealth] (13.75,9.25) -- (13.75,13.25);
    \draw [<->, >=Stealth] (13.75,9) -- (14.5,9)node[pos=0.5, fill=white]{L};
    \draw [<->, >=Stealth] (12.75,10.25) -- (12.75,9.25);
    \draw [<->, >=Stealth] (14,11) -- (14,9.25)node[pos=0.5, fill=white]{$V_0$};
    \draw  (13.75,11) rectangle (14.5,9.25);
    \draw [dashed] (10.75,10.25) -- (13.75,10.25);
    \node [font=\normalsize] at (18.5,9.25) {x};
    \node [font=\normalsize] at (13.5,13.25) {V(x)};
    \node [font=\normalsize] at (13.5,9) {0};
    \end{circuitikz}
    }%
    
      % Specify the path to your TikZ file
      \caption{ 7}
      \label{fig 7}
  \end{figure}
    For x > L there is tunneling with a transmission coefficient $T > 0$. Let $A_0$, AB and AT denote the amplitudes for the incident, reflected and the transmitted waves, respectively.
    \item[71.] Throughout $0 < x < L$, the wave-function \hfill (2008)
      \begin{enumerate}[label=(\Alph*)]
        \item can be chosen to be real
        \item is exponentially decaying
        \item is generally complex
        \item  is zero
      \end{enumerate}
    \item[72.]  Let the probability current associated with the incident wave be $S_0$. Let R be the reflection coefficient. Then\hfill (2008)
      \begin{enumerate}[label=(\Alph*)]
        \item the probability current vanishes in the classically forbidden region
        \item the probability current is $TS_0$ for $x > L$
        \item for, $x < 0$ the probability current is $S_0\brak{1 + R}$
        \item  for $x > L$, the probability current is complex
      \end{enumerate}
    \item[73.] The ratio of the reflected to the incident amplitude $\frac{A_B}{A_0}$ is \hfill (2008)
      \begin{enumerate}[label = (\Alph*)]
        \item $1-\frac{A_T}{A_0}$
        \item $\sqrt{\brak{1-T}}$ in magnitude
        \item a real negative number
        \item $\sqrt{\brak{1-\abs{\frac{A_T}{A_0}}^2}\frac{E}{V_0 - E}}$
      \end{enumerate}
    \textbf{Common Data for questions 74 and 75}: Consider two concentric conducting spherical shells with inner and outer radii a, b and c, d as shown in the figure. Both the shells are given Q amount of positive charges.
    \begin{figure}[!ht]
      \centering
      
    \resizebox{0.3\textwidth}{!}{%
    \begin{circuitikz}
    \tikzstyle{every node}=[font=\normalsize]
    \draw [short] (3.75,10.25) -- (3.75,7.75);
    \draw [short] (3.75,7.75) -- (7.25,7.75);
    \node [font=\normalsize] at (3.75,10.5) {$\frac{C}{T}$};
    \node [font=\normalsize] at (7.5,7.75) {$T^2$};
    \draw [short] (3.75,7.75) -- (6.25,10.25);
    \end{circuitikz}
    }%
      % Specify the path to your TikZ file
      \caption{ 8}
      \label{fig 8}
  \end{figure}
    \item[74.] The electric field in different regions are \hfill (2008)
      \begin{enumerate}[label=(\Alph*)]
        \item $\bar{E} = 0$ for $r < a$; $\bar{E} = \frac{-Q}{4\pi\epsilon_0r^2}\hat{r}$ for $a<r<b$;$\bar{E} = 0$ for $b< r < c$; $\bar{E} = \frac{Q}{4\pi\epsilon_0r^2}\hat{r}$ for $r>d$
        \item $\bar{E} = 0$ for $a < r < b$; $\bar{E} = \frac{-Q}{4\pi\epsilon_0r^2}\hat{r}$ for $r<a$; $\bar{E} = \frac{-Q}{4\pi\epsilon_0r^2}\hat{r}$ for $b< r < c$; $\bar{E} = \frac{Q}{4\pi\epsilon_0r^2}\hat{r}$ for $r>d$ 
        \item $\bar{E} = \frac{-Q}{4\pi\epsilon_0r^2}\hat{r}$ for $r < a$; $\bar{E}$ = 0 for $a<r<b$;$\bar{E} = 0$ for $b< r < c$; $\bar{E} = \frac{2Q}{4\pi\epsilon_0r^2}\hat{r}$ for $r>d$ 
        \item $\bar{E} = 0$ for $r < a$; $\bar{E}$ = 0 for $a<r<b$;$\bar{E} = frac{Q}{4\pi\epsilon_0r^2}\hat{r}$ for $b< r < c$; $\bar{E} = \frac{2Q}{4\pi\epsilon_0r^2}\hat{r}$ for $r>d $ 
      \end{enumerate}
    \item[75.] In order to have equal surface charge densities on the outer surfaces of both the shells, the following conditions should be satisfied \hfill (2008)
     \begin{enumerate}[label=(\Alph*)]
        \item d = 4b and c = 2a
        \item d = 2b and $\sqrt{2}$a
        \item d = $\sqrt{2}$b and $c>a$
        \item $d>b$ and c = $\sqrt{2}$a 
     \end{enumerate}
    \section{Linked Answer Questions:} Q.76 to Q.85 carry 2 marks each.
    \textbf{Statement for linked answer questions 76 and 77}
    Consider the $\beta$-decay of a free neutron at rest in the laboratory.
    \item[76.] Which of the following configurations of the decay products corresponds to the largest energy of the anti-neutrino $\bar{\nu}$? (rest mass of electron $m_e$ = 0.51 $\frac{MeV}{c^2}$, rest mass of proton $m_p$ = 938.27 $\frac{MeV}{c^2}$ and rest mass of neutron mn = 939.57 $\frac{MeV}{c^2}$) \hfill (2008)
      \begin{enumerate}[label=(\Alph*)]
        \item In the laboratory, proton is produced at rest
        \item In the laboratory, momenta of proton, electron and the anti-neutrino all have the same magnitude
        \item In the laboratory, proton and electron fly-off with (nearly) equal and opposite momenta
        \item In the laboratory, electron is produced at rest
      \end{enumerate}
    \item[77.] Using the result of the above problem, answer the following. Which of the following represents approximately the maximum allowed energy of the anti-neutrino $\bar{\nu}$? \hfill (2008)
    \begin{enumerate}[label=(\Alph*)]
        \item 1.3 MeV
        \item 0.8 MeV
        \item 0.5 MeV
        \item 2 MeV
    \end{enumerate}
    \textbf{Statement for Linked Answer Questions 78 and 79 }
    Consider a two dimensional electron gas of N electrons of mass m each in a system of size $L\times L$.
    \item[78.] The density of states between energy $\epsilon \text{and} \epsilon+d\epsilon$ is  \hfill (2008)
    \begin{enumerate}[label=(\Alph*)]
        \item $\frac{4\pi L^2 m}{h^2}d\epsilon$
        \item  $\frac{4\pi L^2 m}{h^2}\frac{1}{\sqrt{\epsilon}} d\epsilon$
        \item  $\frac{4\pi L^2 m}{h^2}\sqrt{\epsilon}d\epsilon$
        \item  $\frac{4\pi L^2 m}{h^2}\epsilon d\epsilon$
    \end{enumerate}
    \item[79.]  The ground state energy $E_0$ of the system in terms of the Fermi energy $E_F$ and the number of electrons N is given by \hfill (2008)
    \begin{enumerate}[label=(\Alph*)]
        \item $\frac{1}{3}NE_F$
        \item $\frac{1}{2}NE_F$
        \item $\frac{2}{3}NE_F$
        \item $\frac{3}{5}NE_F$
    \end{enumerate}
    \textbf{Statement for linked answer questions 80 and 81}:
    The rate of a clock in a spaceship "Suryashakti" is observed from each to be $\frac{3}{5}$ of the rate of the clocks on earth.
    \item[80.]  The speed of the spaceship "Suryashakti" relative to earth is \hfill (2008)
    \begin{enumerate}[label=(\Alph*)]
        \item $\frac{4}{5}c$
        \item $\frac{3}{5}c$
        \item $\frac{9}{10}c$
        \item $\frac{2}{5}c$
    \end{enumerate}
    \item[81.] The rate of a clock in a spaceship "Aakashganga" is observed from earth to be $\frac{5}{13}$ of the rate of the clocks on earth. If both Aakashganga and Suryashakti are moving in the same direction relative to someone on earth, then the speed of Aakashganga relative to Suryashakti is \hfill (2008)
    \begin{enumerate}[label=(\Alph*)]
        \item $\frac{12}{13}c$
        \item $\frac{4}{5}c$
        \item $\frac{8}{17}c$
        \item $\frac{5}{6}c$
    \end{enumerate}
    \textbf{Statement for Linked Answer Questions 82 and 83:}
    The following circuit contains three operational amplifiers and resistors \hfill (2008)
    \begin{figure}[!ht]
      \centering
      
    \resizebox{0.4\textwidth}{!}{%
    \begin{circuitikz}
    \tikzstyle{every node}=[font=\normalsize]
    \draw (10,8.5) node[op amp,scale=1, yscale=-1 ] (opamp2) {};
    \draw (opamp2.+) to[short] (8.5,9);
    \draw  (opamp2.-) to[short] (8.5,8);
    \draw (11.2,8.5) to[short](11.5,8.5);
    \draw (5.5,9) node[op amp,scale=1] (opamp2) {};
    \draw (opamp2.+) to[short] (4,8.5);
    \draw  (opamp2.-) to[short] (4,9.5);
    \draw (6.7,9) to[short](7,9);
    \draw (2,10.25) to[R] (4.75,10.25);
    \draw (4.75,10.25) to[R] (7.5,10.25);
    \draw (7,9) to[short] (8.75,9);
    \draw [short] (4,10.25) -- (4,9.5);
    \draw (4,8.5) to[R] (4,6.5);
    \draw [short] (7.5,10.25) -- (7.5,9);
    \draw (8.5,8) to[R] (8.5,4.5);
    \draw (8.5,7) to[R] (11.25,7);
    \draw [short] (11.25,8.5) -- (11.25,7);
    \draw (8.5,4.5) to (10.5,4.5) node[ground]{};
    \draw (10.25,8.75) to[short, -o] (10.25,9.5) ;
    \draw (10.25,8.25) to[short, -o] (10.25,7.5) ;
    \draw (5.75,8.75) to[short, -o] (5.75,8) ;
    \draw (5.75,9.25) to[short, -o] (5.75,9.75) ;
    \node [font=\normalsize] at (1.75,10.25) {-2V};
    \node [font=\normalsize] at (6,10.75) {1k$\Omega$};
    \node [font=\normalsize] at (3.25,10.5) {1k$\Omega$};
    \node [font=\normalsize] at (10.25,9.75) {+15V};
    \node [font=\normalsize] at (10.5,7.25) {-15V};
    \node [font=\normalsize] at (9.75,6.5) {1k$\Omega$};
    \node [font=\normalsize] at (8,6.25) {1k$\Omega$};
    \node [font=\normalsize] at (6,9.75) {+15V};
    \node [font=\normalsize] at (5.75,7.75) {-15V};
    \node [font=\normalsize] at (3.25,7.5) {1k$\Omega$};
    \node [font=\normalsize] at (11.5,8.75) {$V_{out}$};
    \node [font=\normalsize] at (4,6.25) {+1V};
    \end{circuitikz}
    }%
    
     
      \caption{9} 
      \label{fig9} 
  \end{figure}
  
    \item[82.] The output voltage at the end of second operational amplifier $V_{01}$ is \hfill (2008)
    \begin{enumerate}[label=(\Alph*)]
        \item $V_{01}=3\brak{V_a+V_b+V_c}$
        \item $V_{01}=-\frac{1}{3}\brak{V_a+V_b+V_c}$
        \item $V_{01}=\frac{1}{3}\brak{V_a+V_b+V_c}$
        \item $V_{01}=\frac{4}{3}\brak{V_a+V_b+V_c}$
    \end{enumerate}
    \item[83.] The output $V_{02}$ (at the end of third op amp) of the above circuit is \hfill (2008)
    \begin{enumerate}[label=(\Alph*)]
        \item $V_{02}=2\brak{V_a+V_b+V_c}$
        \item $V_{02}=3\brak{V_a+V_b+V_c}$
        \item $V_{02}=-\frac{1}{2}\brak{V_a+V_b+V_c}$
        \item Zero
    \end{enumerate}
    \textbf{Statement for Linked Answer Questions 84 and 85}
    The set V of all polynomials of a real variable x of degree two or less and with real coefficients, constitutes a real linear vector space $V=\{c_0 + c_1x + c_2 x^2 : c_0, c_1, c_2 \in R\}$.
    \item [84.] For $f \brak{x} = a_0 + a_1x + ax^2 \in V$ and $g\brak{x}=b_0+b_1x+b_2x^2 \in V$, which one of the follwoing constitutes an acceptable scalar product? \hfill (2008)
    \begin{enumerate}[label=(\Alph*)]
        \item $\brak{f,g} = a_0^2b_0+a_1^2b_1+a_2^2b_2$
        \item $\brak{f,g} = a_0^2b_0^2+a_1^2b_1^2+a_2^2b_2^2$
        \item $\brak{f,g} = a_0b_0-a_1b_1+a_2b_2$
        \item $\brak{f,g} = a_0^2b_0+\frac{a_1b_1}{2}+\frac{a_2b_2}{3}$
    \end{enumerate}
    \item[85.] Using the scalar product obtained in the above question, identify the subspace of V that is orthogonal to \brak{1 + x}: \hfill (2008)
    \begin{enumerate}[label=(\Alph*)]
        \item $\bigl\{f\brak{x}:b\brak{1-x}+cx^2;b,c\in R\bigr\}$
        \item  $\bigl\{f\brak{x}:b\brak{1-2x}+cx^2;b,c\in R\bigr\}$
        \item  $\bigl\{f\brak{x}:b+cx^2;b,c\in R\bigr\}$
        \item  $\bigl\{f\brak{x}:bx+cx^2;b,c\in R\bigr\}$
    \end{enumerate}
\end{enumerate}
\end{document}