\let\negmedspace\undefined
\let\negthickspace\undefined
\documentclass[journal]{IEEEtran}
\usepackage[a5paper, margin=10mm, onecolumn]{geometry}
\usepackage{lmodern} % Ensure lmodern is loaded for pdflatex
 % Include tfrupee package
\setlength{\headheight}{1cm} % Set the height of the header box
\setlength{\headsep}{0mm}     % Set the distance between the header box and the top of the text
\usepackage{enumitem}
\usepackage{gvv-book}
\usepackage{gvv}
\usepackage{cite}
\usepackage{amsmath,amssymb,amsfonts,amsthm}
\usepackage{algorithmic}
\usepackage{graphicx}
\usepackage{textcomp}
\usepackage{xcolor}
\usepackage{txfonts}
\usepackage{listings}
\usepackage{enumitem}
\usepackage{mathtools}
\usepackage{gensymb}
\usepackage{graphicx}
\usepackage{wrapfig}
\usepackage{comment}
\usepackage[breaklinks=true]{hyperref}
\usepackage{tkz-euclide} 
\usepackage{listings}
% \usepackage{gvv}                                        
\def\inputGnumericTable{}                                 
\usepackage[latin1]{inputenc}                                
\usepackage{color}                                            
\usepackage{array}                                            
\usepackage{longtable}                                       
\usepackage{calc}                                             
\usepackage{multirow}                                         
\usepackage{hhline}                                           
\usepackage{ifthen}                                           
\usepackage{lscape}
\begin{document}

\bibliographystyle{IEEEtran}
\vspace{3cm}


\author{AI24BTECH11008- Sarvajith
}
\title{Assignment 7}
% \maketitle
% \newpage
% \bigskip
{\let\newpage\relax\maketitle}
\title{2007, CE}
\renewcommand{\thefigure}{\theenumi}
\renewcommand{\thetable}{\theenumi}
\setlength{\intextsep}{10pt} % Space between text and floats
\numberwithin{equation}{enumi}
\numberwithin{figure}{enumi}
\renewcommand{\thetable}{\theenumi}
\begin{enumerate}
    \item[27.] A cantilever beam of length 2 m with a square section of side length 0.1 m is loaded vertically at the free end. The vertical displacement at the free end is 5mm. The beam is made of steel with Young's modulus of $2.0\times 10^{11}\frac{N}{m^2}$. The maximum bending stress at the fixed end of the cantilever is \hfill (2018)
    \begin{enumerate}[label=(\Alph*)]
        \item 20.0 MPa
        \item 37.5 MPa
        \item 60.0 MPa
        \item 75.0 MPa
    \end{enumerate}
    \item[28.] A cylinder of radius 250mm and weight, W = 10kN is rolled up an obstacle of height 50mm by applying a horizontal force P at its centre as shown in the figure
    \begin{figure}[!ht]
        \centering
        \resizebox{0.4\textwidth}{!}{%
        \begin{circuitikz}
        \tikzstyle{every node}=[font=\normalsize]
        \draw  (11.25,9.75) circle (1cm);
        \draw [short] (10,8.75) -- (12.25,8.75);
        \draw [short] (12.25,8.75) -- (12.25,9.5);
        \draw [short] (12.25,9.5) -- (13.75,9.5);
        \draw [->, >=Stealth] (10.25,9.75) -- (11.25,9.75);
        \draw [->, >=Stealth] (11.25,9.75) -- (11.25,9);
        \node [font=\normalsize] at (10,9.75) {\textbf{P}};
        \node [font=\normalsize] at (11.5,9.25) {\textbf{W}};
        \draw [<->, >=Stealth] (13.75,9.5) -- (13.75,8.75)node[pos=0.5, fill=white]{\textbf{50mm}};
        \end{circuitikz}
        }% % Specify the path to your TikZ file
        \caption{1}
        %\label{fig2}
    \end{figure}
    All interfaces are assumed frictionless. The minimum value of P is.\hfill (2018)
    \begin{enumerate}[label=(\Alph*)]
        \item 4.5kN
        \item 5.0kN
        \item 6.0kN
        \item 7.5kN
    \end{enumerate}
    \item[29.] A plate in equilibrium is subjected to uniform stress along its edges with magnitude $\sigma_{xx}=30MPa$ and $\sigma_{yy}=50MPa$ as shown in the figure.
    \begin{figure}[!ht]
        \centering
        \resizebox{0.4\textwidth}{!}{%
    \begin{circuitikz}
    \tikzstyle{every node}=[font=\normalsize]
    \draw  (11,9.25) rectangle (14,6.5);
    \draw [->, >=Stealth] (14,9.25) -- (14,9.75);
    \draw [->, >=Stealth] (13.5,9.25) -- (13.5,9.75);
    \draw [->, >=Stealth] (13,9.25) -- (13,9.75);
    \draw [->, >=Stealth] (12.5,9.25) -- (12.5,9.75);
    \draw [->, >=Stealth] (12,9.25) -- (12,9.75);
    \draw [->, >=Stealth] (11.5,9.25) -- (11.5,9.75);
    \draw [->, >=Stealth] (11.25,9.25) -- (11.25,9.75);
    \draw [->, >=Stealth] (14,9) -- (14.5,9);
    \draw [->, >=Stealth] (14,8.75) -- (14.5,8.75);
    \draw [->, >=Stealth] (14,8.5) -- (14.5,8.5);
    \draw [->, >=Stealth] (14,8.25) -- (14.5,8.25);
    \draw [->, >=Stealth] (14,8) -- (14.5,8);
    \draw [->, >=Stealth] (14,7.75) -- (14.5,7.75);
    \draw [->, >=Stealth] (14,7.5) -- (14.5,7.5);
    \draw [->, >=Stealth] (14,7.25) -- (14.5,7.25);
    \draw [->, >=Stealth] (14,7) -- (14.5,7);
    \draw [->, >=Stealth] (14,6.75) -- (14.5,6.75);
    \draw [->, >=Stealth] (14,6.5) -- (14.5,6.5);
    \draw [->, >=Stealth] (14,6.5) -- (14,6);
    \draw [->, >=Stealth] (13.75,6.5) -- (13.75,6);
    \draw [->, >=Stealth] (13.5,6.5) -- (13.5,6);
    \draw [->, >=Stealth] (13.25,6.5) -- (13.25,6);
    \draw [->, >=Stealth] (13,6.5) -- (13,6);
    \draw [->, >=Stealth] (12.75,6.5) -- (12.75,6);
    \draw [->, >=Stealth] (12.5,6.5) -- (12.5,6);
    \draw [->, >=Stealth] (12.25,6.5) -- (12.25,6);
    \draw [->, >=Stealth] (12,6.5) -- (12,6);
    \draw [->, >=Stealth] (11.75,6.5) -- (11.75,6);
    \draw [->, >=Stealth] (11.5,6.5) -- (11.5,6);
    \draw [->, >=Stealth] (11.25,6.5) -- (11.25,6);
    \draw [->, >=Stealth] (11,6.5) -- (11,6);
    \draw [->, >=Stealth] (11,6.5) -- (10.5,6.5);
    \draw [->, >=Stealth] (11,6.75) -- (10.5,6.75);
    \draw [->, >=Stealth] (11,7) -- (10.5,7);
    \draw [->, >=Stealth] (11,7.25) -- (10.5,7.25);
    \draw [->, >=Stealth] (11,7.75) -- (10.5,7.75);
    \draw [->, >=Stealth] (11,7.5) -- (10.5,7.5);
    \draw [->, >=Stealth] (11,8) -- (10.5,8);
    \draw [->, >=Stealth] (11,8.25) -- (10.5,8.25);
    \draw [->, >=Stealth] (11,8.5) -- (10.5,8.5);
    \draw [->, >=Stealth] (11,8.75) -- (10.5,8.75);
    \draw [->, >=Stealth] (11,9) -- (10.5,9);
    \draw [->, >=Stealth] (11,9.25) -- (10.5,9.25);
    \draw [->, >=Stealth] (11,9.25) -- (11,9.75);
    \draw [->, >=Stealth] (11.75,9.25) -- (11.75,9.75);
    \draw [->, >=Stealth] (12.25,9.25) -- (12.25,9.75);
    \draw [->, >=Stealth] (12.75,9.25) -- (12.75,9.75);
    \draw [->, >=Stealth] (13.25,9.25) -- (13.25,9.75);
    \draw [->, >=Stealth] (13.75,9.25) -- (13.75,9.75);
    \node [font=\normalsize] at (15.5,8) {$\sigma_{xx} = 30MPa$};
    \node [font=\normalsize] at (12.25,10.25) {\textbf{$\sigma_{yy}=50MPa$}};
    \draw [->, >=Stealth, dashed] (12.5,8) -- (12.5,9);
    \draw [->, >=Stealth, dashed] (12.5,8) -- (13.5,8);
    \node [font=\normalsize] at (12.25,8.75) {$y$};
    \node [font=\normalsize] at (13.25,7.75) {$x$};
    \end{circuitikz}
    }%% Specify the path to your TikZ file
        \caption{2}
        %\label{fig2}
    \end{figure}
    The Young's Modulus of the material is $2\times 10^{11}\frac{N}{m^2}$ and the Poissons's ratio is 0.3. If $\sigma_{zz}$ is neglibly small and assumed to be zero, then the strain $\epsilon_{zz}$ is \hfill (2018)
    \begin{enumerate}[label=(\Alph*)]
        \item $-120\times 10^{-6}$
        \item $-60\times 10^{-6}$
        \item $0.0$
        \item $120\times 10^{-6}$
    \end{enumerate}
    \item[30.] The figure shows a simply supported beam PQ of uniform flexural rigidity $EI$ carrying two
    moments M and 2M. 
    \begin{figure}[!ht]
        \centering
        \resizebox{0.4\textwidth}{!}{%
    \begin{circuitikz}
    \tikzstyle{every node}=[font=\normalsize]
    \draw [short] (8,10) -- (9,10);
    \draw [short] (8.25,10) -- (8.5,9.5);
    \draw [short] (8.5,9.5) -- (8.75,10);
    \draw [short] (8.25,9.5) -- (18,9.5);
    \draw [short] (17.25,9.5) -- (17,9);
    \draw [short] (17.25,9.5) -- (17.75,9);
    \draw [short] (16.5,9) -- (18,9);
    \draw [short] (8.5,9) -- (8.5,8.25);
    \draw [short] (11.25,9) -- (11.25,8.25);
    \draw [short] (14,9) -- (14,8.25);
    \draw [short] (17.25,9) -- (17.25,8.25);
    \draw [<->, >=Stealth] (8.5,8.5) -- (11.25,8.5)node[pos=0.5, fill=white]{$\frac{L}{3}$};
    \draw [<->, >=Stealth] (11.25,8.75) -- (14,8.75)node[pos=0.5, fill=white]{$\frac{L}{3}$};
    \draw [<->, >=Stealth] (14,8.75) -- (17.25,8.75)node[pos=0.5, fill=white]{$\frac{L}{3}$};
    \draw [->, >=Stealth] (10.75,10) .. controls (11.75,10) and (12,10) .. (11.25,9.25) ;
    \draw [->, >=Stealth] (13.75,10) .. controls (14.75,10) and (14.75,10) .. (14.5,9) ;
    \node [font=\normalsize] at (7.75,9.75) {P};
    \node [font=\normalsize] at (18.25,9.25) {Q};
    \end{circuitikz}
    }% % Specify the path to your TikZ file
        \caption{3}
        %\label{fig2}
    \end{figure}
    The slope at P will be\hfill (2018)
    \begin{enumerate}[label= (\Alph*)]
        \item 0
        \item $\frac{ML}{9EI}$
        \item $\frac{ML}{6EI}$
        \item $\frac{ML}{3EI}$
    \end{enumerate}
    \item[31.] A $0.5 m \times 0.5 m$ square concrete pile is to be driven in a homogeneous clayey soil having
    undrained shear strength, $c_u = 50 kPa$ and unit weight, $\gamma = 18.0 \frac{kN}{m^3}$. The design capacity
    of the pile is 500 kN. The adhesion factor $\alpha$ is given as 0.75. The length of the pile required
    for the above design load with a factor of safety of 2.0\\ is \hfill (2018)
    \begin{enumerate}[label = (\Alph*)]
        \item 5.2m 
        \item 5.8 m 
        \item 11.8 m
        \item 12.5 m 
    \end{enumerate}
    \item[32.] A closed tank contains 0.5 m thick layer of mercury (specific gravity = 13.6) at the bottom.
    A 2.0 m thick layer of water lies above the mercury layer. A 3.0 m thick layer of oil
    (specific gravity = 0.6) lies above the water layer. The space above the oil layer contains
    air under pressure. The gauge pressure at the bottom of the tank is $196.2 \frac{kN}{m^2}$. The density of water is $1000 \frac{kg}{m^3}$
    and the acceleration due to gravity is $9.81 \frac{m}{s^2}$. The value
    of pressure in the air space is \hfill (2018)
    \begin{enumerate}[label=(\Alph*)]
        \item $92.214 \frac{kN}{m^2}$
        \item $95.644 \frac{kN}{m^2}$
        \item $98.922 \frac{kN}{m^2}$
        \item $99.321 \frac{kN}{m^2}$
    \end{enumerate} 
    \item[33.] A rapid sand filter comprising a number of filter beds is required to produce 99 MLD of
    potable water. Consider water loss during backwashing as 5\%, rate of filtration as $6.0 \frac{m}{h}$
    and length to width ratio of filter bed as 1.35. The width of each filter bed is to be kept
    equal to 5.2 m. One additional filter bed is to be provided to take care of break-down,
    repair and maintenance. The total number of filter beds required will be\hfill (2018)
    \begin{enumerate}[label=(\Alph*)]
        \item 19
        \item 20
        \item 21
        \item 22
    \end{enumerate}
    \item[34.] A priority intersection has a single-lane one-way traffic road crossing an undivided twolane two-way traffic road. The traffic stream speed on the single-lane road is 20 kmph and
    the speed on the two-lane road is 50 kmph. The perception-reaction time is 2.5 s, coefficient of longitudinal friction is 0.38 and acceleration due to gravity is 9.81 m/s2
    . A clear sight triangle has to be ensured at this intersection. The minimum lengths of the sides
    of the sight triangle along the two-lane road and the single-lane road, respectively will be\hfill (2018)
    \begin{enumerate}[label=(\Alph*)]
        \item 50 m and 20 m 
        \item 61 m and 18 m 
        \item 111 m and 15 m 
        \item 122 m and 36 m  
    \end{enumerate}
    \item[35.]The following details refer to a closed traverse:
    \begin{table}[h!]
        \centering
        \begin{tabular}{|c|>{\centering\arraybackslash}m{2.5cm}|>{\centering\arraybackslash}m{2.5cm}|>{\centering\arraybackslash}m{2.5cm}|>{\centering\arraybackslash}m{3cm}|}
            \hline
            \textbf{Line} & \multicolumn{4}{c|}{\textbf{Consecutive coordinate}} \\ \hline
            & \textbf{Northing (m)} & \textbf{Southing (m)} & \textbf{Easting (m)} & \textbf{Westing (m)} \\ \hline
            PQ & ---- & 437 & 173 & ---- \\ \hline
            QR & 101 & ---- & 558 & ---- \\ \hline
            RS & 419 & ---- & ---- & 96 \\ \hline
            SP & ---- & 83 & ---- & 634 \\ \hline
        \end{tabular}
        \caption{Consecutive coordinates for lines}
    \end{table}
    The length and direction(whole circle bearing) of closure, respectively are \hfill (2018)
    \begin{enumerate}[label=(\Alph*)]
        \item 1 m and $90^{\circ}$
        \item 2 m and $90^{\circ}$
        \item 1 m and $270^{\circ}$
        \item 2 m and $270^{\circ}$
    \end{enumerate}
    \item[36.] A square area (on the surface of the earth) with side 100 m and uniform height, appears as
    $1 cm^2$ on a vertical aerial photograph. The topographic map shows that a contour of 650 m
    passes through the area. If focal length of the camera lens is 150 mm, the height from
    which the aerial photograph was taken, is\hfill (2018)
    \begin{enumerate}[label=(\Alph*)]
        \item 800 m 
        \item 1500 m 
        \item 2150 m 
        \item 3150 m 
    \end{enumerate}
    \item[37.]The solution at x = 1,t = 1 of the partial differential equation $\frac{\partial ^2 u}{\partial x^2} = 25\frac{\partial^2u}{\partial t^2}$ subject to initial conditions of $u\brak{0} = 3x$ and $\frac{\partial u}{\partial t}\brak{0}=3$ is \hfill (2018)
    \begin{enumerate}[label=(\Alph*)]
        \item 1
        \item 2
        \item 4
        \item 6
    \end{enumerate}
    \item[38.] The solution (up to three decimal places) at x = 1 of the differential equation $\frac{d^2y}{dx^2}+2\frac{dy}{dx}+y=0$ subject to boundary conditions $y\brak{0}=1$ and $\frac{dy}{dx}\brak{0} = -1$ is \hfill (2018)
    \item[39.] Variation of water depth $\brak{y}$ in a gradually varied open channel flow is given by the first
    order differential equation $$\frac{dy}{dx} = \frac{1-e^{-\frac{10}{3}\ln y}}{250-45e^{-3\ln y}}$$Given initial condition: $y(x = 0) = 0.8 m$. The depth (in m, up to three decimal places) of
    flow at a downstream section at x = 1 m from one calculation step of Single Step Euler Method is \hfill (2018)
     
\end{enumerate}
\end{document}