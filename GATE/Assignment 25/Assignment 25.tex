\let\negmedspace\undefined
\let\negthickspace\undefined
\documentclass[journal]{IEEEtran}
\usepackage[a5paper, margin=10mm, onecolumn]{geometry}
\usepackage{lmodern} % Ensure lmodern is loaded for pdflatex
 % Include tfrupee package
\setlength{\headheight}{1cm} % Set the height of the header box
\setlength{\headsep}{0mm}     % Set the distance between the header box and the top of the text
\usepackage{enumitem}
\usepackage{gvv-book}
\usepackage{gvv}
\usepackage{cite}
\usepackage{amsmath,amssymb,amsfonts,amsthm}
\usepackage{algorithmic}
\usepackage{graphicx}
\usepackage{textcomp}
\usepackage{xcolor}
\usepackage{txfonts}
\usepackage{listings}
\usepackage{enumitem}
\usepackage{mathtools}
\usepackage{gensymb}
\usepackage{graphicx}
\usepackage{wrapfig}
\usepackage{comment}
\usepackage[breaklinks=true]{hyperref}
\usepackage{tkz-euclide} 
\usepackage{listings}
% \usepackage{gvv}                                        
\def\inputGnumericTable{}                                 
\usepackage[latin1]{inputenc}                                
\usepackage{color}                                            
\usepackage{array}                                            
\usepackage{longtable}                                       
\usepackage{calc}                                             
\usepackage{multirow}                                         
\usepackage{hhline}                                           
\usepackage{ifthen}                                           
\usepackage{lscape}
\begin{document}

\bibliographystyle{IEEEtran}
\vspace{3cm}


\author{AI24BTECH11008- Sarvajith
}
\title{Assignment 25}
% \maketitle
% \newpage
% \bigskip
{\let\newpage\relax\maketitle}
\title{2023, PH}
\renewcommand{\thefigure}{\theenumi}
\renewcommand{\thetable}{\theenumi}
\setlength{\intextsep}{10pt} % Space between text and floats
\numberwithin{equation}{enumi}
\numberwithin{figure}{enumi}
\renewcommand{\thetable}{\theenumi}
\begin{enumerate}
    \item[14.] The atomic number of an atom is 6. What is the spectroscopic notation of its
    ground state, according to Hund's rules?
    \begin{enumerate}[label = (\Alph*)]
        \item $^3P_0$
        \item $^3P_1$
        \item $^3D_3$
        \item $^3S_1$
    \end{enumerate}
    \item[15.] H is the Hamiltonian, $\overrightarrow{H}$ the orbital angular momentum and $L_Z$
    is the z-component of $\overrightarrow{L}$. The 1s state of the hydrogen atom in the non-relativistic
    formalism is an eigen function of which one of the following sets of operators?
    \begin{enumerate}[label = (\Alph*)]
        \item $H, L^2, L_Z$
        \item $H,\overrightarrow{L}, L^2, L_Z$
        \item $ L^2, L_Z$ only 
        \item H and $L_Z$ only 
     \end{enumerate} 
    \item[16.] The Hall experiment is carried out with a non-magnetic semiconductor. The
    current $I$ is along the $x$-axis and the magnetic field $B$ is along the z-axis. Which
    one of the following is the CORRECT representation of the variation of the
    magnitude of the Hall resistivity $\rho_{xy}$ as a function of the magnetic field?
    \begin{figure}[!ht]
        \centering
        \resizebox{0.3\textwidth}{!}{%
        \begin{circuitikz}
        \tikzstyle{every node}=[font=\normalsize]
        \draw [short] (3.75,10.25) -- (3.75,7.75);
        \draw [short] (3.75,7.75) -- (7.25,7.75);
        \draw [short] (4,8.75) -- (7,8.75);
        \node [font=\normalsize] at (7.25,7.5) {B};
        \node [font=\normalsize] at (3.25,10) {$\rho_{xy}$};
        \end{circuitikz}
        }% % Specify the path to your TikZ file
        \caption{option1}
        %\label{fig2}
    \end{figure}
    \begin{figure}[!ht]
        \centering
        \resizebox{0.3\textwidth}{!}{%
    \begin{circuitikz}
    \tikzstyle{every node}=[font=\normalsize]
    \draw [short] (3.75,10.25) -- (3.75,7.75);
    \draw [short] (3.75,7.75) -- (7.25,7.75);
    \node [font=\normalsize] at (7.25,7.5) {B};
    \node [font=\normalsize] at (3.25,10) {$\rho_{xy}$};
    \draw [short] (3.75,7.75) -- (6.25,9.75);
    \end{circuitikz}
    }% % Specify the path to your TikZ file
        \caption{option2}
        %\label{fig2}
    \end{figure}
    \begin{figure}[!ht]
        \centering
        \resizebox{0.3\textwidth}{!}{%
    \begin{circuitikz}
    \tikzstyle{every node}=[font=\normalsize]
    \draw [short] (3.75,10.25) -- (3.75,7.75);
    \draw [short] (3.75,7.75) -- (7.25,7.75);
    \node [font=\normalsize] at (7.25,7.5) {B};
    \node [font=\normalsize] at (3.25,10) {$\rho_{xy}$};
    \draw [short] (4,10) .. controls (5,8.5) and (5,8.5) .. (6.25,8);
    \end{circuitikz}
    }%  % Specify the path to your TikZ file
        \caption{option23}
        %\label{fig2}
    \end{figure}
    \begin{figure}[!ht]
        \centering
        \resizebox{0.3\textwidth}{!}{%
    \begin{circuitikz}
    \tikzstyle{every node}=[font=\normalsize]
    \draw [short] (3.75,10.25) -- (3.75,7.75);
    \draw [short] (3.75,7.75) -- (7.25,7.75);
    \node [font=\normalsize] at (7.25,7.5) {B};
    \node [font=\normalsize] at (3.25,10) {$\rho_{xy}$};
    \draw [short] (3.75,7.75) .. controls (4.75,8.75) and (4.25,8.75) .. (4.5,9.75);
    \draw [short] (4.5,9.75) .. controls (6,10.25) and (5.75,10) .. (6.75,10);
    \end{circuitikz}
    }% % Specify the path to your TikZ file
        \caption{option4}
        %\label{fig2}
    \end{figure}
    \item[17.] Consider a two dimensional Cartesian coordinate system in which a rank 2
    contravariant tensor is represented by the matrix $\begin{bmatrix}0&1\\1&0\end{bmatrix}$. The coordinate system
    is rotated anticlockwise by an acute angle $\theta$ with the origin fixed. Which one of
    the following matrices represents the tensor in the new coordinate system?
    \begin{enumerate}[label = (\Alph*)]
        \item $\begin{bmatrix}0&cos2\theta\\-sin2\theta&0\end{bmatrix}$
        \item $\begin{bmatrix}sin2\theta&cos2\theta\\cos2\theta&-sin2\theta\end{bmatrix}$
        \item $\begin{bmatrix}sin2\theta&-cos2\theta\\cos2\theta&sin2\theta\end{bmatrix}$
        \item $\begin{bmatrix}sin2\theta&0\\0&-cos2\theta\end{bmatrix}$
    \end{enumerate}
    \item[18.] A compound consists of three ions X, Y and Z. The Z ions are arranged in an FCC
    arrangement. The X ions occupy $\frac{1}{6}$ of the tetrahedral voids and the Y ions occupy
    $\frac{1}{3}$ of the octahedral voids. Which one of the following is the CORRECT chemical
    formula of the compound? 
    \begin{enumerate}[label = (\Alph*)]
        \item $XY_2Z_4$
        \item $XYZ_3$
        \item $XYZ_2$
        \item $XYZ_4$
    \end{enumerate}
    \newpage
    \item[19.] For a non-magnetic metal, which one of the following graphs best represents the
    behaviour of $\frac{C}{T} \text{ vs. } T^2$ where C is the heat capacity and T is the temperature? 
    \begin{figure}[!ht]
        \centering
        \resizebox{0.3\textwidth}{!}{%
    \begin{circuitikz}
    \tikzstyle{every node}=[font=\normalsize]
    \draw [short] (3.75,10.25) -- (3.75,7.75);
    \draw [short] (3.75,7.75) -- (7.25,7.75);
    \node [font=\normalsize] at (3.75,10.5) {$\frac{C}{T}$};
    \draw [short] (3.75,8.25) .. controls (5.75,9) and (5.75,9.25) .. (6.5,10.5);
    \node [font=\normalsize] at (7.5,7.75) {$T^2$};
    \end{circuitikz}
    }%  % Specify the path to your TikZ file
        \caption{option1}
        %\label{fig2}
    \end{figure}
    \begin{figure}[!ht]
        \centering
        \resizebox{0.3\textwidth}{!}{%
    \begin{circuitikz}
    \tikzstyle{every node}=[font=\normalsize]
    \draw [short] (3.75,10.25) -- (3.75,7.75);
    \draw [short] (3.75,7.75) -- (7.25,7.75);
    \node [font=\normalsize] at (3.75,10.5) {$\frac{C}{T}$};
    \node [font=\normalsize] at (7.5,7.75) {$T^2$};
    \draw [short] (3.75,8.25) -- (6.5,10);
    \end{circuitikz}
    }%% Specify the path to your TikZ file
        \caption{option2}
        %\label{fig2}
    \end{figure}
    \begin{figure}[!ht]
        \centering
        \resizebox{0.3\textwidth}{!}{%
    \begin{circuitikz}
    \tikzstyle{every node}=[font=\normalsize]
    \draw [short] (3.75,10.25) -- (3.75,7.75);
    \draw [short] (3.75,7.75) -- (7.25,7.75);
    \node [font=\normalsize] at (3.75,10.5) {$\frac{C}{T}$};
    \node [font=\normalsize] at (7.5,7.75) {$T^2$};
    \draw [short] (3.75,8.25) .. controls (4.75,9.75) and (5,10) .. (6.5,10);
    \end{circuitikz}
    }%  % Specify the path to your TikZ file
        \caption{option3}
        %\label{fig2}
    \end{figure}
    \begin{figure}[!ht]
        \centering
        \resizebox{0.3\textwidth}{!}{%
    \begin{circuitikz}
    \tikzstyle{every node}=[font=\normalsize]
    \draw [short] (3.75,10.25) -- (3.75,7.75);
    \draw [short] (3.75,7.75) -- (7.25,7.75);
    \node [font=\normalsize] at (3.75,10.5) {$\frac{C}{T}$};
    \node [font=\normalsize] at (7.5,7.75) {$T^2$};
    \draw [short] (3.75,7.75) -- (6.25,10.25);
    \end{circuitikz}
    }% % Specify the path to your TikZ file
        \caption{option4}
        %\label{fig2}
    \end{figure}
    \newpage
    \item[20.] For nonrelativistic electrons in solid, different energy dispersion relations (with effective masses $m_a^*,m_b^*,m_c^*$) are schematically shown in the plots. Which one of the following options is correct?
    \begin{figure}[!ht]
        \centering
        \resizebox{0.4\textwidth}{!}{%
    \begin{circuitikz}
    \tikzstyle{every node}=[font=\normalsize]
    \draw [short] (2.75,7) -- (8.75,7);
    \draw [short] (3.25,7.75) .. controls (4.5,7.25) and (4.25,7.25) .. (6,7);
    \draw [short] (6,7) .. controls (8,7.25) and (7.75,7.25) .. (8.75,7.75);
    \draw [dashed] (3,9.5) .. controls (4.25,8) and (4.25,8) .. (6,7);
    \draw [dashed] (6,7) .. controls (8,8.25) and (7.75,8.25) .. (8.75,9.5);
    \draw [dashed] (4.75,10.5) .. controls (5,8.75) and (5,8.75) .. (6,7);
    \draw [dashed] (6,7) .. controls (7,8.75) and (6.75,8.75) .. (7,10.5);
    \draw [short] (6,7) -- (6,11);
    \node [font=\normalsize] at (9,7.75) {$m_a^*$};
    \node [font=\normalsize] at (9,9.5) {$m_b^*$};
    \node [font=\normalsize] at (7.25,10.5) {$m_c^*$};
    \node [font=\normalsize] at (6,11.25) {E};
    \end{circuitikz}
    }% % Specify the path to your TikZ file
        \caption{}
        %\label{fig2}
    \end{figure}
    \begin{enumerate}[label = (\Alph*)]
        \item $m_a^*=m_b^*=m_c^*$
        \item $m_b^*>m_c^*>m_a^*$
        \item $m_c^*>m_b^*>m_a^*$
        \item $m_a^*>m_b^*>m_c^*$
    \end{enumerate}
   
    \item[21.] The figure schematically shows the M (magnetization) - H (magnetic field) plots
    for certain types of materials. Here M and H are plotted in the same scale and
    units. Which one of the following is the most appropriate combination? 
    \begin{figure}[!ht]
        \centering
        \resizebox{0.4\textwidth}{!}{%
        \begin{circuitikz}
        \tikzstyle{every node}=[font=\normalsize]
        \draw [short] (3.75,11.25) -- (3.75,3.75);
        \draw [short] (1.25,8.75) -- (10.5,8.75);
        \draw [dashed] (3.75,8.75) -- (8.75,9.25);
        \draw [dashed] (3.75,8.75) -- (8.25,8.5);
        \draw [dashed] (3.75,8.75) -- (4.5,5.25);
        \draw [short] (3.75,8.75) -- (7,6.25);
        \node [font=\normalsize] at (9,9.25) {P};
        \node [font=\normalsize] at (8.5,8.5) {Q};
        \node [font=\normalsize] at (7.25,6.25) {R};
        \node [font=\normalsize] at (4.5,5) {S};
        \node [font=\normalsize] at (3.75,11.5) {M};
        \node [font=\normalsize] at (10.75,8.75) {H};
        \end{circuitikz}
        }% % Specify the path to your TikZ file
        \caption{}
        %\label{fig2}
    \end{figure}
    \begin{enumerate}[label = (\Alph*)]
        \item (Q) - Paramagnet; (R) - Type-I Superconductor; (S) - Antiferromagnet
        \item (P) - Paramagnet; (Q) - Diamagnet; (R) - Type-I Superconductor
        \item (P) - Paramagnet; (Q) - Antiferromagnet; (R) - Type-I Superconductor
        \item (P) - Diamagnet; (R) - Paramagnet; (S) - Type-I Superconductor
    \end{enumerate}
    \item[22.] Graphene is a two dimensional material, in which carbon atoms are arranged in a
    honeycomb lattice with lattice constant $a$. As shown in the figure, $\overrightarrow{a}_1$ and $\overrightarrow{a}_2$
    are two lattice vectors. Which one of the following is the area of the first Brillouin
    zone for this lattice? 
    \begin{figure}[!ht]
        \centering
        \resizebox{0.4\textwidth}{!}{%
    \begin{circuitikz}
    \tikzstyle{every node}=[font=\normalsize]
    \draw [short] (4.5,10) -- (6.25,10);
    \draw [short] (4.5,10) -- (4,9);
    \draw [short] (4,9) -- (4.5,8);
    \draw [short] (6.25,10) -- (6.75,9);
    \draw [short] (6.75,9) -- (6.25,8);
    \draw [short] (4.5,8) -- (6.25,8);
    \draw [short] (6.75,9) -- (8,9);
    \draw [short] (8,9) -- (8.5,8);
    \draw [short] (6.25,8) -- (6.75,7.25);
    \draw [short] (6.75,7.25) -- (8,7.25);
    \draw [short] (8,7.25) -- (8.5,8);
    \draw [short] (4.5,8) -- (4,7.25);
    \draw [short] (4,7.25) -- (4.5,6.5);
    \draw [short] (4.5,6.5) -- (6.25,6.5);
    \draw [short] (6.25,6.5) -- (6.75,7.25);
    \draw [short] (8,7.25) -- (8.5,6.5);
    \draw [short] (6.25,6.5) -- (6.75,5.75);
    \draw [short] (6.75,5.75) -- (8,5.75);
    \draw [short] (8,5.75) -- (8.5,6.5);
    \draw [short] (4.5,6.5) -- (4,5.75);
    \draw [short] (4,5.75) -- (4.5,5);
    \draw [short] (4.5,5) -- (6.25,5);
    \draw [short] (6.25,5) -- (6.75,5.75);
    \draw [short] (4,7.25) -- (2.75,7.25);
    \draw [short] (2.75,7.25) -- (2.25,6.5);
    \draw [short] (2.25,6.5) -- (2.75,5.75);
    \draw [short] (2.75,5.75) -- (4,5.75);
    \draw [short] (4,9) -- (2.75,9);
    \draw [short] (2.75,9) -- (2,8);
    \draw [short] (2,8) -- (2.75,7.25);
    \draw [->, >=Stealth] (6.25,6.5) -- (6.75,7.25);
    \draw [->, >=Stealth] (6.25,6.5) -- (6.75,5.75);
    \node [font=\normalsize] at (5.5,7.25) {$\overrightarrow{a}_1$};
    \node [font=\normalsize] at (5.5,6) {$\overrightarrow{a}_2$};
    \end{circuitikz}
    }% % Specify the path to your TikZ file
        \caption{}
        %\label{fig2}
    \end{figure}
    \begin{enumerate}[label = (\Alph*)]
        \item $\frac{8\pi^2}{3\sqrt{3}a^2}$
        \item $\frac{4\pi^2}{3\sqrt{3}a^2}$
        \item $\frac{8\pi^2}{\sqrt{3}a^2}$
        \item $\frac{4\pi^2}{\sqrt{3}a^2}$
    \end{enumerate}
    \item[23.] A $^{60}Co$ nucleus emits a $\beta$ particle and is converted to $^{60}Ni^*$ with $J^P = 4^+$, which in turn decays ot the $^{60}Ni^*$ ground state with $J^P = 0+$ by emitting two photons in succession, as shown in the figure. Which one of the folowing statements is correct?
    \begin{figure}[!ht]
        \centering
        \resizebox{0.5\textwidth}{!}{%
        \begin{circuitikz}
        \tikzstyle{every node}=[font=\normalsize]
        \draw [short] (3.25,10.25) -- (6.25,10.25);
        \draw [short] (5.25,8) -- (7.75,8);
        \draw [short] (5.25,6.75) -- (7.75,6.75);
        \draw [short] (5.25,5.75) -- (7.75,5.75);
        \draw [->, >=Stealth, dashed] (4.75,10.25) -- (6,8);
        \draw [->, >=Stealth, dashed] (6.25,8) -- (6.25,6.75);
        \draw [->, >=Stealth, dashed] (6.5,6.75) -- (6.5,5.75);
        \node [font=\normalsize] at (4,10.5) {$^{60}Co$};
        \node [font=\normalsize] at (5.5,9.25) {$\beta$};
        \node [font=\normalsize] at (5.5,7) {$^{60}Ni$};
        \node [font=\normalsize] at (5.5,6) {$^{60}Ni$};
        \node [font=\normalsize] at (6.75,7) {$\gamma$};
        \node [font=\normalsize] at (6.75,6) {$\gamma$};
        \node [font=\normalsize] at (6.75,8.25) {$^{60}Ni$};
        \node [font=\normalsize] at (6.25,10.5) {$5^+$};
        \node [font=\normalsize] at (8,8) {$4+$};
        \node [font=\normalsize] at (8,6.75) {$2+$};
        \node [font=\normalsize] at (8,5.75) {$0+$};
        \end{circuitikz}
        }% % Specify the path to your TikZ file
        \caption{}
        %\label{fig2}
    \end{figure}
    \begin{enumerate}[label = (\Alph*)]
        \item $4^+\to 2^+$  is an electric octupole transition
        \item $4^+\to 2^+$  is a magnetic quadrupole transition
        \item $2^+\to 0^+$  is an electric quadrupole transition
        \item $2^+\to 0^+$  is a magnetic quadrupole transition
    \end{enumerate}
    \item[24.] Which one of the following options is CORRECT for the given logic circuit?
    \begin{figure}[!ht]
        \centering
        \resizebox{0.5\textwidth}{!}{%
    \begin{circuitikz}
    \tikzstyle{every node}=[font=\normalsize]
    \draw (4.5,9.5) to[short, -o] (3.25,9.5) ;
    \draw (4.5,9) to[short, -o] (3.25,9) ;
    \draw (4.5,9.5) to[short] (4.75,9.5);
    \draw (4.5,9) to[short] (4.75,9);
    \draw (4.75,9.5) node[ieeestd and port, anchor=in 1, scale=0.89](port){} (port.out) to[short] (6.5,9.25);
    \draw [short] (4,9.5) -- (4,11.25);
    \draw [short] (4,11.25) -- (4.75,11.25);
    \draw (4.75,11.5) to[short] (5,11.5);
    \draw (4.75,11) to[short] (5,11);
    \draw (5,11.5) node[ieeestd nand port, anchor=in 1, scale=0.89](port){} (port.out) to[short] (6.75,11.25);
    \draw [short] (4.75,11.5) -- (4.75,11);
    \draw [short] (6.75,11.25) -- (8.25,11.25);
    \draw [short] (8.25,11.25) -- (8.25,10);
    \draw [short] (6.5,9.25) -- (8.75,9.25);
    \draw [short] (8.25,10.25) -- (8.25,9.75);
    \draw [short] (8.25,9.75) -- (8.75,9.75);
    \draw (8.75,9.75) to[short] (9,9.75);
    \draw (8.75,9.25) to[short] (9,9.25);
    \draw (9,9.75) node[ieeestd or port, anchor=in 1, scale=0.89](port){} (port.out) to[short] (10.75,9.5);
    \node [font=\normalsize] at (11,9.5) {$X$};
    \node [font=\normalsize] at (2.75,9.5) {$P$};
    \node [font=\normalsize] at (2.75,8.75) {$Q$};
    \end{circuitikz}
    }%  % Specify the path to your TikZ file
        \caption{}
        %\label{fig2}
    \end{figure}
    \begin{enumerate}[label = (\Alph*)]
        \item P = 1, Q = 1; X = 0
        \item P = 1, Q = 0; X = 1
        \item P = 0, Q = 1; X = 0
        \item P = 0, Q = 0; X = 1
    \end{enumerate}
    \item[25.] An atom with non-zero magnetic moment has an angular momentum of magnitude $\sqrt{12}\hslash$. When a beam of such atoms is passed through a Stern-Gerlach apparatus, how many beams does it split into?
    \begin{enumerate}[label = (\Alph*)]
        \item 3
        \item 7
        \item 9
        \item 25
    \end{enumerate}
    \item[26.] A $4\times4$ matrix M has the property $M^\dagger$ = -M and $M^4=1$,  where $\mathbf{1}$ is the $4\times4$ identity matrix. Which one of the following is the CORRECT set of eigen values of the matrix M?
    \begin{enumerate}[label=(\Alph*)]
        \item $\brak{1,1,-1,-1}$
        \item $\brak{i,i,-i,-i}$
        \item $\brak{i,i,i,-i}$
        \item $\brak{1,1,-i,-i}$
    \end{enumerate}
\end{enumerate}
\end{document}
