\let\negmedspace\undefined
\let\negthickspace\undefined
\documentclass[journal]{IEEEtran}
\usepackage[a5paper, margin=10mm, onecolumn]{geometry}
\usepackage{lmodern} % Ensure lmodern is loaded for pdflatex
 % Include tfrupee package
\setlength{\headheight}{1cm} % Set the height of the header box
\setlength{\headsep}{0mm}     % Set the distance between the header box and the top of the text
\usepackage{enumitem}
\usepackage{gvv-book}
\usepackage{gvv}
\usepackage{cite}
\usepackage{amsmath,amssymb,amsfonts,amsthm}
\usepackage{algorithmic}
\usepackage{graphicx}
\usepackage{textcomp}
\usepackage{xcolor}
\usepackage{txfonts}
\usepackage{listings}
\usepackage{enumitem}
\usepackage{mathtools}
\usepackage{gensymb}
\usepackage{graphicx}
\usepackage{wrapfig}
\usepackage{comment}
\usepackage[breaklinks=true]{hyperref}
\usepackage{tkz-euclide} 
\usepackage{listings}
% \usepackage{gvv}                                        
\def\inputGnumericTable{}                                 
\usepackage[latin1]{inputenc}                                
\usepackage{color}                                            
\usepackage{array}                                            
\usepackage{longtable}                                       
\usepackage{calc}                                             
\usepackage{multirow}                                         
\usepackage{hhline}                                           
\usepackage{ifthen}                                           
\usepackage{lscape}
\begin{document}

\bibliographystyle{IEEEtran}
\vspace{3cm}


\author{AI24BTECH11008- Sarvajith
}
\title{Assignment 22}
% \maketitle
% \newpage
% \bigskip
{\let\newpage\relax\maketitle}
\title{2022, XE}
\renewcommand{\thefigure}{\theenumi}
\renewcommand{\thetable}{\theenumi}
\setlength{\intextsep}{10pt} % Space between text and floats
\numberwithin{equation}{enumi}
\numberwithin{figure}{enumi}
\renewcommand{\thetable}{\theenumi}
\begin{enumerate}
    \item[14.] Let $\Gamma$ be the positively oriented circle $x^2 + y^2 = 9$ in the xy-plane. If $$\oint_{\Gamma}\brak{3y + e^{xsinx}dx + \brak{7x+\sqrt{e^y + 2}}}dy =\alpha \pi,$$ where $\alpha$ is a real constant then $\alpha$ is equal to..............
    \item[15.] Let $y_1\brak{x}$ and $y_2\brak{x}$ be two linearly independent solutions of $$x^2\frac{d^2y}{dx^2}-2x\frac{dy}{dx}+2y=, \hfill x>0.$$ Let $W\brak{y_1,y_2}\brak{x}$ denote the Wronskian of $y_1\brak{x}$ and $y_2\brak{x}$ at x. If $W\brak{y_1, y_2}\brak{1}=1$ then $W\brak{y_1, y_2}\brak{2}$ is equal to.........
    \item[16.] Let $A = \begin{bmatrix}2 &0&1&1\\1&2&5&-5\\0&0&3&0\\0&0&1&3\end{bmatrix}$. Then the sum of the geometric multiplicities of the distinct eigenvalues of A is equal to ...........
    \item[17.] In a cosmopolitan city, the population comprises of 30\% female and 70\% male.
    Suppose that 5\% of female and 30\% of male in the population are foreigners. A person is selected at random from this population. Given that the selected person
    is a foreigner, the probability that the person is a female is ............. (round off to
    three decimal places). 
    \item[18.] Let $f:\brak{0, \infty}\to \mathcal{R}$ be the continuous function such that $f\brak{x} = 2 + \frac{g\brak{x}}{x}$ for all $x>0$, where $g\brak{x}=\int_{1}^{x}f\brak{t}dt$ for all x>0. Then $f\brak{2}$ is equal to 
    \begin{enumerate}[label = (\Alph*)]
        \item $2+\ln 2$
        \item $2-\ln 2$
        \item $2+\ln 4$
        \item $2-\ln 4$
    \end{enumerate}
    \item[19.] Let A and B be $n\times n$ matrices with real entries. Consider the following statements:\\
    P: If A is symmetric then rank(A) = Number of nonzero eigenvalues(counting multiplicity) of A.
    Q: If AB = 0 then $rank\brak{A} + rank\brak{B}\leq n$.\\
    Then 
    \begin{enumerate}[label = (\Alph*)]
        \item both P and Q are TRUE
        \item P is TRUE and Q is FALSE 
        \item Q is TRUE and P is FALSE
        \item both P and Q are FALSE  
    \end{enumerate}
    \item[20.] Let $f:\mathcal{R}^2\to R$ be given by $f\brak{x,y} = 4xy-2x^2-y^4+1$. The number of critical points where $f$ has local maximum is equal to ...........
    \item[21.] If the quadrature rule $$\int_{-1}^{1}f\brak{x}dx\approx f\brak{\alpha} + \gamma f\brak{\beta},$$ where $\alpha,\beta$ and $\gamma$ are real constants, is exact for all polynomials for degree$\leq 3$, then $\gamma + 3\brak{\alpha^2 + \beta^2} + \brak{\alpha^3 + \beta^3}$ is equal to .............
    \item[22.] A heavy horizontal cylinder of diameter D supports a mass of liqud having density $\rho$ as shown in the figure. Find out the vertical compononent of force exerted by the liquid per unit length of the cylinder if g is the acceleration due to gravity.
    \begin{figure}[!ht]
        \centering
        
    \resizebox{0.3\textwidth}{!}{%
    \begin{circuitikz}
    \tikzstyle{every node}=[font=\normalsize]
    \draw [short] (3.75,10.25) -- (3.75,7.75);
    \draw [short] (3.75,7.75) -- (7.25,7.75);
    \draw [short] (4,8.75) -- (7,8.75);
    \node [font=\normalsize] at (7.25,7.5) {B};
    \node [font=\normalsize] at (3.25,10) {$\rho_{xy}$};
    \end{circuitikz}
    }%
    
    % Specify the path to your TikZ file
        \caption{}
        %\label{fig2}
    \end{figure}
    \begin{enumerate}[label = (\Alph*)]
        \item $\frac{\pi D^2}{4}\rho g$
        \item $\frac{\pi D^2}{8}\rho g$
        \item $\frac{\pi D^2}{2}\rho g$
        \item $\frac{\pi D^2}{3}\rho g$
    \end{enumerate} 
    \item[23.] The figure shows the developing zone and the fully developed region in a pipe
    flow where the steady flow takes place from left to right. The wall shear stress in
    the sections A, B, C, and D are given by $\tau_A,\tau_B,\tau_C,\tau_D$, respectively. Select the
    correct statement. 
    \begin{figure}[!ht]
        \centering
        
    \resizebox{0.3\textwidth}{!}{%
    \begin{circuitikz}
    \tikzstyle{every node}=[font=\normalsize]
    
    \draw (-5.5,13.25) to[R] (-3.5,13.25);
    \draw (-2,13.75) node[op amp,scale=1, yscale=-1 ] (opamp2) {};
    \draw (opamp2.+) to[short] (-3.5,14.25);
    \draw  (opamp2.-) to[short] (-3.5,13.25);
    \draw (-0.8,13.75) to[short](-0.5,13.75);
    \draw (-3.5,14.25) to (-4,14.25) node[ground]{};
    \draw [short] (-3,13.25) -- (-3,12.25);
    \draw [short] (-1,13.75) -- (-1,12.25);
    \draw (-3,12.25) to[empty Schottky diode] (-1,12.25);
    \node [font=\normalsize] at (-2,11.75) {5.1V};
    \end{circuitikz}
    }%
    
     % Specify the path to your TikZ file
        \caption{}
        %\label{fig2}
    \end{figure}
    \begin{enumerate}[label = (\Alph*)]
        \item $\tau_A > \tau_B$
        \item $\tau_B > \tau_A$
        \item $\tau_C > \tau_B$
        \item $\tau_C > \tau_D$
    \end{enumerate}
    \item[24.] The left hand column lists some non-dimensional numbers and the right hand column lists some physical phenomena. Indicate the correct combination 
    \begin{table}
        \centering
        \begin{tabular}{|c|c|c|c|}
    \hline
    $\beta$ & Airplane A & Airplane B & Airplane C \\ \hline
    $\beta = -5^\circ$ & -0.030 & -0.025 & 0.040 \\ \hline
    $\beta = 0^\circ$ & 0 & 0 & 0 \\ \hline
    $\beta = 5^\circ$ & 0.030 & 0.025 & -0.040 \\ \hline
\end{tabular}  % Specify the path to your TikZ file
        \caption{}
        %\label{fig2}
    \end{table}
    \begin{enumerate}[label = (\Alph*)]
        \item 1-iii, 2-i, 3-ii, 4-iv
        \item 1-i, 2-ii, 3-iv, 4-iii
        \item 1-iv, 2-iii, 3-iv, 4-iii
        \item 2-iv, 1-iii, 3-ii, 4-i
    \end{enumerate}
    \item[25.] As temperature increases
    \begin{enumerate}[label = (\Alph*)]
        \item the dynamic viscosity of a gas increases. 
        \item the dynamic viscosity of a liquid decreases.
        \item the dynamic viscosity of a liquid does not change.
        \item the dynamic viscosity of a gas decreases. 
    \end{enumerate}
    \item[26.] Which of the following statement(s) regarding a venturimeter is/are correct? 
    \begin{enumerate}[label = (\Alph*)]
        \item In the direction of flow, it consists of a converging section, a throat, and a
        diverging section
        \item In the direction of flow, it consists of a diverging section, a throat, and a
        converging section.
        \item It is used for flow measurement at a very low Reynolds number.
        \item Pressure tappings are provided just upstream of the venturimeter and at the throat.
    \end{enumerate}
\end{enumerate}
\end{document}