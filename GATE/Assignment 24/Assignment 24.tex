\let\negmedspace\undefined
\let\negthickspace\undefined
\documentclass[journal]{IEEEtran}
\usepackage[a5paper, margin=10mm, onecolumn]{geometry}
\usepackage{lmodern} % Ensure lmodern is loaded for pdflatex
 % Include tfrupee package
\setlength{\headheight}{1cm} % Set the height of the header box
\setlength{\headsep}{0mm}     % Set the distance between the header box and the top of the text
\usepackage{enumitem}
\usepackage{gvv-book}
\usepackage{gvv}
\usepackage{cite}
\usepackage{amsmath,amssymb,amsfonts,amsthm}
\usepackage{algorithmic}
\usepackage{graphicx}
\usepackage{textcomp}
\usepackage{xcolor}
\usepackage{txfonts}
\usepackage{listings}
\usepackage{enumitem}
\usepackage{mathtools}
\usepackage{gensymb}
\usepackage{graphicx}
\usepackage{wrapfig}
\usepackage{comment}
\usepackage[breaklinks=true]{hyperref}
\usepackage{tkz-euclide} 
\usepackage{listings}
% \usepackage{gvv}                                        
\def\inputGnumericTable{}                                 
\usepackage[latin1]{inputenc}                                
\usepackage{color}                                            
\usepackage{array}                                            
\usepackage{longtable}                                       
\usepackage{calc}                                             
\usepackage{multirow}                                         
\usepackage{hhline}                                           
\usepackage{ifthen}                                           
\usepackage{lscape}
\begin{document}

\bibliographystyle{IEEEtran}
\vspace{3cm}


\author{AI24BTECH11008- Sarvajith
}
\title{Assignment 24}
% \maketitle
% \newpage
% \bigskip
{\let\newpage\relax\maketitle}
\title{2023, MA}
\renewcommand{\thefigure}{\theenumi}
\renewcommand{\thetable}{\theenumi}
\setlength{\intextsep}{10pt} % Space between text and floats
\numberwithin{equation}{enumi}
\numberwithin{figure}{enumi}
\renewcommand{\thetable}{\theenumi}
\begin{enumerate}
    \item[27.] Let $$\frac{z}{1-z-z^2} = \sum_{n=0}^{\infty}a_nz^n, a_n \in R$$ for all z in some neighbourhood of 0 in C.\\ Then the value of $a_6+a_5$ is equal to......... 
    \item[28.] Let $p\brak{x}=x^3-2x+2$. If $q\brak{x}$ is the interpolating polynomial of degree less than or equal to 4 for the data in the table \textbf{polynomial data}\\
    \begin{table}
        \centering
        \begin{tabular}{|c|c|c|c|c|c|}
            \hline
            x & -2&-1&0&1&3\\ \hline
            $q\brak{x}$&$p\brak{-2}$&$p\brak{-1}$&2.5&$p\brak{1}$&$p\brak{3}$\\ \hline
        \end{tabular} % Specify the path to your TikZ file
        \caption{polynomial data}
        %\label{fig2}
    \end{table}
    then the value of $\frac{d^4q}{dx^4}$ at x=0 is ...........
    \item[29.] For a fixed $c \in R$, let $\alpha = \int_{0}^{2}\brak{9x^2-5cx^4}dx$.\\ If the value of $\int_{0}^{2}\brak{9x^2-5cx^4}dx$ obtained by using the Trapezoidal rule is equal to $\alpha$, then the value of c is ................(rounded off to 2 decimal places).
    \item[30.] If for some $\alpha \in R$ $$\int_{1}^{4}\int_{-x}^{x}\frac{1}{x^2+y^2}dydx = \int_{\frac{-\pi}{4}}^{\frac{\pi}{4}}\int_{sec\theta}^{\alpha sec\theta}\frac{1}{r}drd\theta,$$ then the value of $\alpha$ equals...........
    \item[31.] Let S be the portion of the plane $z=2x+2y-100$ which lies inside the cylinder $x^2+y^2=1$. If the surface area of S is $\alpha\pi$, theen the value of $\alpha$ is equal to ............ 
    \item[32.] Let $L^2[-1,1] = \{f:[-1,1]\to R:\text{ f is Lebesgue measureable and }\int_{-1}^{1}\abs{f\brak{x}}^2dx<\infty\}$ and the norm $\norm{f}_2=\brak{\int_{-1}^{1}\abs{f\brak{x}}^2dx}^{\frac{1}{2}}$ for $f\in L^2[-1,1]$.\\
    Let $F:\brak{L^2[-1,1],\norm{.}_2}\to R$ be denoted by $$F\brak{f} = \int_{-1}^{1}f\brak{x}x^2dx\text{  for all }f\in L^2[-1,1].$$ If $\norm{F}$ denotes the norm of the linear functional F, then $5\norm{F}^2$ is equal to 
    \item[33.] Let $y\brak{t}$ be the solution of the initial value problem $$y"+4y = \begin{cases}t&0\leq t\leq 2,\\2&2<t<\infty \end{cases} and y\brak{0} = y'\brak{0} = 0.$$ If $\alpha = y\brak{\frac{\pi}{2}}$ then the value of $\frac{4}{\pi}\alpha$ is ............(rounded off to 2 decimal places).
    \item[34.] Consider $R^4$ with the inner product $<x,y> = \sum_{i=1}^{4}x_iy_i,$ for $x = \brak{x_1, x_2, x_3, x_4}$ and $y = \brak{y_1, y_2, y_3, y_4}.$\\Let $M = \{\brak{x_1, x_2, x_3, x_4}\in R^4 : x_1=x_3\}$ and $M^\bot$ denote the orthogonal complement of M. The dimension of $M^{\bot}$ is equal to .......... 
    \item[35.] Let $M = \begin{bmatrix}3&-1&-2\\0&2&4\\0&0&1\end{bmatrix}$ and $I = \begin{bmatrix}1&0&0\\0&1&0\\0&0&1\end{bmatrix}$. If $6M^{-1} = M^2 - 6M + \alpha I$ for some $\alpha \in R$, then the value of $\alpha$ is equal to .......... 
    \item[36.] Let $GL_2\brak{C}$ denote the group of $2 \times 2$ invertible complex matrices with usual
    matrix multiplication. For $S, T \in GL_2\brak{C}, < S, T >$ denotes the subgroup generated by S and T. Let $S = \begin{bmatrix}0&-1\\1&0\end{bmatrix} \in GL_2\brak{C}$and $G_1, G_2, G_3$be three
    subgroups of $GL_2\brak{C}$ given by\\
    \begin{align*}
        G_1 = <S,T_1>, \text{ where }T_1 = \begin{bmatrix}i&0\\0&i\end{bmatrix} \\
        G_2 = <S,T_2>, \text{ where }T_2 = \begin{bmatrix}i&0\\0&-i\end{bmatrix}\\
        G_1 = <S,T_3>, \text{ where }T_3 = \begin{bmatrix}0&1\\1&0\end{bmatrix}
    \end{align*}
    Let $Z\brak{G_i}$ denote  the center of $G_i$ for i = 1,2,3.
    
    Which of the following statements is correct?    
    \begin{enumerate}[label=(\Alph*)]
        \item $G_1$ is isomorphic to $G_3$
        \item $Z\brak{G_1}$ is isomorphic to $Z\brak{G_2}$
        \item $Z\brak{G_3} = \begin{bmatrix}1&0\\0&1\end{bmatrix}$
        \item $Z\brak{G_2}$ is isomorphic to $Z\brak{G_3}$
    \end{enumerate}
    \item[37.] Let $l^2 = \{\brak{x_1, x_2, x_3,....}: x_n \in R \text{ for all }n\in N \text{ and }\sum_{n=1}^{\infty}x_n^2 <\infty\}.$\\ For a sequence $\brak{x_1, x_2, x_3, ........}\in l^2,$ define $\norm{\brak{x_1,x_2,x_3,.....}}_2 = \brak{\sum_{n=1}^{\infty}x_n^2}^\frac{1}{2}$.Let $S:\brak{l^2, \norm{.}_2}\to \brak{l^2, \norm{.}_2} \text{ and }T:\brak{l^2, \norm{.}_2}\to \brak{l^2, \norm{.}_2}$ be defined by $$S\brak{x_1, x_2, x_3,.....}=\brak{y_1,y_2,y_3,......}, \text{ where  } y_n = \begin{cases}0 & n=1\\x_{n-1}&n\geq 2\end{cases}$$  $$T(x_1, x_2, x_3, \dots) = (y_1, y_2, y_3, \dots), \quad \text{where } y_n = \begin{cases} 0, & \text{if } n \text{ is odd} \\x_n, & \text{if } n \text{ is even}\end{cases}$$
    \begin{enumerate}[label=(\Alph*)]
        \item S is a compact linear map and T is NOT a compact linear map
        \item S is NOT a compact linear map and T is a compact linear map
        \item both S and T are compact linear maps
        \item NEITHER S NOR T is a compact linear map
    \end{enumerate}
    \item[38.] Let $$
c_{00} = \{(x_1, x_2, x_3, \dots) : x_i \in \mathbb{R}, i \in \mathbb{N}, x_i \neq 0 \text{ only for finitely many indices } i\}.$$

For $\brak{x_1, x_2, x_3, \dots} \in c_{00}$, let
$$
\abs{\brak{x_1, x_2, x_3, \dots}}_{\infty} = \sup \{ \abs{x_i} : i \in \mathbb{N} \}.
$$

Define $F, G : \brak{c_{00}, \|\cdot\|_{\infty}} \to \brak{c_{00}, \|\cdot\|_{\infty}} $ by
$$
F\brak{\brak{x_1, x_2, \dots, x_n, \dots}} = \brak{ \brak{1 + \frac{1}{1}}x_1, \brak{2 + \frac{1}{2}}x_2, \dots, \brak{n + \frac{1}{n}}x_n, \dots},
$$
$$
G\brak{\brak{x_1, x_2, \dots, x_n, \dots}} = \brak{ \frac{x_1}{1 + \frac{1}{1}}, \frac{x_2}{2 + \frac{1}{2}}, \dots, \frac{x_n}{n + \frac{1}{n}}, \dots },
$$
for all $(x_1, x_2, \dots, x_n, \dots) \in c_{00}$
\begin{enumerate}[label=(\Alph*)]
    \item F is continuous but G is NOT continuous
    \item F is NOT continuous but G is continuous
    \item both F and G are continuous 
    \item NEITHER F NOR G is continuous 
\end{enumerate}
\item[39.] Consider the Cauchy problem
$$
x \frac{\partial u}{\partial x} + y \frac{\partial u}{\partial y} = u;
$$
$$
u = f(t) \text{ on the initial curve } \Gamma = (t, t), \quad t > 0.
$$

Consider the following statements:

\begin{itemize}
    \item $ P $: If $ f\brak{t} = 2t + 1 $, then there exists a unique solution to the Cauchy problem in a neighbourhood of \( \Gamma \).
    \item $ Q $: If $ f\brak{t} = 2t - 1 $, then there exist infinitely many solutions to the Cauchy problem in a neighbourhood of \( \Gamma \).
\end{itemize}
\begin{enumerate}[label=(\Alph*)]
    \item both P and Q are TRUE
    \item P is FALSE and Q is TRUE
    \item P is TRUE and Q is FALSE
    \item both P and Q are FALSE

\end{enumerate}

 
\end{enumerate}
\end{document}