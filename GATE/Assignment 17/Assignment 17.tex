\let\negmedspace\undefined
\let\negthickspace\undefined
\documentclass[journal]{IEEEtran}
\usepackage[a5paper, margin=10mm, onecolumn]{geometry}
\usepackage{lmodern} % Ensure lmodern is loaded for pdflatex
 % Include tfrupee package
\setlength{\headheight}{1cm} % Set the height of the header box
\setlength{\headsep}{0mm}     % Set the distance between the header box and the top of the text
\usepackage{enumitem}
\usepackage{gvv-book}
\usepackage{gvv}
\usepackage{cite}
\usepackage{amsmath,amssymb,amsfonts,amsthm}
\usepackage{algorithmic}
\usepackage{graphicx}
\usepackage{textcomp}
\usepackage{xcolor}
\usepackage{txfonts}
\usepackage{listings}
\usepackage{enumitem}
\usepackage{mathtools}
\usepackage{gensymb}
\usepackage{graphicx}
\usepackage{wrapfig}
\usepackage{comment}
\usepackage[breaklinks=true]{hyperref}
\usepackage{tkz-euclide} 
\usepackage{listings}
% \usepackage{gvv}                                        
\def\inputGnumericTable{}                                 
\usepackage[latin1]{inputenc}                                
\usepackage{color}                                            
\usepackage{array}                                            
\usepackage{longtable}                                       
\usepackage{calc}                                             
\usepackage{multirow}                                         
\usepackage{hhline}                                           
\usepackage{ifthen}                                           
\usepackage{lscape}
\begin{document}

\bibliographystyle{IEEEtran}
\vspace{3cm}


\author{AI24BTECH11008- Sarvajith
}
\title{Assignment 7}
% \maketitle
% \newpage
% \bigskip
{\let\newpage\relax\maketitle}
\title{2017, XE}
\renewcommand{\thefigure}{\theenumi}
\renewcommand{\thetable}{\theenumi}
\setlength{\intextsep}{10pt} % Space between text and floats
\numberwithin{equation}{enumi}
\numberwithin{figure}{enumi}
\renewcommand{\thetable}{\theenumi}
\begin{enumerate}
    \item[1.] If $$\int_{0}^{\frac{\pi}{\alpha}}\int_{x}^{\frac{\pi}{\alpha}}\frac{siny}{y}dydx = \frac{1}{2}$$ for some $\alpha \geq 1$, then the value of $\alpha$ is ............. \hfill (2017)
    \item[2.] Three fair dice are rolled simultaneously. The probability of getting a sum of 5 is \hfill (2017)
    \begin{enumerate}[label = (\Alph*)]
        \item $\frac{1}{108}$
        \item $\frac{1}{72}$
        \item $\frac{1}{54}$
        \item $\frac{1}{36}$
    \end{enumerate} 
    \item[3.] Suppose $\alpha, \beta, \gamma$ and $\delta$ are constants such that $$p\brak{x} = \delta + \gamma\brak{x+1}+\beta x\brak{x+1} + \alpha x\brak{x+1}\brak{x-1}$$ is the interpolating polynomial for the data $\brak{-1,-3},\brak{0,1},\brak{1,-1},\brak{2,-3}$. Then the value of $\gamma - \beta$ is ..............\hfill (2017)
    \item[4.] Consider the ordinary differential equation $$y'' + \alpha y' + \beta y =0$$, where $\alpha$ and $\beta$ are constants. If $y\brak{x} = xe^x$ is a solution of the above equation, then the value of $\beta-\alpha$ is .............\hfill (2017)
    \item[5.] Consider the system of linear equations \begin{align*}2x_2 + x_3 =0,\\-2x_1 - x_3=0,\\-x_1 + x_2=1\end{align*}. The above system has \hfill (2017)
    \begin{enumerate}[label = (\Alph*)]
        \item a unique solution 
        \item infinite number of solutions
        \item no solution
        \item only 2 distinct solutions
    \end{enumerate}
    \item[6.] Let C be a simple closed curve enclosing the region R in the xy-plane. Let C be oreinted counterclockwise. If the value of the integral $$\oint_C \brak{y+e^{x^2}}dx + \brak{3x+cosy}dy$$ is 16, then the area of R is .................... \hfill (2017)
    \item[7.] Consider the ordinary differential equation $$x^2y''+xy'-y=x, x>0$$. In terms of arbitrary constants $c_1$ and $c_2$, te general solution of the above equation is \hfill (2017)
    \begin{enumerate}[label = (\Alph*)]
        \item $y\brak{x} = c_1x+c_2x^{-1}+x^3$
        \item $y\brak{x} = c_1x+c_2x^{-1}+\frac{1}{2}x$
        \item $y\brak{x} = c_1x+c_2x^{-1}+\frac{1}{2}x\ln x$
        \item $y\brak{x} = c_1x+c_2+x^{-1}$
    \end{enumerate}
    \item[8.] Let $f:R\rightarrow R$ and $g:R\rightarrow R$ be defined by \begin{align*}f\brak{x}=\begin{cases}x\brak{sinx}cos\frac{1}{x} & x\neq 0\\0,&x=0\end{cases}\\g\brak{x}=\begin{cases}xcos\frac{1}{x},&x\neq 0\\0 &x\neq 0\end{cases}\end{align*} where $R$ denotes the set of real numbers. Then at x=0, \hfill (2017)
    \begin{enumerate}[label = (\Alph*)]
       \item f is differntiable but g is not differentiable
       \item f is not differentiable but g is differentiable
       \item both f and g are differentiable
       \item neither f nor g is differentiable
    \end{enumerate}
    \item[9.] If $u\brak{x,t} = g\brak{t}sinx$ is the solution of the wave equation\\$$u_{tt} = u_{xx}, t>0, 0<x<\pi$$ 
     with the initial conditions $$u\brak{x,0}=2sinx, u_t\brak{x,0}=0, 0\leq x\leq \pi$$ The boundary conditions $$u\brak{0,t}=u\brak{\pi,t}=0, t\geq 0$$then the value of $g\brak{\frac{\pi}{3}}$ is ............. \hfill (2017)
    \item[10.] Let $$I = \int_{0}^{1}\frac{1}{1+t}dt + \frac{\pi i}{2}\int_{0}^{1}\frac{e^{\frac{i\pi t}{2}}}{1+e^{\frac{i\pi t}{2}}}dt - i\int_{0}^{1}\frac{1}{1+it}dt$$, where t is real variable and $i = \sqrt{-1}$. The value of I is ................ \hfill (2017)
    \item[11.] Let $$a_k = 2^{-k}k^4 sink  $$ and $$b_k = 2^{-k^2}ksin^2k$$ for k=1,2..... then \hfill (2017)
    \begin{enumerate}[label=(\Alph*)]
        \item $\sum_{k=1}^{\infty}a_k$ converges but $\sum_{k=1}^{\infty}b_k $ does NOT converge 
        \item $\sum_{k=1}^{\infty}a_k$ does NOT converges but $\sum_{k=1}^{\infty}b_k $ converges 
        \item both $\sum_{k=1}^{\infty}a_k$ and $\sum_{k=1}^{\infty}b_k $ converge 
        \item neither $\sum_{k=1}^{\infty}a_k$ nor $\sum_{k=1}^{\infty}b_k $ converges 
    \end{enumerate}
    \item[12.] In a given flow field, the velocity vector in Cartesian coordinate system is given as:$$\overrightarrow{V}=\brak{x^2+y^2+z^2}\hat{i} +\brak{xy+yz+y^2}\hat{j}+\brak{xz-z^2}\hat{k}  $$What is the volume dilation rate of the fluid at a point where x=1, y=2 and z=3?\hfill (2017)
    \begin{enumerate}[label=(\Alph*)]
        \item 6
        \item 5
        \item 10
        \item 0
    \end{enumerate}
    \item[13.] A steady, incompressible, two-dimensional velociy fluid in Cartesian coordinate system is represented by the following expression.$$\overrightarrow{V} = \brak{0.7+0.4x}\hat{i} + \brak{1.2  0.4y}\hat{j}$$The coordinates of the point $\brak{x,y}$ in the flow field having "zero" velocity is, \hfill (2017)
    \begin{enumerate}[label=(\Alph*)]
        \item $\brak{1.75,-3}$
        \item $\brak{-1.75,3}$
        \item $\brak{1.75,3}$
        \item $\brak{-1.75,-3}$
    \end{enumerate}
\end{enumerate}
\end{document}