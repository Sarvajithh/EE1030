\let\negmedspace\undefined
\let\negthickspace\undefined
\documentclass[journal]{IEEEtran}
\usepackage[a5paper, margin=10mm, onecolumn]{geometry}
\usepackage{lmodern} % Ensure lmodern is loaded for pdflatex
 % Include tfrupee package
\setlength{\headheight}{1cm} % Set the height of the header box
\setlength{\headsep}{0mm}     % Set the distance between the header box and the top of the text
\usepackage{enumitem}
\usepackage{gvv-book}
\usepackage{gvv}
\usepackage{cite}
\usepackage{amsmath,amssymb,amsfonts,amsthm}
\usepackage{algorithmic}
\usepackage{graphicx}
\usepackage{textcomp}
\usepackage{xcolor}
\usepackage{txfonts}
\usepackage{listings}
\usepackage{enumitem}
\usepackage{mathtools}
\usepackage{gensymb}
\usepackage{graphicx}
\usepackage{wrapfig}
\usepackage{comment}
\usepackage[breaklinks=true]{hyperref}
\usepackage{tkz-euclide} 
\usepackage{listings}
% \usepackage{gvv}                                        
\def\inputGnumericTable{}                                 
\usepackage[latin1]{inputenc}                                
\usepackage{color}                                            
\usepackage{array}                                            
\usepackage{longtable}                                       
\usepackage{calc}                                             
\usepackage{multirow}                                         
\usepackage{hhline}                                           
\usepackage{ifthen}                                           
\usepackage{lscape}
\begin{document}

\bibliographystyle{IEEEtran}
\vspace{3cm}


\author{AI24BTECH11008- Sarvajith
}
\title{Assignment 14}
% \maketitle
% \newpage
% \bigskip
{\let\newpage\relax\maketitle}
\title{2013, EE}
\renewcommand{\thefigure}{\theenumi}
\renewcommand{\thetable}{\theenumi}
\setlength{\intextsep}{10pt} % Space between text and floats
\numberwithin{equation}{enumi}
\numberwithin{figure}{enumi}
\renewcommand{\thetable}{\theenumi}
\begin{enumerate}
    \item[27.] In the circuit shown below, the knee current of the ideal Zener diode is 10 mA. To maintain 5 V
    across $R_L$, the minimum value of RL in Ω and the minimum power rating of the Zener diode in mW
    respectively are \hfill (2013)
    \begin{figure}[!ht]
        \centering
        \resizebox{0.4\textwidth}{!}{%
    \begin{circuitikz}
    \tikzstyle{every node}=[font=\small]
    \draw [short] (-5,13.25) -- (-5,13.25);
    \draw [short] (-3.5,13.75) -- (-1.5,13.75);
    \draw [short] (-3.5,13.75) -- (-3.5,12.5);
    \draw (-3.5,12.5) to[battery1] (-3.5,11.25);
    \draw (-3.5,11.25) to[short] (-1.5,11.25);
    \draw (-2,11.25) to (-2,11) node[ground]{};
    \draw (-1.5,11.25) to[empty Schottky diode] (-1.5,12.75);
    \draw (-1.5,13.75) to[R] (-1.5,12.75);
    \draw [->, >=Stealth] (-1.5,12.5) -- (-0.75,12.5);
    \draw [short] (-0.75,12.5) -- (-0.25,12.5);
    \draw (-0.25,12.5) to[R] (-0.25,11.25);
    \draw (-1.5,11.25) to[short] (-0.25,11.25);
    \node [font=\small] at (-1,13.25) {100$\Omega$};
    \node [font=\small] at (-4,12) {10V};
    \node [font=\small] at (-2.25,12) {$V_z =5V$};
    \node [font=\small] at (0.25,11.75) {$R_L$};
    \end{circuitikz}
    }% % Specify the path to your TikZ file
        \caption{1}
        %\label{fig2}
    \end{figure}
    \begin{enumerate}[label=(\Alph*)]
        \item 125 and 125
        \item 125 and 250
        \item 250 and 125
        \item 250 and 250
    \end{enumerate}
    \item[28.] The open-loop transfer function of a dc motor is given as frac$\frac{\omega\brak{s}}{V_a\brak{s}} = \frac{10}{1+10s}.$ When connected in
    feedback as shown below, the approximate value of $K_a$ that will reduce the time constant of the
    closed loop system by one hundred times as compared to that of the open-loop system is \hfill (2013)
    \begin{figure}[!ht]
        \centering
        \resizebox{0.4\textwidth}{!}{%
        \begin{circuitikz}
        \tikzstyle{every node}=[font=\normalsize]
        \draw  (3.25,9.25) circle (0.5cm);
        \draw [->, >=Stealth] (3.75,9.25) -- (4.75,9.25);
        \draw [->, >=Stealth] (1.25,9.25) -- (2.75,9.25);
        \draw [->, >=Stealth] (3.25,7.5) -- (3.25,8.75);
        \draw [short] (3.25,7.5) -- (10.75,7.5);
        \draw  (4.75,9.75) rectangle (6.25,8.5);
        \node [font=\normalsize] at (5.5,9.25) {$K_a$};
        \draw [->, >=Stealth] (6.25,9.25) -- (7.75,9.25);
        \draw  (7.75,9.75) rectangle (9.75,8.5);
        \node [font=\normalsize] at (8.75,9.25) {$\frac{10}{1+10s}$};
        \draw [->, >=Stealth] (9.75,9.25) -- (11.5,9.25);
        \draw [short] (10.75,7.5) -- (10.75,9.25);
        \node [font=\normalsize] at (10.25,9.5) {$\omega$(s)};
        \node [font=\normalsize] at (6.75,9.5) {$V_a(s)$};
        \node [font=\normalsize] at (3,9.25) {+};
        \node [font=\normalsize] at (3.25,9) {-};
        \node [font=\normalsize] at (2,9.5) {R(s)};
        \end{circuitikz}
        }% % Specify the path to your TikZ file
        \caption{2}
        %\label{fig2}
    \end{figure}
    \begin{enumerate}[label=(\Alph*)]
        \item 1
        \item 5
        \item 10
        \item 100
    \end{enumerate}
    \item[29.] In the circuit shown below, if the source voltage $V_S = 100\angle53.13^{\circ} V$ then the Thevenin's equivalent
    voltage in Volts as seen by the load resistance $R_L$ is \hfill (2013)
    \begin{figure}[!ht]
        \centering
        \resizebox{0.4\textwidth}{!}{%
        \begin{circuitikz}
        \tikzstyle{every node}=[font=\small]
        \draw (3,9) to[R] (3,9);
        \draw (3.5,9.25) to[R] (5.5,9.25);
        \draw (5.5,9.25) to[L ] (7.5,9.25);
        \draw (3.5,7.5) to[sinusoidal voltage source, sources/symbol/rotate=auto] (3.5,9.25);
        \draw (3.5,7.5) to[short] (7.5,7.5);
        \draw  (7.75,8.375) -- (7.5,8) -- (7.25,8.375) -- (7.5,8.75) -- cycle;
        \draw [short] (7.5,9.25) -- (7.5,8.75);
        \draw [short] (7.5,7.5) -- (7.5,8);
        \draw [->, >=Stealth] (4.75,8.25) .. controls (5,9.25) and (5.75,8.75) .. (5.5,7.75) ;
        \draw [<->, >=Stealth] (6,9) -- (7,9)node[pos=0.5, fill=white]{$V_{L1}$};
        \node [font=\small] at (7,8.25) {$j40I_2$};
        \node [font=\small] at (4.5,9.75) {$3\Omega$};
        \node [font=\small] at (6.5,9.75) {$j4\Omega$};
        \node [font=\small] at (7.5,8.5) {$+$};
        \node [font=\small] at (7.5,8.25) {$-$};
        \node [font=\large] at (6,9.5) {$+$};
        \node [font=\large] at (7,9.5) {$-$};
        \node [font=\normalsize] at (3,8.25) {$V_s$};
        \draw [short] (7.5,7.5) -- (8.5,7.5);
        \draw  (8.25,8.375) -- (8.5,8) -- (8.75,8.375) -- (8.5,8.75) -- cycle;
        \draw [short] (8.5,8) -- (8.5,7.5);
        \draw [short] (8.5,8.75) -- (8.5,9.25);
        \draw (8.5,9.25) to[L ] (10.5,9.25);
        \node [font=\normalsize] at (9.5,9.75) {j6$\Omega$};
        \draw (10.5,9.25) to[R] (11.75,9.25);
        \draw (11.75,7.5) to[R] (11.75,9.25);
        \draw (8.5,7.5) to[short] (11.75,7.5);
        \node [font=\small] at (8.5,8.5) {+};
        \node [font=\small] at (8.5,8.25) {-};
        \node [font=\small] at (11.25,9.5) {5$\Omega$};
        \node [font=\small] at (12.5,8.25) {$R_L = 10\Omega$};
        \draw [->, >=Stealth] (9.5,8.25) .. controls (10,9.25) and (11,8.75) .. (10.25,7.75) ;
        \node [font=\small] at (10,7.75) {$I_2$};
        \node [font=\small] at (5.25,8) {$I_1$};
        \node [font=\small] at (9,8.75) {$10V_{L1}$};
        \end{circuitikz}
        }%% Specify the path to your TikZ file
        \caption{3}
        %\label{fig2}
    \end{figure}
    \begin{enumerate}[label=(\Alph*)]
        \item $100\angle90^{\circ}$
        \item $800\angle0^{\circ}$
        \item $800\angle90^{\circ}$
        \item $100\angle60^{\circ}$
    \end{enumerate}
    \item[30.] Three capacitors $C_1, C_2, \text{and} C_3$, whose values are $10\mu F, 5\mu F, \text{and} 2\mu F$ respectively, have
    breakdown voltages of 10V, 5V, and 2V respectively. For the interconnection shown, the maximum
    safe voltage in Volts that can be applied across the combination and the corresponding total charge
    in $\mu C$ stored in the effective capacitance across the terminals are respectively\hfill (2013)
    \begin{figure}[!ht]
        \centering
        \resizebox{0.4\textwidth}{!}{%
        \begin{circuitikz}
        \tikzstyle{every node}=[font=\normalsize]
        \draw (4.75,8.5) to[short, -o] (3.25,8.5) ;
        \draw [short] (4.25,8.5) -- (4.25,7.75);
        \draw (4.25,7.75) to[curved capacitor] (6.25,7.75);
        \draw (3.5,8.5) to[curved capacitor] (5.5,8.5);
        \draw (5,8.5) to[curved capacitor] (6.25,8.5);
        \draw [short] (6.25,8.5) -- (6.25,7.75);
        \draw (6.25,8.5) to[short, -o] (7.5,8.5) ;
        \node [font=\normalsize] at (4.5,9) {$C_2$};
        \node [font=\normalsize] at (5.5,9) {$C_3$};
        \node [font=\normalsize] at (5.25,7.25) {$C_1$};
        \end{circuitikz}
        }% % Specify the path to your TikZ file
        \caption{4}
        %\label{fig2}
    \end{figure}
    \begin{enumerate}[label=(\Alph*)]
        \item 2.8 and 36
        \item 7 and 119
        \item 1.8 and 32
        \item 7 and 80
    \end{enumerate}
    \item[31.] A voltage $1000sin\omega t$ Volts is applied across YZ. Assuming ideal diodes, the voltage measured
    across WX in Volts is \hfill (2013)
    \begin{figure}[!ht]
        \centering
        \resizebox{0.4\textwidth}{!}{%
    \begin{circuitikz}
    \tikzstyle{every node}=[font=\Large]
    \draw [short] (3,10) -- (3,6.75);
    \draw (3,10) to[D] (5,10);
    \draw (5,7.5) to[D] (3,7.5);
    \draw (5,10) to[R] (5,9);
    \draw (5,7.5) to[short, -o] (5,7.75) ;
    \draw (5,9) to[short, -o] (5,8.5) ;
    \draw (5,7.5) to[D] (6.75,7.5);
    \draw (6.75,10) to[D] (5,10);
    \draw (3,8.25) to[short, -o] (4,8.25) ;
    \draw (6.75,8.25) to[short, -o] (6,8.25) ;
    \draw [short] (6.75,10) -- (6.75,6.75);
    \node [font=\normalsize] at (4.25,8.25) {W};
    \node [font=\normalsize] at (5,8.25) {Y};
    \node [font=\normalsize] at (5.25,7.75) {Z};
    \node [font=\normalsize] at (5.75,8.25) {X};
    \node [font=\normalsize] at (5,7.25) {1k$\Omega$};
    \draw (3,6.75) to[R] (6.75,6.75);
    \node [font=\Large] at (4.25,6.5) {+};
    \node [font=\Large] at (5.5,6.5) {-};
    \end{circuitikz}
    }%% Specify the path to your TikZ file
        \caption{5}
        %\label{fig2}
    \end{figure}
    \begin{enumerate}[label=(\Alph*)]
        \item $sin\omega t$
        \item $\frac{sin\omega t+\abs{sin\omega t}}{2}$
        \item  $\frac{sin\omega t-\abs{sin\omega t}}{2}$
        \item 0 for all t
    \end{enumerate}
    \item[32.] The separately excited dc motor in the figure below has a rated armature current of 20 A and a rated
    armature voltage of 150 V. An ideal chopper switching at 5 kHz is used to control the armature
    voltage. If $L_a$= 0.1 mH, $R_a= 1\varOmega$ , neglecting armature reaction, the duty ratio of the chopper to
    obtain 50\% of the rated torque at the rated speed and the rated field current is \hfill (2013)
    \begin{figure}[!ht]
        \centering
        \resizebox{0.4\textwidth}{!}{%
    \begin{circuitikz}
    \tikzstyle{every node}=[font=\normalsize]
    \draw [short] (2.5,10.5) -- (2.5,10.5);
    \draw [short] (3,10.25) -- (4,10.25);
    \draw [short] (4,10.25) -- (4.5,9.75);
    \draw [short] (4,9.75) -- (5.5,9.75);
    \draw [short] (4,9.5) -- (5.5,9.5);
    \draw [short] (4.75,9.5) -- (4.75,8.75);
    \draw [->, >=Stealth] (5,9.75) -- (5.5,10.25);
    \draw [short] (5.5,10.25) -- (5.75,10.25);
    \draw [short] (5.75,10.25) -- (7.25,10.25);
    \draw (7.25,7.75) to[D] (7.25,10.25);
    \draw [short] (3,10.25) -- (3,9.25);
    \draw  (3,9) circle (0.25cm);
    \draw [short] (3,8.75) -- (3,7.75);
    \draw [short] (3,7.75) -- (8.5,7.75);
    \draw [short] (8.5,7.75) -- (8.5,8.75);
    \draw [short] (7.25,10.25) -- (8.5,10.25);
    \draw [short] (8.5,10.25) -- (8.5,9.25);
    \draw  (8.5,9) circle (0.25cm);
    \node [font=\small] at (3,9.25) {+};
    \node [font=\large] at (3,9.25) {+};
    \node [font=\large] at (2.5,9.25) {+};
    \node [font=\large] at (2.5,8.75) {-};
    \node [font=\normalsize] at (9.25,9) {$L_a,R_a$};
    \end{circuitikz}
    }% % Specify the path to your TikZ file
        \caption{6}
        %\label{fig2}
    \end{figure}
    \begin{enumerate}[label=(\Alph*)]
        \item 0.4
        \item 0.5
        \item 0.6
        \item 0.7
    \end{enumerate}
    \item[33.] For a power system network with n nodes, $Z_33$ of its bus impedance matrix is j0.5 per unit. The
    voltage at node 3 is $1.3\angle-10^{\circ}$ per unit. If a capacitor having reactance of -j3.5 per unit is now
    added to the network between node 3 and the reference node, the current drawn by the capacitor per
    unit is \hfill (2013)
    \begin{enumerate}[label=(\Alph*)]
        \item $0.325\angle-100^{\circ}$
        \item $0.325\angle80^{\circ}$
        \item $0.371\angle-100^{\circ}$
        \item $0.433\angle80^{\circ}$
    \end{enumerate}
    \item[34.] A dielectric slab with $500mm\times500 mm$ cross-section is 0.4 m long. The slab is subjected to a uniform
    electric field of $\mathbf{E}=6a_x + 8a_y \frac{kV}{mm}$. The relative permittivity of the dielectric material is equal to 2. The value of constant $\epsilon_0 $\text{is} $8.85\times 10^{-12}$. The energy stored in the dielectric in Joules is \hfill (2013)
    
    \begin{enumerate}[label=(\Alph*)]
        \item  $8.85\times 10^{-11}$ 
        \item $8.85\times 10^{-5}$ 
        \item  88.5
        \item  885 
    \end{enumerate} 
    
    \item[35.] A matrix has eigenvalues -1 and -2. The corresponding eigenvectors are $\begin{bmatrix}1\\-1\end{bmatrix}$ and $\begin{bmatrix}1\\2\end{bmatrix}$ respectively. The matrix is \hfill (2013)
    \begin{enumerate}[label=(\Alph*)]
        \item $\begin{bmatrix}1&1\\-1&-2\end{bmatrix}$
        \item $\begin{bmatrix}1&2\\-2&-4\end{bmatrix}$
        \item $\begin{bmatrix}-1&0\\0&-2\end{bmatrix}$
        \item $\begin{bmatrix}0&1\\-2&-3\end{bmatrix}$
    \end{enumerate}
    \item[36.] $\int\frac{z^2-4}{z^2+4}$  evaluated anticlockwise around the circle $\abs{z-i}=2$, where $i=\sqrt{-1}$ is \hfill (2013)
     \begin{enumerate}[label=(\Alph*)]
        \item $-4\pi$
        \item 0
        \item $2+\pi$
        \item $2+2i$
     \end{enumerate}
    \item[37.] The clock frequency applied to the digital circuit shown in the figure below is 1 kHz. If the initial
    state of the output Q of the flip-flop is '0', then the frequency of the output waveform Q in kHz is \hfill (2013)
    \begin{figure}[!ht]
        \centering
        \resizebox{0.4\textwidth}{!}{%
        \begin{circuitikz}
        \tikzstyle{every node}=[font=\normalsize]
        \draw [short] (3,10.5) -- (9.75,10.5);
        \draw [short] (9.75,10.5) -- (9.75,9.5);
        \draw [short] (3,10.5) -- (3,9.75);
        \draw (3,10) to[short] (3.5,10);
        \draw (3,9.5) to[short] (3.5,9.5);
        \draw (3.5,10) node[ieeestd xor port, anchor=in 1, scale=0.89](port){} (port.out) to[short] (5.5,9.75);
        \draw [short] (3,9.5) -- (2,9.5);
        \draw [short] (2,9.5) -- (2,7.75);
        \draw (3.25,8.5) to[short] (3.5,8.5);
        \draw (3.25,8) to[short] (3.5,8);
        \draw (3.5,8.5) node[ieeestd xnor port, anchor=in 1, scale=0.89](port){} (port.out) to[short] (5.25,8.25);
        \draw [short] (3,9.75) .. controls (3.25,9.75) and (3.5,9.5) .. (3,9.25);
        \draw [short] (3,9.25) -- (3,8.5);
        \draw [short] (3,8.5) -- (3.25,8.5);
        \draw [short] (2,8) -- (3.25,8);
        \draw [short] (2,8) -- (2,7.25);
        \draw [short] (2,7.25) -- (9.75,7.25);
        \draw [short] (5.5,9.75) -- (5.5,9.25);
        \draw [short] (5.5,9.25) -- (6,9.25);
        \draw [short] (5.25,8.25) -- (5.25,8.75);
        \draw (5.75,9.25) to[short] (6,9.25);
        \draw (5.75,8.75) to[short] (6,8.75);
        \draw (6,9.25) node[ieeestd nand port, anchor=in 1, scale=0.89](port){} (port.out) to[short] (7.75,9);
        \draw [short] (5.25,8.75) -- (5.75,8.75);
        \draw [short] (7.5,9) -- (8,9);
        \draw  (8,9.25) rectangle (9.25,8);
        \draw (9.25,9) to[short, -o] (10.75,9) ;
        \draw (9.25,8.5) to[short, -o] (10.75,8.5) ;
        \draw (7.25,8.5) to[short, -o] (8,8.5) ;
        \draw [short] (9.75,7.25) -- (9.75,8.5);
        \draw [short] (9.75,9.5) -- (9.75,9);
        \node [font=\normalsize] at (7.75,9.25) {X};
        \node [font=\normalsize] at (7.5,8.25) {Clk};
        \node [font=\normalsize] at (8.25,9) {T};
        \node [font=\normalsize] at (9,9) {Q};
        \node [font=\normalsize] at (9,8.25) {Q};
        \node [font=\normalsize] at (11,9.25) {Q};
        \node [font=\normalsize] at (11,8.25) {Q};
        \end{circuitikz}
        }%
        % Specify the path to your TikZ file
        \caption{7}
        %\label{fig2}
    \end{figure}
    \begin{enumerate}[label=(\Alph*)]
        \item 0.25
        \item 0.5
        \item 1
        \item 2
    \end{enumerate}
    \item[38.]  In the circuit shown below, $Q_1$ has negligible collector-to-emitter saturation voltage and the diode
    drops negligible voltage across it under forward bias. If $V_{cc}$ is +5 V, X and Y are digital signals
    with 0 V as logic 0 and $V_{cc}$ as logic 1, then the Boolean expression for Z is \hfill (2013)
    \begin{figure}[!ht]
        \centering
        \resizebox{0.4\textwidth}{!}{%
    \begin{circuitikz}
    \tikzstyle{every node}=[font=\normalsize]
    \draw (5.25,10.75) to[R] (5.25,9.25);
    \draw (5.25,9.25) to[short, -o] (8,9.25) ;
    \draw [short] (5.25,9.25) -- (5.25,9);
    \draw [short] (5.25,9) -- (5,8.75);
    \draw [short] (5,8.75) -- (5,8.5);
    \draw [->, >=Stealth] (5,8.5) -- (5.5,8.25);
    \draw (5.5,8.25) to (5.5,7) node[ground]{};
    \draw (6.75,9.25) to[D] (6.75,6.75);
    \draw (5,8.5) to[R] (2.25,8.5);
    \node [font=\normalsize] at (5.75,10) {$R_1$};
    \node [font=\normalsize] at (5.25,11) {$+V_{cc}$};
    \node [font=\normalsize] at (6.75,6.5) {Y};
    \node [font=\normalsize] at (7.75,9.5) {Z};
    \node [font=\normalsize] at (3.75,9) {$R_2$};
    \node [font=\normalsize] at (2,8.5) {X};
    \node [font=\normalsize] at (5.5,8.75) {$Q_1$};
    \node [font=\normalsize] at (7.5,8) {Diode};
    \end{circuitikz}
    }%  % Specify the path to your TikZ file
        \caption{8}
        %\label{fig2}
    \end{figure}
    \begin{enumerate}[label=(\Alph*)]
        \item XY
        \item $\bar{X}Y$
        \item $\bar{Y}X$
        \item $\bar{XY}$
    \end{enumerate}
    
    \item[39.] In the circuit shown below the op-amps are ideal. Then $V_{out}$ in Volts is \hfill (2013)
    \begin{figure}[!ht]
        \centering
        \resizebox{0.4\textwidth}{!}{%
        \begin{circuitikz}
        \tikzstyle{every node}=[font=\normalsize]
        \draw (10,8.5) node[op amp,scale=1, yscale=-1 ] (opamp2) {};
        \draw (opamp2.+) to[short] (8.5,9);
        \draw  (opamp2.-) to[short] (8.5,8);
        \draw (11.2,8.5) to[short](11.5,8.5);
        \draw (5.5,9) node[op amp,scale=1] (opamp2) {};
        \draw (opamp2.+) to[short] (4,8.5);
        \draw  (opamp2.-) to[short] (4,9.5);
        \draw (6.7,9) to[short](7,9);
        \draw (2,10.25) to[R] (4.75,10.25);
        \draw (4.75,10.25) to[R] (7.5,10.25);
        \draw (7,9) to[short] (8.75,9);
        \draw [short] (4,10.25) -- (4,9.5);
        \draw (4,8.5) to[R] (4,6.5);
        \draw [short] (7.5,10.25) -- (7.5,9);
        \draw (8.5,8) to[R] (8.5,4.5);
        \draw (8.5,7) to[R] (11.25,7);
        \draw [short] (11.25,8.5) -- (11.25,7);
        \draw (8.5,4.5) to (10.5,4.5) node[ground]{};
        \draw (10.25,8.75) to[short, -o] (10.25,9.5) ;
        \draw (10.25,8.25) to[short, -o] (10.25,7.5) ;
        \draw (5.75,8.75) to[short, -o] (5.75,8) ;
        \draw (5.75,9.25) to[short, -o] (5.75,9.75) ;
        \node [font=\normalsize] at (1.75,10.25) {-2V};
        \node [font=\normalsize] at (6,10.75) {1k$\Omega$};
        \node [font=\normalsize] at (3.25,10.5) {1k$\Omega$};
        \node [font=\normalsize] at (10.25,9.75) {+15V};
        \node [font=\normalsize] at (10.5,7.25) {-15V};
        \node [font=\normalsize] at (9.75,6.5) {1k$\Omega$};
        \node [font=\normalsize] at (8,6.25) {1k$\Omega$};
        \node [font=\normalsize] at (6,9.75) {+15V};
        \node [font=\normalsize] at (5.75,7.75) {-15V};
        \node [font=\normalsize] at (3.25,7.5) {1k$\Omega$};
        \node [font=\normalsize] at (11.5,8.75) {$V_{out}$};
        \node [font=\normalsize] at (4,6.25) {+1V};
        \end{circuitikz}
        }% % Specify the path to your TikZ file
        \caption{9}
        %\label{fig2}
    \end{figure}
    \begin{enumerate}[label=(\Alph*)]
        \item 4
        \item 6
        \item 8
        \item 10
    \end{enumerate} 
\end{enumerate}
\end{document}