\let\negmedspace\undefined
\let\negthickspace\undefined
\documentclass[journal]{IEEEtran}
\usepackage[a5paper, margin=10mm, onecolumn]{geometry}
\usepackage{lmodern} % Ensure lmodern is loaded for pdflatex
 % Include tfrupee package
\setlength{\headheight}{1cm} % Set the height of the header box
\setlength{\headsep}{0mm}     % Set the distance between the header box and the top of the text
\usepackage{enumitem}
%\usepackage{rupee} 
\usepackage{gvv-book}
\usepackage{gvv}
\usepackage{cite}
\usepackage{amsmath,amssymb,amsfonts,amsthm}
\usepackage{algorithmic}
\usepackage{graphicx}
\usepackage{textcomp}
\usepackage{xcolor}
\usepackage{txfonts}
\usepackage{listings}
\usepackage{enumitem}
\usepackage{mathtools}
\usepackage{gensymb}
\usepackage{graphicx}
\usepackage{wrapfig}
\usepackage{comment}
\usepackage[breaklinks=true]{hyperref}
\usepackage{tkz-euclide} 
\usepackage{listings}
% \usepackage{gvv}                                        
\def\inputGnumericTable{}                                 
\usepackage[latin1]{inputenc}                                
\usepackage{color}                                            
\usepackage{array}                                            
\usepackage{longtable}                                       
\usepackage{calc}                                             
\usepackage{multirow}                                         
\usepackage{hhline}                                           
\usepackage{ifthen}                                           
\usepackage{lscape}
\begin{document}

\bibliographystyle{IEEEtran}
\vspace{3cm}


\author{AI24BTECH11008- Sarvajith
}
\title{Assignment 26}
% \maketitle
% \newpage
% \bigskip
{\let\newpage\relax\maketitle}
\title{2023, PH}
\renewcommand{\thefigure}{\theenumi}
\renewcommand{\thetable}{\theenumi}
\setlength{\intextsep}{10pt} % Space between text and floats
\numberwithin{equation}{enumi}
\numberwithin{figure}{enumi}
\renewcommand{\thetable}{\theenumi}
\begin{enumerate}
    \item[27.] Let the minimum, maximum, mean and standard deviation values for the attribute
    income of data scientists be 46000, 170000, 96000, and 21000, respectively.
    The z-score normalized income value of 106000 is closest to which ONE of the
    following options?(in Indian rupees)
    \begin{enumerate}[label = (\Alph*)]
        \item 0.217
        \item 0.476
        \item 0.623
        \item 2.304
    \end{enumerate}
    \item[28.] Consider the following tree traversals on a full binary tree:
    \begin{itemize}
        \item[(i)] Preorder
        \item[(ii)] Inorder
        \item[(iii)] Postorder  
    \end{itemize}
    Which of the following traversal options is/are sufficient to uniquely reconstruct the full binary tree?
    \begin{enumerate}[label = (\Alph*)]
        \item i and ii
        \item ii and iii 
        \item i and iii 
        \item ii only 
    \end{enumerate} 
    \item[29.] Let x and y be two propositions. Which of the following statements is a tautology
    /are tautologies?
    \begin{enumerate}[label = (\Alph*)]
        \item $\brak{\neg x \land y}\implies \brak{y\implies x} $
        \item ${ x \land \neg y}\implies \brak{\neg x \implies y} $
        \item ${\neg x \land y}\implies \brak{\neg x \implies y} $
        \item ${ x \land\neg y}\implies \brak{y\implies x} $
    \end{enumerate}
    \item[30.] Consider sorting the followin array of integers in ascending order using an in-place Quicksort algorithm that uses the last element as the pivot 
    \begin{table}
        \centering
        \begin{tabular}{|c|c|c|c|c|}
            \hline
            60&70&80&90&100\\
            \hline 
        \end{tabular}
    \end{table}
    The minimum number of swaps performed during this Quicksort is ..........
    \item[31.] Consider the following two tables named Raider and Team in a relational database maintained by a Kabaddi league. The attribute ID in table team references the primary key of the Raider table, ID. 
    \begin{table}
        \centering
        \begin{tabular}{|c|c|c|c|}
            \hline
            \multicolumn{4}{|c|}{\textbf{Raider}} \\
            \hline
            \textbf{ID} & \textbf{Name} & \textbf{Raids} & \textbf{RaidPoints} \\
            \hline
            1 & Arjun & 200 & 250 \\
            2 & Ankush & 190 & 219 \\
            3 & Sunil & 150 & 200 \\
            4 & Reza & 150 & 190 \\
            5 & Pratham & 175 & 220 \\
            6 & Gopal & 193 & 215 \\
            \hline
        \end{tabular}
    \end{table}
    \begin{table}
        \centering
        \renewcommand{\arraystretch}{1.2} % Increase row height for better spacing
\begin{tabular}{|c|c|c|}
\hline
\multicolumn{3}{|c|}{\textbf{Team}} \\
\hline
\textbf{City} & \textbf{ID} & \textbf{BidPoints} \\
\hline
Jaipur & 2 & 200 \\
Patna & 3 & 195 \\
Hyderabad & 5 & 175 \\
Jaipur & 1 & 250 \\
Patna & 4 & 200 \\
Jaipur & 6 & 200 \\
\hline
\end{tabular}
    \end{table}
    The SQL query described below is execuuted on this database:\\
    SELECT *\\
    FROM Raider, Team\\
    WHERE Raider.ID = Team.ID AND City = "Jaipur" AND \\
    RaidPoints > 200;
    The number of rows returned by this query is ....... 
    \item[32.] The fundamental operations in a double-ended queue D are:\\
    insertFirst(e) - Insert a new element e at the beginning of D.\\
    insertLast(e) - Insert a new element e at the end of D.\\
    removeFirst() - Remove and return the first element of D.\\
    removeLast() - Remove and return the last element of D.\\
    In an empty double-ended queue, the following operations are performed:\\
    insertFirst(10)\\
    insertLast(32)\\
    a ←removeFirst()\\
    insertLast(28)\\
    insertLast(17)\\
    a ←removeFirst()\\
    a ← removeLast()\\
    The value of a is .............
    \item[33.] Let $f:R\to R$ be the function $f\brak{x} = \frac{1}{1+e^{-x}}$. The value of the derivative of f at x where $f\brak{x} = 0.4$ is ......(rounded off to 2 decimal places).
    \item[34.] The sample average of 50 data points is 40. The updated sample average after
    including a new data point taking the value of 142 is............... 
    \item[35.] Consider the $3\times 3$ matrix $M = \begin{bmatrix}1&2&3\\3&1&3\\4&3&6\end{bmatrix}$ \\ The determinant of $\brak{M^2 + 12M} $is .......... 
    \item[36.] A fair six-sided die (with faces numbered 1, 2, 3, 4, 5, 6) is repeatedly thrown
    independently.\\
    What is the expected number of times the die is thrown until two consecutive throws
    of even numbers are seen?
    \begin{enumerate}[label = (\Alph*)]
        \item 2
        \item 4
        \item 6
        \item 8
    \end{enumerate}
    \item[37.] Let $f: R\to R$ be a function $$f\brak{x} = \begin{cases} -x, & x<-2\\ax^2 + bx + c, &x\in [-2,2]\\ x,& x>2\end{cases}$$ 
    Which ONE of the following choices gives the values of a, b, c that make the
    function f continuous and differentiable?
    \begin{enumerate}[label = (\Alph*)]
        \item $a=\frac{1}{4}, b = 0, c=1$
        \item $a=\frac{1}{2}, b = 0, c=0$
        \item $a=0, b = 0, c=0$
        \item $a=1, b = 1, c=-4$
    \end{enumerate}
    \item[38.] Consider the following Python code:\\
    def count(child\_dict, i):\\
     if i not in child\_dict.keys():\\
     return 1\\
     ans = 1\\
     for j in child\_dict[i]:\\
     ans += count(child\_dict, j)\\
     return ans\\
    child\_dict = dict()\\
    child\_dict[0] = [1,2]\\
    child\_dict[1] = [3,4,5]\\
    child\_dict[2] = [6,7,8]\\
    print(count(child\_dict,0))\\
    Which ONE of the following is the output of this code?
    \begin{enumerate}[label = (\Alph*)]
        \item 6
        \item 1
        \item 8
        \item 9 
    \end{enumerate}
    \item[39.]  Consider the function computeS(X) whose pseudocode is given below:
    computeS(X)\\
    S[1] $\leftarrow$ 1\\
    for i $\leftarrow$2 to length(X)\\
     S[i] $\leftarrow$ 1\\
     if $X[i - 1] \leq$ X[i]\\
     S[i] $\leftarrow$ S[i] + S[i - 1]\\
     end if\\
    end for\\
    return S\\
    Which ONE of the following values is returned by the function computeS(X)
    for X = [6, 3, 5, 4, 10]?
    \begin{enumerate}[label = (\Alph*) ]
        \item [1,1,2,3,4]
        \item [1,1,2,3,3]
        \item [1,1,2,1,2]
        \item [1,1,2,1,5]
    \end{enumerate}

\end{enumerate}
\end{document}