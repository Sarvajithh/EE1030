\let\negmedspace\undefined
\let\negthickspace\undefined
\documentclass[journal]{IEEEtran}
\usepackage[a5paper, margin=10mm, onecolumn]{geometry}
\usepackage{lmodern} % Ensure lmodern is loaded for pdflatex
 % Include tfrupee package
\setlength{\headheight}{1cm} % Set the height of the header box
\setlength{\headsep}{0mm}     % Set the distance between the header box and the top of the text
\usepackage{enumitem}
\usepackage{gvv-book}
\usepackage{gvv}
\usepackage{cite}
\usepackage{amsmath,amssymb,amsfonts,amsthm}
\usepackage{algorithmic}
\usepackage{graphicx}
\usepackage{textcomp}
\usepackage{xcolor}
\usepackage{txfonts}
\usepackage{listings}
\usepackage{enumitem}
\usepackage{mathtools}
\usepackage{gensymb}
\usepackage{graphicx}
\usepackage{wrapfig}
\usepackage{comment}
\usepackage[breaklinks=true]{hyperref}
\usepackage{tkz-euclide} 
\usepackage{listings}
% \usepackage{gvv}                                        
\def\inputGnumericTable{}                                 
\usepackage[latin1]{inputenc}                                
\usepackage{color}                                            
\usepackage{array}                                            
\usepackage{longtable}                                       
\usepackage{calc}                                             
\usepackage{multirow}                                         
\usepackage{hhline}                                           
\usepackage{ifthen}                                           
\usepackage{lscape}
\begin{document}

\bibliographystyle{IEEEtran}
\vspace{3cm}


\author{AI24BTECH11008- Sarvajith
}
\title{Assignment 19}
% \maketitle
% \newpage
% \bigskip
{\let\newpage\relax\maketitle}
\title{2018, PH}
\renewcommand{\thefigure}{\theenumi}
\renewcommand{\thetable}{\theenumi}
\setlength{\intextsep}{10pt} % Space between text and floats
\numberwithin{equation}{enumi}
\numberwithin{figure}{enumi}
\renewcommand{\thetable}{\theenumi}
\begin{enumerate}
    \item[40.] Inside a large nucleus, a nucleon with mass $939 MeVc^{-2}$ has Fermi momentum $1.40 fm^{-1}$
    at absolute zero temperature. Its velocity is Xc , where the value of X is.............. (up
    to two decimal places). \hfill (2018)
    \item[41.] 4MeV $\gamma$ - rays emitted by the de-excitation of $^{19}F$ are attributed, assuming spherical
    symmetry, to the transition of protons from $1d_{3/2}$ state to $1d_{5/2}$ state. If the contribution
    of spin-orbit term to the total energy is written as C$\langle \overrightarrow{l}.\overrightarrow{s}\rangle$
    the magnitude of C is ..........MeV (up to one decimal place). \hfill (2018)
    \item[42.] An  $\alpha$ particle is emitted by a $^{230}_{90}Th$ nucleus. Assuming the potential to be purely
    Coulombic beyond the point of separation, the height of the Coulomb barrier is.............. MeV (up to two decimal places).\hfill (2018)
    \item[43.] For the transformation $$Q=\sqrt{2q}e^{-1+2\alpha}cosp, P = \sqrt{2q}e^{-1-\alpha}sinp$$ (where $\alpha$ is a constant) to be canonical, the value of $\alpha$ is .........\hfill (2018)
    \item[44.] Given $$\frac{d^2f\brak{x}}{dx^2}-2\frac{df\brak{x}}{dx}+f\brak{x} = 0,$$ and boundary conditions $f\brak{0} = 1$ and $f\brak{1}=0$, the value of the $f\brak{0.5}$ is ............(upto 2 decimal places)\hfill (2018)
    \item[45.] The absolute value of the integral $$\frac{5z^3 + 3z^2}{z^2-4}dz$$,over the circle $\abs{z-1.5}=1$ in complex plane, is ...........(upto 2 decimal places).\hfill (2018)
    \item[46.] A uniform circular disc of mass m and radius R is rotating with angular
    speed $\omega$ about an axis passing through its centre and making an angle
    $\theta = 30^{\circ}$ with the axis of the disc. If the kinetic energy of the disc is $\alpha m\omega^2 R^2$, the value of $\alpha$ is..............(up to two decimal places). \hfill (2018)
    \begin{figure}[!ht]
        \centering
        
    \resizebox{0.4\textwidth}{!}{%
    \begin{circuitikz}
    \tikzstyle{every node}=[font=\normalsize]
    \draw [ fill={rgb,255:red,27; green,24; blue,24} ] (12,15.5) rectangle (19.25,15.25);
    \draw [ fill={rgb,255:red,14; green,201; blue,225} ] (12,17.5) rectangle (17.25,15.5);
    \draw [ fill={rgb,255:red,31; green,30; blue,30} ] (17.25,16.5) circle (1cm);
    \draw [ color={rgb,255:red,252; green,252; blue,252}, <->, >=Stealth] (17.25,15.5) -- (17.25,17.5);
    \node [font=\normalsize, color={rgb,255:red,252; green,252; blue,252}] at (17.5,16.5) {D};
    \end{circuitikz}
    }%
    
  % Specify the path to your TikZ file
        \caption{1}
        %\label{fig2}
    \end{figure}
    \newpage
    \item[47.] The ground state energy of a particle of mass m in an infinite potential well is $E_0$. It
    changes to $E_0\brak{1+\alpha\times 10^{-3}}$, when there is a small potential pump of height $V_0 =\frac{\pi^2 \hslash^2}{50mL^2}$and width $a=\frac{L}{100}$ , as shown in the figure. The value of $\alpha$ is ........ (up to two
    decimal places). \hfill (2018)
    \begin{figure}[!ht]
        \centering
        
    \resizebox{1\textwidth}{!}{%
    \begin{circuitikz}
    \tikzstyle{every node}=[font=\small]
    \draw [short] (4,9.5) -- (8.75,9.5);
    \draw [short] (2.25,7.75) -- (10.75,7.75);
    \draw [short] (2.25,7.75) -- (4,9.5);
    \draw [short] (8.75,9.5) -- (10.75,7.75);
    \draw [short] (4,9.5) -- (4,7.75);
    \draw [short] (8.75,9.5) -- (8.75,7.75);
    \draw [short] (4,9.5) -- (5,7.75);
    \draw [short] (5,7.75) -- (5,9.5);
    \draw [short] (5,9.5) -- (6.25,7.75);
    \draw [short] (6.25,7.75) -- (7,9.5);
    \draw [short] (7,9.5) -- (7,7.75);
    \draw [short] (7,7.75) -- (8.75,9.5);
    \draw [short] (2.25,7.75) -- (2.5,7.5);
    \draw [short] (2.25,7.75) -- (2,7.5);
    \draw [short] (2,7.5) -- (2.5,7.5);
    \draw [short] (10.75,7.75) -- (11,7.5);
    \draw [short] (10.75,7.75) -- (10.5,7.5);
    \draw [short] (10.5,7.5) -- (11,7.5);
    \node [font=\LARGE] at (4,9.75) {P};
    \node [font=\LARGE] at (5,10) {Q};
    \node [font=\LARGE] at (4,7.5) {R};
    \node [font=\LARGE] at (5,7.5) {S};
    \end{circuitikz}
    }%
    
    
     % Specify the path to your TikZ file
        \caption{2}
        %\label{fig2}
    \end{figure}
    \item[48.] An electromagnetic plane wave is propagating with an intensity $I = 1.0\times 10^5 Wm^{-2}$ in a
    medium with $\epsilon = 3\epsilon_0$ and $\mu = \mu_0$ . The amplitude of the electric field inside the medium
    is ............ $\times0^3Vm^{-1}$ (up to one decimal place). \hfill (2018)
    \item[49.]  A microcanonical ensemble consists of 12 atoms with each taking either energy 0 state,
    or energy $\epsilon$ state. Both states are non-degenerate. If the total energy of this ensemble is
    $4\epsilon$ , its entropy will be ........... $k_B$ (up to one decimal place), where $k_B$ is the
    Boltzmann constant.\hfill (2018)
    \item[50.] A two-state quantum system has energy eigenvalues $\pm \epsilon$ corresponding to the normalized
    states $|\psi_{\pm}\rangle$. At time t = 0 , the system is in quantum state $\frac{1}{\sqrt{2}}[|\psi_{+}\rangle+|\psi_{-}\rangle]$ . The
    probability that the system will be in the same state at $t=\frac{h}{6\epsilon}$ is .........  (up to
    two decimal places). \hfill (2018)
    \item[51.] An air-conditioner maintains the room temperature at $27^{\circ}$C while the outside temperature
    is  $47^{\circ} C$ . The heat conducted through the walls of the room from outside to inside due to
    temperature difference is 7000 W . The minimum work done by the compressor of the
    air-conditioner per unit time is ........... W .\hfill (2018)
    \item[52.] Two solid spheres A and B have same emissivity. The radius of A is four times the
    radius of B and temperature of A is twice the temperature of B . The ratio of the rate of
    heat radiated from A to that from B is ...........\hfill (2018)
\end{enumerate}
\end{document}