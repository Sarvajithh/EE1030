\let\negmedspace\undefined
\let\negthickspace\undefined
\documentclass[journal]{IEEEtran}
\usepackage[a5paper, margin=10mm, onecolumn]{geometry}
\usepackage{lmodern} % Ensure lmodern is loaded for pdflatex
 % Include tfrupee package
\setlength{\headheight}{1cm} % Set the height of the header box
\setlength{\headsep}{0mm}     % Set the distance between the header box and the top of the text
\usepackage{enumitem}
\usepackage{gvv-book}
\usepackage{gvv}
\usepackage{cite}
\usepackage{amsmath,amssymb,amsfonts,amsthm}
\usepackage{algorithmic}
\usepackage{graphicx}
\usepackage{textcomp}
\usepackage{xcolor}
\usepackage{txfonts}
\usepackage{listings}
\usepackage{enumitem}
\usepackage{mathtools}
\usepackage{gensymb}
\usepackage{graphicx}
\usepackage{wrapfig}
\usepackage{comment}
\usepackage[breaklinks=true]{hyperref}
\usepackage{tkz-euclide} 
\usepackage{listings}
% \usepackage{gvv}                                        
\def\inputGnumericTable{}                                 
\usepackage[latin1]{inputenc}                                
\usepackage{color}                                            
\usepackage{array}                                            
\usepackage{longtable}                                       
\usepackage{calc}                                             
\usepackage{multirow}                                         
\usepackage{hhline}                                           
\usepackage{ifthen}                                           
\usepackage{lscape}
\begin{document}

\bibliographystyle{IEEEtran}
\vspace{3cm}


\author{AI24BTECH11008- Sarvajith
}
\title{Assignment 19}
% \maketitle
% \newpage
% \bigskip
{\let\newpage\relax\maketitle}
\title{2018, PH}
\renewcommand{\thefigure}{\theenumi}
\renewcommand{\thetable}{\theenumi}
\setlength{\intextsep}{10pt} % Space between text and floats
\numberwithin{equation}{enumi}
\numberwithin{figure}{enumi}
\renewcommand{\thetable}{\theenumi}
\begin{enumerate}
    \item[30.] A constant and uniform magnetic field $\overrightarrow{B}=B_0\hat{k}$ pervades all space. Which one of the
    following is the correct choice for the vector potential in Coulomb gauge? \hfill (2018)
    \begin{enumerate}[label = (\Alph*)]
        \item $-B_0\brak{x+y}\hat{i}$
        \item $B_0\brak{x+y}\hat{j}$
        \item $-B_0\brak{x\hat{j}}$
        \item $-\frac{1}{2}B_0\brak{x\hat{i}-y\hat{j}}$       
    \end{enumerate}
    \item[31.]If H is the Hamiltonian for a free particle with mass m , the commutator $[x,[x,H]]$ is \hfill (2018)
    \begin{enumerate}[label = (\Alph*)]
        \item $\frac{\hslash ^2}{m}$
        \item $-\frac{\hslash ^2}{m}$
        \item $-\frac{\hslash ^2}{2m}$
        \item $\frac{\hslash ^2}{2m}$
    \end{enumerate}
    \item[32.]  A long straight wire, having radius a and resistance per unit length r , carries a current
    I . The magnitude and direction of the Poynting vector on the surface of the wire is \hfill (2018)
    \begin{enumerate}[label = (\Alph*)]
        \item $\frac{I^2r}{2\pi a}$, perpendicular to axis of the wire and pointing inwards
        \item $\frac{I^2r}{2\pi a}$, perpendicular to axis of the wire and pointing outwards
        \item  $\frac{I^2r}{\pi a}$, perpendicular to axis of the wire and pointing inwards
        \item  $\frac{I^2r}{\pi a}$, perpendicular to axis of the wire and pointing outwards 
    \end{enumerate}
    \item[33.] Three particles are to be distributed in four non-degenerate energy levels. The possible
    number of ways of distribution: (i) for distinguishable particles, and (ii) for identical
    Boson, respectively, is \hfill (2018)
    \begin{enumerate}[label = (\Alph*)]
        \item  (i) 24, (ii) 4 
        \item  (i) 24, (ii) 20 
        \item  (i) 64, (ii) 20 
        \item  (i) 60, (ii) 16
    \end{enumerate}
    \item[34.] The term symbol for the electronic ground state of oxygen atom is    \hfill (2018)
    \begin{enumerate}[label=(\Alph*)]
        \item $^1S_0$
        \item $^1D_2$
        \item $^3P_0$
        \item $^3P_2$
    \end{enumerate}
    \item[35.] The energy dispersion for electrons in one dimensional lattice with lattice parameter a is
    given by $E\brak{k} = E_0-\frac{1}{2}W\cos ka$, where W and $E_0$ are constants. The effective mass of
    the electron near the bottom of the band is \hfill (2018)
    \begin{enumerate}[label=(\Alph*)]
        \item $\frac{2\hslash^2}{Wa^2}$
        \item $\frac{\hslash^2}{Wa^2}$
        \item $\frac{\hslash^2}{2Wa^2}$
        \item $\frac{\hslash^2}{4Wa^2}$
    \end{enumerate}
    \item[36.] Amongst electrical resistivity $\rho$ , thermal conductivity $\kappa$, specific heat $C$ , Young’s
    modulus Y  and magnetic susceptibility $\chi$ , which quantities show a sharp change at
    the superconducting transition temperature?  \hfill (2018)
    \begin{enumerate} [label = (\Alph*)]
     \item $\rho, \kappa, C,Y$ 
     \item $\rho,C,\chi$
     \item $\rho, \kappa, C,\chi$
     \item $\kappa,Y,\chi$ 
    \end{enumerate}
    \item[37.] A quarter wave plate introduces a path difference of $\frac{\lambda}{4}$ between the two components
    of polarization parallel and perpendicular to the optic axis. An electromagnetic wave with $\overrightarrow{E}=\brak{\hat{x}+\hat{y}}E_0e^{i\brak{kz-\omega t}}$is incident normally on a quarter wave plate which has its optic axis
    making an angle $135^{\circ}$ with the x - axis as shown.  \hfill (2018)
    \begin{figure}[!ht]
        \centering
        
    \resizebox{0.4\textwidth}{!}{%
    \begin{circuitikz}
    \tikzstyle{every node}=[font=\normalsize]
    \draw [ fill={rgb,255:red,27; green,24; blue,24} ] (12,15.5) rectangle (19.25,15.25);
    \draw [ fill={rgb,255:red,14; green,201; blue,225} ] (12,17.5) rectangle (17.25,15.5);
    \draw [ fill={rgb,255:red,31; green,30; blue,30} ] (17.25,16.5) circle (1cm);
    \draw [ color={rgb,255:red,252; green,252; blue,252}, <->, >=Stealth] (17.25,15.5) -- (17.25,17.5);
    \node [font=\normalsize, color={rgb,255:red,252; green,252; blue,252}] at (17.5,16.5) {D};
    \end{circuitikz}
    }%
    
  % Specify the path to your TikZ file
        \caption{1}
        %\label{fig2}
    \end{figure}
    The emergent electromagnetic wave would be 
    \begin{enumerate}[label=(\Alph*)]
        \item elliptically polarized 
        \item circularly polarized 
        \item  linearly polarized with polarization as that of incident wave 
        \item  linearly polarized but with polarization at $90^{\circ}$ to that of the incident wave 
    \end{enumerate}
    \item[38.]  A p - doped semiconductor slab carries a current $I=100mA$ in a magnetic field
    $B = 0.2T$ as shown. One measures $V_y$ = 0.25 mV and $V_x$ = 2mV . The mobility of holes
    in the semiconductor is ...........$m^2V^{-1}s^{-1}$ \hfill (2018)
    \begin{figure}[!ht]
        \centering
        
    \resizebox{1\textwidth}{!}{%
    \begin{circuitikz}
    \tikzstyle{every node}=[font=\small]
    \draw [short] (4,9.5) -- (8.75,9.5);
    \draw [short] (2.25,7.75) -- (10.75,7.75);
    \draw [short] (2.25,7.75) -- (4,9.5);
    \draw [short] (8.75,9.5) -- (10.75,7.75);
    \draw [short] (4,9.5) -- (4,7.75);
    \draw [short] (8.75,9.5) -- (8.75,7.75);
    \draw [short] (4,9.5) -- (5,7.75);
    \draw [short] (5,7.75) -- (5,9.5);
    \draw [short] (5,9.5) -- (6.25,7.75);
    \draw [short] (6.25,7.75) -- (7,9.5);
    \draw [short] (7,9.5) -- (7,7.75);
    \draw [short] (7,7.75) -- (8.75,9.5);
    \draw [short] (2.25,7.75) -- (2.5,7.5);
    \draw [short] (2.25,7.75) -- (2,7.5);
    \draw [short] (2,7.5) -- (2.5,7.5);
    \draw [short] (10.75,7.75) -- (11,7.5);
    \draw [short] (10.75,7.75) -- (10.5,7.5);
    \draw [short] (10.5,7.5) -- (11,7.5);
    \node [font=\LARGE] at (4,9.75) {P};
    \node [font=\LARGE] at (5,10) {Q};
    \node [font=\LARGE] at (4,7.5) {R};
    \node [font=\LARGE] at (5,7.5) {S};
    \end{circuitikz}
    }%
    
    
     % Specify the path to your TikZ file
        \caption{2}
        %\label{fig2}
    \end{figure}
    \item[39.]  An n - channel FET having Gate-Source switch-off voltage $V_{GS\brak{OFF} = -2V}$ is used to
    invert a 0-5 V square-wave signal as shown. The maximum allowed value of R would
    be ...........$k\Omega$\hfill (2018)
    \begin{figure}[!ht]
        \centering
        
    \resizebox{0.4\textwidth}{!}{%
    \begin{circuitikz}
    \tikzstyle{every node}=[font=\normalsize]
    \draw [short] (8,10) -- (9,10);
    \draw [short] (8.25,10) -- (8.5,9.5);
    \draw [short] (8.5,9.5) -- (8.75,10);
    \draw [short] (8.25,9.5) -- (18,9.5);
    \draw [short] (17.25,9.5) -- (17,9);
    \draw [short] (17.25,9.5) -- (17.75,9);
    \draw [short] (16.5,9) -- (18,9);
    \draw [short] (8.5,9) -- (8.5,8.25);
    \draw [short] (11.25,9) -- (11.25,8.25);
    \draw [short] (14,9) -- (14,8.25);
    \draw [short] (17.25,9) -- (17.25,8.25);
    \draw [<->, >=Stealth] (8.5,8.5) -- (11.25,8.5)node[pos=0.5, fill=white]{$\frac{L}{3}$};
    \draw [<->, >=Stealth] (11.25,8.75) -- (14,8.75)node[pos=0.5, fill=white]{$\frac{L}{3}$};
    \draw [<->, >=Stealth] (14,8.75) -- (17.25,8.75)node[pos=0.5, fill=white]{$\frac{L}{3}$};
    \draw [->, >=Stealth] (10.75,10) .. controls (11.75,10) and (12,10) .. (11.25,9.25) ;
    \draw [->, >=Stealth] (13.75,10) .. controls (14.75,10) and (14.75,10) .. (14.5,9) ;
    \node [font=\normalsize] at (7.75,9.75) {P};
    \node [font=\normalsize] at (18.25,9.25) {Q};
    \end{circuitikz}
    }%
    
    % Specify the path to your TikZ file
        \caption{3}
        %\label{fig2}
    \end{figure}
    \item[40.] Inside a large nucleus, a nucleon with mass $939 MeVc^{-2}$ has Fermi momentum $1.40 fm^{-1}$
    at absolute zero temperature. Its velocity is Xc , where the value of X is.............. (up
    to two decimal places). \hfill (2018)
    \item[41.] 4MeV $\gamma$ - rays emitted by the de-excitation of $^{19}F$ are attributed, assuming spherical
    symmetry, to the transition of protons from $1d_{3/2}$ state to $1d_{5/2}$ state. If the contribution
    of spin-orbit term to the total energy is written as C$\langle \overrightarrow{l}.\overrightarrow{s}\rangle$
    the magnitude of C is ..........MeV (up to one decimal place). \hfill (2018)
    \item[42.] An  $\alpha$ particle is emitted by a $^{230}_{90}Th$ nucleus. Assuming the potential to be purely
    Coulombic beyond the point of separation, the height of the Coulomb barrier is.............. MeV (up to two decimal places).\hfill (2018)
\end{enumerate}
\end{document}