\let\negmedspace\undefined
\let\negthickspace\undefined
\documentclass[journal]{IEEEtran}
\usepackage[a5paper, margin=10mm, onecolumn]{geometry}
\usepackage{lmodern} % Ensure lmodern is loaded for pdflatex
 % Include tfrupee package
\setlength{\headheight}{1cm} % Set the height of the header box
\setlength{\headsep}{0mm}     % Set the distance between the header box and the top of the text
\usepackage{enumitem}
\usepackage{gvv-book}
\usepackage{gvv}
\usepackage{cite}
\usepackage{amsmath,amssymb,amsfonts,amsthm}
\usepackage{algorithmic}
\usepackage{graphicx}
\usepackage{textcomp}
\usepackage{xcolor}
\usepackage{txfonts}
\usepackage{listings}
\usepackage{enumitem}
\usepackage{mathtools}
\usepackage{gensymb}
\usepackage{graphicx}
\usepackage{wrapfig}
\usepackage{comment}
\usepackage[breaklinks=true]{hyperref}
\usepackage{tkz-euclide} 
\usepackage{listings}
% \usepackage{gvv}                                        
\def\inputGnumericTable{}                                 
\usepackage[latin1]{inputenc}                                
\usepackage{color}                                            
\usepackage{array}                                            
\usepackage{longtable}                                       
\usepackage{calc}                                             
\usepackage{multirow}                                         
\usepackage{hhline}                                           
\usepackage{ifthen}                                           
\usepackage{lscape}
\begin{document}

\bibliographystyle{IEEEtran}
\vspace{3cm}


\author{AI24BTECH11008- Sarvajith
}
\title{Assignment 11}
% \maketitle
% \newpage
% \bigskip
{\let\newpage\relax\maketitle}
\title{2010, AE}
\renewcommand{\thefigure}{\theenumi}
\renewcommand{\thetable}{\theenumi}
\setlength{\intextsep}{10pt} % Space between text and floats
\numberwithin{equation}{enumi}
\numberwithin{figure}{enumi}
\renewcommand{\thetable}{\theenumi}
\begin{enumerate}
    \item[14.] To detect trace amounts of a gaseous species in a mixture of gases, the preferred probing tool is
    \begin{enumerate}[label=(\Alph*)]
        \item Ionization spectroscopy with X-rays
        \item NMR spectroscopy
        \item ESR spectroscopy
        \item Laser spectroscopy
    \end{enumerate}
    \item[15.] A collection of N atoms is exposed to a strong resonant elctromagnetic radiation with $N_g$ atoms in the ground state and $N_e$ atoms in the excited state, such that $N_g+N_e = N$. This collection of two-level atoms will have the following population distribution:
    \begin{enumerate}[label=(\Alph*)]
        \item $N_g<<N_e$
        \item $N_g>>N_e$
        \item $N_g\approx N_e\approx \frac{N}{2}$
        \item $N_g-N_e\approx \frac{N}{2}$
    \end{enumerate}
    \item[16.] Two states of an atom ave definite parities. An electric dipole transition between these states is
    \begin{enumerate}[label=(\Alph*)]
        \item Allowed if both the states have even parity
        \item Allowed if both the states have odd parity
        \item Allowed if both the states have oppsite parity
        \item Not allowed unless a static electric field is applied
    \end{enumerate}
    \item[17.] The spectrum of radiation emitted by a black body at a temperature 1000K peaks in the
    \begin{enumerate}[label=(\Alph*)]
        \item Visible range of frequencies 
        \item Infrared range of frquencies 
        \item Ultraviolet range of frequencies
        \item Microwave range of frequencies
    \end{enumerate}
    \item[18.] An insulating sphere of radius a carries a charge density $\rho \brak{\bar{r}}=\rho_0\brak{a^2-r^2}\cos\theta; r<a$. The leading order term for the electric field at a distance d, far away from the charge distribution is propotional to
    \begin{enumerate}[label=(\Alph*)]
        \item $d^{-1}$
        \item $d^{-2}$
        \item $d^{-3}$
        \item $d^{-4}$
    \end{enumerate}
    \item[19.] The voltage resolution of 12-bit digital to analog converter(DAC), whose output varis from -10V to +10V is approximately
    \begin{enumerate}[label=(\Alph*)]
        \item 1mV
        \item 5mV
        \item 20mV
        \item 100mV
    \end{enumerate}
    \item[20.] In one of the following circuits, negative feedback does not operate for a negative input. Which one is it? The opamps are running from $\pm$15V supplies.
    \begin{figure}[!ht]
        \centering
        
    \resizebox{0.4\textwidth}{!}{%
    \begin{circuitikz}
    \tikzstyle{every node}=[font=\normalsize]
    \draw [ fill={rgb,255:red,27; green,24; blue,24} ] (12,15.5) rectangle (19.25,15.25);
    \draw [ fill={rgb,255:red,14; green,201; blue,225} ] (12,17.5) rectangle (17.25,15.5);
    \draw [ fill={rgb,255:red,31; green,30; blue,30} ] (17.25,16.5) circle (1cm);
    \draw [ color={rgb,255:red,252; green,252; blue,252}, <->, >=Stealth] (17.25,15.5) -- (17.25,17.5);
    \node [font=\normalsize, color={rgb,255:red,252; green,252; blue,252}] at (17.5,16.5) {D};
    \end{circuitikz}
    }%
    
  % Specify the path to your TikZ file
        \caption{option1}
        %\label{fig2}
    \end{figure}
    \begin{figure}[!ht]
        \centering
        
    \resizebox{1\textwidth}{!}{%
    \begin{circuitikz}
    \tikzstyle{every node}=[font=\small]
    \draw [short] (4,9.5) -- (8.75,9.5);
    \draw [short] (2.25,7.75) -- (10.75,7.75);
    \draw [short] (2.25,7.75) -- (4,9.5);
    \draw [short] (8.75,9.5) -- (10.75,7.75);
    \draw [short] (4,9.5) -- (4,7.75);
    \draw [short] (8.75,9.5) -- (8.75,7.75);
    \draw [short] (4,9.5) -- (5,7.75);
    \draw [short] (5,7.75) -- (5,9.5);
    \draw [short] (5,9.5) -- (6.25,7.75);
    \draw [short] (6.25,7.75) -- (7,9.5);
    \draw [short] (7,9.5) -- (7,7.75);
    \draw [short] (7,7.75) -- (8.75,9.5);
    \draw [short] (2.25,7.75) -- (2.5,7.5);
    \draw [short] (2.25,7.75) -- (2,7.5);
    \draw [short] (2,7.5) -- (2.5,7.5);
    \draw [short] (10.75,7.75) -- (11,7.5);
    \draw [short] (10.75,7.75) -- (10.5,7.5);
    \draw [short] (10.5,7.5) -- (11,7.5);
    \node [font=\LARGE] at (4,9.75) {P};
    \node [font=\LARGE] at (5,10) {Q};
    \node [font=\LARGE] at (4,7.5) {R};
    \node [font=\LARGE] at (5,7.5) {S};
    \end{circuitikz}
    }%
    
    
     % Specify the path to your TikZ file
        \caption{option2}
        %\label{fig3}
    \end{figure}
    \begin{figure}[!ht]
        \centering
        
    \resizebox{0.4\textwidth}{!}{%
    \begin{circuitikz}
    \tikzstyle{every node}=[font=\normalsize]
    \draw [short] (8,10) -- (9,10);
    \draw [short] (8.25,10) -- (8.5,9.5);
    \draw [short] (8.5,9.5) -- (8.75,10);
    \draw [short] (8.25,9.5) -- (18,9.5);
    \draw [short] (17.25,9.5) -- (17,9);
    \draw [short] (17.25,9.5) -- (17.75,9);
    \draw [short] (16.5,9) -- (18,9);
    \draw [short] (8.5,9) -- (8.5,8.25);
    \draw [short] (11.25,9) -- (11.25,8.25);
    \draw [short] (14,9) -- (14,8.25);
    \draw [short] (17.25,9) -- (17.25,8.25);
    \draw [<->, >=Stealth] (8.5,8.5) -- (11.25,8.5)node[pos=0.5, fill=white]{$\frac{L}{3}$};
    \draw [<->, >=Stealth] (11.25,8.75) -- (14,8.75)node[pos=0.5, fill=white]{$\frac{L}{3}$};
    \draw [<->, >=Stealth] (14,8.75) -- (17.25,8.75)node[pos=0.5, fill=white]{$\frac{L}{3}$};
    \draw [->, >=Stealth] (10.75,10) .. controls (11.75,10) and (12,10) .. (11.25,9.25) ;
    \draw [->, >=Stealth] (13.75,10) .. controls (14.75,10) and (14.75,10) .. (14.5,9) ;
    \node [font=\normalsize] at (7.75,9.75) {P};
    \node [font=\normalsize] at (18.25,9.25) {Q};
    \end{circuitikz}
    }%
    
    % Specify the path to your TikZ file
        \caption{option3}
       % \label{fig4}
    \end{figure}
    \begin{figure}[!ht]
        \centering
        
    \resizebox{0.3\textwidth}{!}{%
    \begin{circuitikz}
    \tikzstyle{every node}=[font=\normalsize]
    \draw (2.5,9.5) to[R] (4.5,9.5);
    \draw (4.5,9.5) to[R] (7.75,9.5);
    \draw (4.75,9.5) to[R] (4.75,6.25);
    \draw (4.75,9.5) to[R] (6.5,7.75);
    \draw (4.75,6.25) to[R] (6,7.5);
    \draw (7.75,9.5) to[R] (7.75,6.25);
    \draw (7.75,9.5) to[R] (11,9.5);
    \draw (11,9.5) to[R] (11,6.25);
    \draw (4.75,6.25) to[R] (2.5,6.25);
    \draw [short] (6,7.5) .. controls (6.5,7.5) and (6.5,7.5) .. (6.5,8);
    \draw [short] (6.5,8) -- (7.75,9.5);
    \draw [short] (4.75,6.25) -- (6.75,6.25);
    \draw [short] (4.75,6.25) -- (7.75,6.25);
    \draw [short] (6.5,7.75) -- (7.75,6.25);
    \draw [short] (7.75,6.25) -- (11,6.25);
    \draw (2.5,9.5) to[short, -o] (1.75,9.5) ;
    \draw (2.5,6.25) to[short, -o] (1.75,6.25) ;
    \node [font=\normalsize] at (3.5,10) {1$\Omega$};
    \node [font=\normalsize] at (6,9.75) {2$\Omega$};
    \node [font=\normalsize] at (9.25,9.75) {1$\Omega$};
    \node [font=\normalsize] at (10.5,8) {1$\Omega$};
    \node [font=\normalsize] at (8.25,8) {6$\Omega$};
    \node [font=\normalsize] at (6,9) {6$\Omega$};
    \node [font=\normalsize] at (5.75,6.5) {3$\Omega$};
    \node [font=\normalsize] at (4.25,8) {3$\Omega$};
    \node [font=\normalsize] at (3.5,5.75) {0.8$\Omega$};
    \end{circuitikz}
    }%
   % Specify the path to your TikZ file
        \caption{option4}
       % \label{fig5}
    \end{figure}
    \item[21.] A system of N non-interacting classical point particles is constrained to move on the two-dimensional surface of a sphere. The internal energy of system is 
    \begin{enumerate}[label=(\Alph*)]
        \item $\frac{3}{2}Nk_BT$
        \item $\frac{1}{2}Nk_BT$
        \item $Nk_BT$
        \item $\frac{5}{2}Nk_BT$
    \end{enumerate} 
    \item[22.] Which of the following atoms can't exhibit Bose-Einstein condensation, even in principle?
    \begin{enumerate}[label=(\Alph*)]
        \item $^1H_1$
        \item $^4He_2$
        \item $^{23}Na_{11}$
        \item $^{40}K_{19}$
    \end{enumerate}
    \item[23.] For the set of all Lorentz transformations with velocities along the x-axis, consider the two statements given below:\\
     P: If L is a Lorentz transformation then $L^{-1}$ is also called Lorentz transformation
     \\Q: If $L_1 \text{and} L_2$ is a Lorentz transformations then $L_1L_2$ is also called Lorentz transformation.
     \\Choose the correct otpion.
     \begin{enumerate}[label=(\Alph*)]
        \item P is true and Q is false
        \item Both P and Q are true 
        \item Both P and Q are false
        \item Q is true and P is false
     \end{enumerate}
    \item[24.] Which of the follwing is an allowed wavefunction for a particle in a bound state? N is a sonstant and $\alpha,\beta>0$
    \begin{enumerate}[label=(\Alph*)]
        \item $\psi=N\frac{e^{-\alpha r}}{r^3}$
        \item $\psi=N\brak{1-e^{-\alpha r}}$
        \item $\psi=Ne^{-\alpha x}e^{\beta\brak{x^2+y^2+z^2}}$
        \item $\psi=\begin{cases}\text{non-zero constant} & if r<R\\0 & if r>R\end{cases}$
    \end{enumerate}
    \item[25.] A particle is confined within a spherical region of radius one femtometer $\brak{10^{-15}}$. Its momentum can be expected to be about
    \begin{enumerate}[label=(\Alph*)]
        \item $20\frac{keV}{c}$
        \item $200\frac{keV}{c}$
        \item $200\frac{MeV}{c}$
        \item $2\frac{GeV}{c}$
    \end{enumerate}
    \textbf{Q.26-Q.55 carry two marks each.}
    \item[26.] For the complex function, $f\brak{z}=\frac{e^{\sqrt{z}}-e^{-\sqrt{z}}}{\sin \sqrt{z}}$, which of the following statements is correct?
    \begin{enumerate}[label=(\Alph*)]
        \item $z=0$ is a branch point
        \item $z=0$ is a pole of order one
        \item $z=0$ is a removable singularity
        \item $z=0$ is an essential singularity
    \end{enumerate} 
\end{enumerate}
\end{document}