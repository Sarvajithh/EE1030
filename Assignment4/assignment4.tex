\let\negmedspace\undefined
\let\negthickspace\undefined
\documentclass[journal,12pt,twocolumn]{IEEEtran}
\usepackage{cite}
\usepackage{amsmath,amssymb,amsfonts,amsthm}
\usepackage{algorithmic}
\usepackage{graphicx}
\usepackage{textcomp}
\usepackage{xcolor}
\usepackage{txfonts}
\usepackage{listings}
\usepackage{enumitem}
\usepackage{mathtools}
\usepackage{gensymb}
\usepackage{comment}
\usepackage[breaklinks=true]{hyperref}
\usepackage{tkz-euclide}
\usepackage{listings}
\usepackage{gvv}  
\usepackage{tikz}
\usepackage{circuitikz}
\usepackage{caption}
\def\inputGnumericTable{}              
\usepackage[latin1]{inputenc}          
\usepackage{color}                    
\usepackage{array}                    
\usepackage{longtable}                
\usepackage{calc}
\usepackage{multirow}                
\usepackage{multicol}
\usepackage{hhline}                    
\usepackage{ifthen}                    
\usepackage{lscape}
\usepackage{amsmath}
\newtheorem{theorem}{Theorem}[section]
\newtheorem{problem}{Problem}
\newtheorem{proposition}{Proposition}[section]
\newtheorem{lemma}{Lemma}[section]
\newtheorem{corollary}[theorem]{Corollary}
\newtheorem{example}{Example}[section]
\newtheorem{definition}[problem]{Definition}

\newcommand{\BEQA}{\begin{eqnarray}}
\newcommand{\EEQA}{\end{eqnarray}}
\newcommand{\define}{\stackrel{\triangle}{=}}
\theoremstyle{remark}
\newtheorem{rem}{Remark}
\title{Assignment-4

% }
}
\author{Sarvajith-AI24BTECH11008}

\begin{document}

\maketitle


16: If P and Q are two statements, then which of the following compound statements is a tautology?

\begin{enumerate}
    \item [a.] $\brak{\brak{P\rightarrow Q}\land \neg Q}\rightarrow P$
    \item [b.] $\brak{\brak{P\rightarrow Q}\land \neg Q}\rightarrow \neg P$
    \item [c.] $\brak{\brak{P\rightarrow Q}\land \neg Q}$
    \item [d.] $\brak{\brak{P\rightarrow Q}\land \neg Q}\rightarrow \neg Q$
\end{enumerate}
17:  Consider a hyperbola $H : x^2 - 2y^2 = 4$. Let the tangent at a point $P \brak{4, \sqrt{6}}$ meet the x-axis at $Q$ and the latus rectum at $ R \brak{x_1, y_1}$, where  $x_1 > 0$. If $ F $ is a focus of $H$  which is nearer to the point  $P$, then the area of $\Delta QFR$ is equal to:
\begin{multicols}{2}
\begin{enumerate}
    \item [a.] $\sqrt{6}-1$
    \item [b.] $4\sqrt{6}-1$
	    \columnbreak
    \item [c.] $4\sqrt{6}$
    \item [d.] $\frac{7}{\sqrt{6}}-2$
\end{enumerate}
\end{multicols}
18: Let $f:R\rightarrow R$ be a function defined as 
$$f\brak{x} = \begin{cases}
    \frac{\sin\brak{a+1}x + \sin 2x}{2x}, & x<0\\
    b, & x=0\\
    \frac{\sqrt{x+bx^3-\sqrt{x}}}{bx^{5/2}}, & x>0
\end{cases}$$
If f is continuous at x = 0, then the value of a+b is equal to
\begin{multicols}{2}
\begin{enumerate}
    \item [a.] -2
    \item [b.] $\frac{-2}{5}$
	    \columnbreak
    \item [c.] $\frac{-3}{2}$
    \item [d.] -3
\end{enumerate}
\end{multicols}
19: Let $y=y\brak{x}$ be the solution of the differential equation $\frac{dy}{dx} =\brak{y+1}\brak{\brak{y+1}e^{x^2/2}-x}$, $0<x<2.1$, with y\brak{2} = 0. Then the value of $\frac{dy}{dx}$ at x=1 is equal to
\begin{multicols}{2}
\begin{enumerate}
    \item [a.] $\frac{e^{5/2}}{\brak{1+e^2}^2}$
    \item [b.] $\frac{5e^{1/2}}{\brak{e^2}+1}$
	    \columnbreak
    \item [c.] $\frac{-2e^2}{\brak{1+e^2}^2}$
    \item [d.] $\frac{-e^{3/2}}{\brak{e^2+1}^2}$
\end{enumerate}
\end{multicols}
20: Let a tangent be drawn to the ellipse $ \frac{x^2}{27} + y^2 = 1 $ at $ \brak{3\sqrt{3} \cos \theta, \sin \theta}$ where $\theta \in \brak{0, \frac{\pi}{2}}$ . Then the value of $\theta$  such that the sum of intercepts on axes made by a tangent is minimum is equal to:
\begin{enumerate}
    \item [a.] $\frac{\pi}{8}$
    \item [b.]  $\frac{\pi}{6}$
    \item [c.]  $\frac{\pi}{3}$
    \item [d.]  $\frac{\pi}{4}$
\end{enumerate}
\section{Section-B}
1: Let $ P $ be a plane containing the line 
$\frac{x - 1}{3} = \frac{y + 6}{4} = \frac{z + 5}{2}$
and parallel to the line $\frac{x - 3}{4} = \frac{y - 2}{-3} = \frac{z + 5}{7}.$ If the point $\brak{1, -1, \alpha}$ lies on the plane $P$, then the value of $\brak{\abs{5\alpha}}$ is equal to?\vspace{1mm}\\
2:$ \sum_{r=1}^{10} r! \brak{r^3 + 6r^2 + 2r + 5} = \alpha\brak{11!}$
Then the value of $\alpha$ is equal to:\vspace{1mm}\\
3:The term independent of $x$ in the expansion of$\brak{\frac{x + 1}{x^{2/3} - x^{1/3} + 1} - \frac{x - 1}{x - x^{1/2}}}^{10}, \quad x \neq 1$ is equal to?\vspace{1mm} \\ 
4:  Let $^nC_r$ denote the binomial coefficient of $x^r$ in the expansion of $\brak{1 + x}^n$. If $\sum_{k=0}^{10} \brak{22 + 3k}   ^nC_k = \alpha \cdot 3^{10} + \beta \cdot 2^{10},$then $\alpha + \beta$ is equal to?\vspace{1mm}\\
5:Let $P\brak{x}$ be a real polynomial of degree 3 which vanishes at $x = -3$. Let $P\brak{x}$ have local minima at $x = 1$, local maxima at $x = -1$, and $\int_{-1}^{1} P(x) \, dx = 18,$then the sum of all the coefficients of the polynomial $P\brak{x}$ is equal to?\vspace{1mm}\\
6: Let the mirror image of the point \brak{1, 3, a} with respect to the plane $\mathbf{r} \cdot \brak{2\mathbf{i} - \mathbf{j} + \mathbf{k}} - b = 0$ be \brak{-3, 5, 2}, Then, the value of $\abs{a + b}$ is equal to?\vspace{1mm}\\
7: If $f\brak{x}$ and $g{x}$ are two polynomials such that the polynomial $P\brak{x} = f\brak{x^3} + x g\brak{x^3}$ is divisible by $x^2 + x + 1$, then $P\brak{1}$ is equal to?\vspace{1mm}\\
8: Let I be an identity matrix of order $2 \times 2$ and $$P = \myvec{2&-1\\5&-3}$$ $P_n = 5I - 8P$ Then the value of $n \in \mathbb{N}$ for which $P_n = 5I - 8P$ is equal to?\vspace{1mm}\\
9: Let $f : \mathbb{R} \to \mathbb{R}$ satisfy the equation$f\brak{x + y} = f\brak{x} \cdot f\brak{y}$ for all $x, y \in \mathbb{R}$ and $f\brak{x} \neq 0$ for any $x \in \mathbb{R}$. If the function $f$ is differentiable at $x = 0$ and $f'\brak{0} = 3$, then $\lim_{h \to 0} \frac{1}{h} [f(h) - 1] $
is equal to?\vspace{1mm}\\
10: Let $y = y\brak{x}$ be the solution of the differential equation 
$x dy - y dx = \sqrt{x^2 - y^2} \, dx, \quad x \geq 1,$ with $y\brak{1} = 0$. If the area bounded by the line $x = 1$, $x = e^\pi$, $y = 0$ and $y = y\brak{x}$ is $\alpha e^{2\pi} + \beta, $then the value of $10 \brak{\alpha + \beta}$ is equal to?
\end{document}

