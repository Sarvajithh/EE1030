\let\negmedspace\undefined
\let\negthickspace\undefined
\documentclass[journal]{IEEEtran}
\usepackage[a5paper, margin=10mm, onecolumn]{geometry}
\usepackage{lmodern} % Ensure lmodern is loaded for pdflatex
 % Include tfrupee package
\setlength{\headheight}{1cm} % Set the height of the header box
\setlength{\headsep}{0mm}     % Set the distance between the header box and the top of the text
\usepackage{enumitem}
\usepackage{gvv-book}
\usepackage{gvv}
\usepackage{cite}
\usepackage{amsmath,amssymb,amsfonts,amsthm}
\usepackage{algorithmic}
\usepackage{graphicx}
\usepackage{textcomp}
\usepackage{xcolor}
\usepackage{txfonts}
\usepackage{listings}
\usepackage{enumitem}
\usepackage{mathtools}
\usepackage{gensymb}
\usepackage{comment}
\usepackage[breaklinks=true]{hyperref}
\usepackage{tkz-euclide} 
\usepackage{listings}
% \usepackage{gvv}                                        
\def\inputGnumericTable{}                                 
\usepackage[latin1]{inputenc}                                
\usepackage{color}                                            
\usepackage{array}                                            
\usepackage{longtable}                                       
\usepackage{calc}                                             
\usepackage{multirow}                                         
\usepackage{hhline}                                           
\usepackage{ifthen}                                           
\usepackage{lscape}
\begin{document}

\bibliographystyle{IEEEtran}
\vspace{3cm}


\author{AI24BTECH11008- Sarvajith
}
\title{Assignment 6}
% \maketitle
% \newpage
% \bigskip
{\let\newpage\relax\maketitle}
\title{Feb 2021, shift 2}
\renewcommand{\thefigure}{\theenumi}
\renewcommand{\thetable}{\theenumi}
\setlength{\intextsep}{10pt} % Space between text and floats
\numberwithin{equation}{enumi}
\numberwithin{figure}{enumi}
\renewcommand{\thetable}{\theenumi}
\begin{enumerate}
\item [1]: If $\phi \brak{x} = \frac{1}{\sqrt{x}}\int _{\frac{\pi}{4}} ^x \brak{4\sqrt{2}\sin t - 3\phi '\brak{t}}dt, x>0$, then $\phi '\brak{\frac{\pi}{4}}$ is equal to
\begin{enumerate}
    \item [a.] $\frac{8}{\sqrt{\pi}}$
    \item [b.]  $\frac{4}{6+\sqrt{\pi}}$
    \item [c.]  $\frac{8}{6+\sqrt{\pi}}$
    \item [d.]  $\frac{4}{6-\sqrt{\pi}}$
\end{enumerate}
\item[2]: If a point $P\brak{\alpha, \beta, \gamma}$ satisfying $\begin{bmatrix}\alpha & \beta & \gamma \end{bmatrix}$ $\begin{bmatrix}
    2 & 10 & 8\\ 9&3&8\\8&4&8
\end{bmatrix} $ = $\begin{bmatrix}
    0&0&0
\end{bmatrix}$ lies on the plane $2x+4y+3z=5$ then $6\alpha + 9\beta + 7\gamma$ is equal to
\begin{enumerate}
    \item [a.] -1
    \item [b.] $\frac{-11}{5}$
    \item [c.] $\frac{5}{4}$
    \item [d.] 11
\end{enumerate}
\item[3]. Let $a_1, a_2, a_3, \ldots$ be an A.P. If $a_7=3$ the product $a_1a_4$ is minimum and the sum of its first n terms is zero, then n!-4$a_{n\brak{n+2}}$ is equal to  
\begin{enumerate}
    \item [a.] 24
    \item [b.] $\frac{33}{4}$
    \item [c.] $\frac{381}{4}$
    \item [d.] 9
\end{enumerate}
\item[4]: Let $\brak{a, b} \subset \brak{ 0, 2\pi}$ be the largest interval for which$\sin ^{-1}\brak{\sin \theta} - \cos^{-1}\brak{\sin\theta} > 0,  \theta \in \brak{ 0, 2\pi}$holds. If$\alpha x^2 + \beta x+ \sin^{-1}\brak{x^2 - 6x + 10}+\cos ^{-1}{x^2 - 6x + 10} = 0$ and $\alpha - \beta = b - a,$ then $\brak{\alpha}$ is equal to:

 \begin{enumerate}
    \item [a.] $\frac{\pi}{48}$
    \item [b.] $\frac{\pi}{16}$
    \item [c.] $\frac{\pi}{8}$
    \item [d.] $\frac{\pi}{12}$
\end{enumerate}
\item[5]: Let $y=y\brak{x}$ be the solution of the differential equation $\brak{3y^2-5x^2}ydx+2x\brak{x^2-y^2}dy=0$ such that y\brak{1} = 1 then $\abs{\brak{y\brak{2}}^3-12y\brak{2}}$ is equal to 
\begin{enumerate}
    \item [a.] $32\sqrt{2}$
    \item [b.] 64
    \item [c.] $16\sqrt{2}$
    \item [d.] 32
\end{enumerate}
\item[6]: The set of all values of $a^2$ for which the line $x+y=0$ bisects two distinct chords drawn from a point $P\brak{\frac{1+a}{2}, \frac{1-a}{2}}$ on the circle $2x^2 + 2y^2-\brak{1+a}x-\brak{1-a}y=0$
\begin{enumerate}
\item [a.] $\brak{8,\infty}$
    \item [b.] $\brak{4,\infty}$
    \item [c.] $(0,4]$
    \item [d.] $(2,12]$
\end{enumerate}
\item[7]: If $S = \{\brak{a,b}:a,b \in \mathbf{R}-\{0\},2\frac{a}{b}>0\}$ And T = $\{\brak{a,b}:a,b \in \mathbf{R}, a^2-b^2 \in \mathbf{Z}\}$:
\begin{enumerate}
 \item [a.] S is transitive but T is not 
    \item [b.] T is symmetric but S is not 
    \item [c.] Neither S nor T is transitive 
    \item [d.] Both S and T are symmetric 
\end{enumerate}
\item[8]: The equation $e^{4x} + 8e^{3x} + 13e^{2x} - 8e^x + 1 = 0, x \in R$ has:
\begin{enumerate}
    \item [a.] two solutions and both are negative 
    \item [b.] no solution 
    \item [c.] four solutions two of which are negative 
    \item [d.] two solutions and only one of them is negative
\end{enumerate}
\item[9]: The number of values of $r \in \{p,q,\neg p, \neg q\}$ for which $\brak{\brak{p\land q}\rightarrow\brak{r\lor q}} \land \brak{\brak{p\land r}\rightarrow q}$ is a tautology is: 
\begin{enumerate}
     \item [a.] 3
    \item [b.] 2
    \item [c.] 1
    \item [d.] 4
\end{enumerate}
\item[10]: Let $f:\mathbf{R}-\{2,6\}\rightarrow \mathbf{R}$ be a real-valued function defined as $ f\brak{x}=\frac{x^2+2x+1}{x^2-8x+12}$ Then the range of f is  
\begin{enumerate}
     \item [a.] $\big(-\infty,-\frac{21}{4}\big]\cup[0,\infty)$
    \item [b.] $\big(-\infty,-\frac{21}{4}\big]\cup(0,\infty)$
    \item [c.] $\big(-\infty,-\frac{21}{4}\big]\cup\big[\frac{21}{4},\infty\big)$
    \item [d.] $\big(-\infty,-\frac{21}{4}\big]\cup[1,\infty)$
\end{enumerate}
\item[11]:$\lim_{x\rightarrow\infty} \frac{\brak{\sqrt{3x+1}+\sqrt{3x-1}}^6+\brak{\sqrt{3x+1}-\sqrt{3x-1}}^6}{\brak{x+\sqrt{x^2-1}}^6+\brak{x-\sqrt{x^2-1}}^6}$
\begin{enumerate}
    \item [a.] is equal to 9
    \item [b.] is equal to 27
    \item [c.] does not exist
    \item [d.] is equal to $\frac{27}{2}$
\end{enumerate}

\item[12]: Let P be the plane, passing through the point \brak{1,-1,-5} and perpendicular to the line joining the points \brak{4,1,-3} and \brak{2,4,3}. Then the distance of P from the point \brak{3,-2,2} is
\begin{enumerate}
     \item [a.] 6
    \item [b.] 4
    \item [c.] 5
    \item [d.] 7
\end{enumerate}
\item[13]: The absolute minimum values of the function $f\brak{x}=\abs{x^2-x+1}+[x^2-x+1]$, where [t] denotes the greatest integer function, in the interval [-1,2] is
\begin{enumerate}
     \item [a.] $\frac{3}{4}$
    \item [b.] $\frac{3}{2}$
    \item [c.] $\frac{1}{2}$
    \item [d.] $\frac{5}{4}$
\end{enumerate}
\item[14]: Let the plane $P:8x+\alpha _1y+\alpha _2z+12=0$ be parallel to the line $L:\frac{x+2}{2}=\frac{y-3}{3}=\frac{z+4}{5}$. If the intercept of P on the y-axis is 1, then the distance between P and L is 
\begin{enumerate}
\item [a.] $\sqrt{14}$
    \item [b.]$\frac{6}{\sqrt{14}}$
    \item [c.] $\sqrt{\frac{2}{7}}$
    \item [d.] $\sqrt{\frac{7}{2}}$
\end{enumerate}
\item[15]: The foot of the perpendicular from the origin O to a plane P which meets to coordinate axes at the points A, B, C
is \brak{2,a,4},  $a\in N$.if the volume of the tetrahedron OABC is $144 unit^3$, then which of the following points is NOT on P?
\begin{enumerate}
\item [a.] \brak{2,2,4}
    \item [b.] \brak{0,4,4}
    \item [c.] \brak{3,0,4}
    \item [d.] \brak{0,6,3}
\end{enumerate}
\end{enumerate}
\end{document}

