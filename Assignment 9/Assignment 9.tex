\let\negmedspace\undefined
\let\negthickspace\undefined
\documentclass[journal]{IEEEtran}
\usepackage[a5paper, margin=10mm, onecolumn]{geometry}
\usepackage{lmodern} % Ensure lmodern is loaded for pdflatex
 % Include tfrupee package
\setlength{\headheight}{1cm} % Set the height of the header box
\setlength{\headsep}{0mm}     % Set the distance between the header box and the top of the text
\usepackage{enumitem}
\usepackage{gvv-book}
\usepackage{gvv}
\usepackage{cite}
\usepackage{amsmath,amssymb,amsfonts,amsthm}
\usepackage{algorithmic}
\usepackage{graphicx}
\usepackage{textcomp}
\usepackage{xcolor}
\usepackage{wrapfig}
\usepackage{txfonts}
\usepackage{listings}
\usepackage{enumitem}
\usepackage{mathtools}
\usepackage{gensymb}
\usepackage{graphicx}
\usepackage{wrapfig}
\usepackage{comment}
\usepackage[breaklinks=true]{hyperref}
\usepackage{tkz-euclide} 
\usepackage{listings}
\usepackage{gvv}                                        
\def\inputGnumericTable{}                                 
\usepackage[latin1]{inputenc}                                
\usepackage{color}                                            
\usepackage{array}                                            
\usepackage{longtable}                                       
\usepackage{calc}   
\usepackage{multicol}                                          
\usepackage{multirow}                                         
\usepackage{hhline}                                           
\usepackage{ifthen}                                           
\usepackage{lscape}
\begin{document}

\bibliographystyle{IEEEtran}
\vspace{3cm}


\author{AI24BTECH11008- Sarvajith
}
\title{Assignment 9}
 %\maketitle
 %\newpage
 %\bigskip
{\let\newpage\relax\maketitle}
\title{2009, PH}
\renewcommand{\thefigure}{\theenumi}
\renewcommand{\thetable}{\theenumi}
\setlength{\intextsep}{10pt} % Space between text and floats
\numberwithin{equation}{enumi}
\numberwithin{figure}{enumi}
\renewcommand{\thetable}{\theenumi}
\begin{enumerate}
    \item[49.] The Lagrangian of a particle of mass m moving in one dimension is $ L = exp\brak{\alpha t}\big[\frac{m\dot{x}^2}{2}-\frac{kx^2}{2}],$ where $\alpha$ and k are positive constants. The equation of motion of the particle is 
     \begin{enumerate}[label=(\Alph*)]
        \item $\ddot{x}+\alpha \dot{x} = 0$
        \item $\ddot{x}+\frac{k}{m}x = 0$
        \item $\ddot{x}-\alpha \dot{x}+\frac{k}{m}x = 0$
        \item $\ddot{x}+\alpha \dot{x}+\frac{k}{m}x = 0$
     \end{enumerate}
     \item[50.] Two monochromatic waves having frequencies $\omega$ and $\omega + \triangle \omega \brak{\Delta\omega<<<\omega}$ and corresonding wavelength $\lambda$ and $\lambda-\Delta\lambda\brak{\Delta\lambda<<<\lambda}$ of same polarization, travelling along x-axis are superimposed on each other. The phase velocity and group velocity of the resultant wave are respectively given by
     \begin{enumerate}[label=(\Alph*)]
         \item $\frac{\omega\lambda}{2\pi},\frac{\Delta\lambda ^2}{2\pi\Delta\lambda}$
         \item $\omega\lambda,\frac{\Delta\lambda ^2}{\Delta\lambda}$
         \item $\frac{\omega\Delta\lambda}{2\pi},\frac{\Delta\lambda}{2\pi\Delta\lambda}$
         \item $\omega\Delta\lambda,\omega\Delta\lambda$
     \end{enumerate}
     \textbf{Common data questions}
     Common data questions 51 and 52
     Consider a two level quantum system with energies $\epsilon_1=0 \text{and} \epsilon_2 = \epsilon$
     \item[51.] The Helmholtz free energy of the system is given by
     \begin{enumerate}[label=(\Alph*)]
        \item $-k_BTln\brak{1+e^{\frac{-\epsilon}{k_BT}}}$
        \item $k_BTln\brak{1+e^{\frac{-\epsilon}{k_BT}}}$
        \item $\frac{3}{2}k_B T$
        \item $\epsilon - k_BT$
     \end{enumerate}
     \item [52.]  The specific heat of the system is given by
     \begin{enumerate}[label=(\Alph*)]
        \item $\frac{\epsilon}{k_BT}\frac{e^{\frac{-\epsilon}{k_BT}}}{\brak{1+e^{\frac{-\epsilon}{k_BT}}}^2}$
        \item $\frac{\epsilon^2}{k_BT^2}\frac{e^{\frac{-\epsilon}{k_BT}}}{\brak{1+e^{\frac{-\epsilon}{k_BT}}}}$
        \item $-\frac{\epsilon^2e^{\frac{-\epsilon}{k_BT}}}{\brak{1+e^{\frac{-\epsilon}{k_BT}}}^2}$
        \item  $\frac{\epsilon^2}{k_BT^2}\frac{e^{\frac{-\epsilon}{k_BT}}}{\brak{1+e^{\frac{-\epsilon}{k_BT}}}^2}$
     \end{enumerate}
     Common data questions 53 and 54
     A free particle of mass m moves along the x-direction. At t = 0, the normalized wave function of the particle is given by $\psi\brak{x,0}=\frac{1}{\brak{2\pi\alpha}^{1/4}}exp{-\frac{x^2}{4\alpha^2}+ix}$, where $\alpha$ is a real constant 
     \item[53.] The expectation value of the momentum, in this state is 
     \begin{enumerate}[label=(\Alph*)]
        \item $\hslash \alpha$
        \item $\hslash \sqrt{\alpha}$
        \item $\alpha$
        \item $\frac{\hslash}{\sqrt{\alpha}}$
     \end{enumerate}
     \item[54.] The expectation value of the particle energy is 
     \begin{enumerate}[label=(\Alph*)]
        \item $\frac{\hslash^2}{2m}\frac{1}{2\alpha^{3/2}}$
        \item $\frac{\hslash^2}{2m}\alpha^2$
        \item $\frac{\hslash^2}{2m}\frac{4\alpha^2+1}{4\alpha^{3/2}}$
         \item $\frac{\hslash^2}{8m\alpha^{3/2} }$
     \end{enumerate}
     Common data questions 55 and 56
     Consider the Zeeman splitting of a single electron system for the 3d to 3p electric dipole transition
     \item[55.] The Zeeman spectrum is
     \begin{enumerate}[label=(\Alph*)]
        
        \item Randomly polarized
        \item only $\pi$ polarized
        \item only $\sigma$ polarized
        \item both $\pi$ and $\sigma$ polarized
     \end{enumerate}
     \item[56.]  The fine structure line having the longest wavelength will split into
     \begin{enumerate}[label=(\Alph*)]
        \item  17 components
        \item  10 components
        \item  8 components
        \item  4 components
     \end{enumerate}
     \textbf{Linked Answer Questions}
     Statement for Linked Answer Questions 57 ans 58:
    The primitive translation vectors of the face centered cubic (fcc) lattice are 
    $$\hat{a}_1 = \frac{a}{2}\brak{\hat{j}+\hat{k}};\hat{a}_2 = \frac{a}{2}\brak{\hat{i}+\hat{k}};\hat{a}_1 = \frac{a}{2}\brak{\hat{j}+\hat{i}}$$
    \item[57.] The primitive transition vectors of the fccreciprocal lattice are 
    \begin{enumerate}[label=(\Alph*)]
        \item $\hat{b}_1 = \frac{2\pi}{a}\brak{\hat{j}+\hat{k}-\hat{i}};\hat{b}_2 = \frac{2\pi}{a}\brak{-\hat{j}+\hat{k}+\hat{i}};\hat{b}_3 = \frac{2\pi}{a}\brak{\hat{j}-\hat{k}+\hat{i}}$
        \item $\hat{b}_1 = \frac{\pi}{a}\brak{\hat{j}+\hat{k}-\hat{i}};\hat{b}_2 = \frac{\pi}{a}\brak{-\hat{j}+\hat{k}+\hat{i}};\hat{b}_3 = \frac{\pi}{a}\brak{\hat{j}-\hat{k}+\hat{i}}$
        \item $\hat{b}_1 = \frac{\pi}{2a}\brak{\hat{j}+\hat{k}-\hat{i}};\hat{b}_2 = \frac{\pi}{2a}\brak{-\hat{j}+\hat{k}+\hat{i}};\hat{b}_3 = \frac{\pi}{2a}\brak{\hat{j}-\hat{k}+\hat{i}}$
        \item $\hat{b}_1 = \frac{3\pi}{a}\brak{\hat{j}+\hat{k}-\hat{i}};\hat{b}_2 = \frac{3\pi}{a}\brak{-\hat{j}+\hat{k}+\hat{i}};\hat{b}_3 = \frac{3\pi}{a}\brak{\hat{j}-\hat{k}+\hat{i}}$
    \end{enumerate}
    \item[58.] The volume of the primitive cell of the fcc reciprocal lattice is
    \begin{enumerate}[label=(\Alph*)]
        \item $4\brak{\frac{\pi}{a}}^3$
        \item $4\brak{\frac{2\pi}{a}}^3$
        \item $4\brak{\frac{\pi}{2a}}^3$
        \item $4\brak{\frac{3\pi}{a}}^3$
    \end{enumerate}
    Statement for Linked Answer Questions 59 and 60:
    The Karnaugh map of logic circuit shown is below 
    \begin{figure}[!ht]
        \centering
        
    \resizebox{0.4\textwidth}{!}{%
    \begin{circuitikz}
    \tikzstyle{every node}=[font=\normalsize]
    \draw [ fill={rgb,255:red,27; green,24; blue,24} ] (12,15.5) rectangle (19.25,15.25);
    \draw [ fill={rgb,255:red,14; green,201; blue,225} ] (12,17.5) rectangle (17.25,15.5);
    \draw [ fill={rgb,255:red,31; green,30; blue,30} ] (17.25,16.5) circle (1cm);
    \draw [ color={rgb,255:red,252; green,252; blue,252}, <->, >=Stealth] (17.25,15.5) -- (17.25,17.5);
    \node [font=\normalsize, color={rgb,255:red,252; green,252; blue,252}] at (17.5,16.5) {D};
    \end{circuitikz}
    }%
    
  % Specify the path to your TikZ file
        \caption{1}
        \label{fig1}
    \end{figure}
    %image
    \item[59.] The minimized logic expression for the above map is 
    \begin{enumerate}[label=(\Alph*)]
        \item $Y=\bar{PR} + \bar{Q}$
        \item $Y=\bar{Q} .PR$
        \item $Y=PR + \bar{Q}$
        \item $Y=\bar{PR}.Q$
    \end{enumerate}
    \item[60.] The corresponding logic implementation using gates is given as:
     \end{enumerate}
    \begin{figure}[!ht]
        \centering
        
    \resizebox{1\textwidth}{!}{%
    \begin{circuitikz}
    \tikzstyle{every node}=[font=\small]
    \draw [short] (4,9.5) -- (8.75,9.5);
    \draw [short] (2.25,7.75) -- (10.75,7.75);
    \draw [short] (2.25,7.75) -- (4,9.5);
    \draw [short] (8.75,9.5) -- (10.75,7.75);
    \draw [short] (4,9.5) -- (4,7.75);
    \draw [short] (8.75,9.5) -- (8.75,7.75);
    \draw [short] (4,9.5) -- (5,7.75);
    \draw [short] (5,7.75) -- (5,9.5);
    \draw [short] (5,9.5) -- (6.25,7.75);
    \draw [short] (6.25,7.75) -- (7,9.5);
    \draw [short] (7,9.5) -- (7,7.75);
    \draw [short] (7,7.75) -- (8.75,9.5);
    \draw [short] (2.25,7.75) -- (2.5,7.5);
    \draw [short] (2.25,7.75) -- (2,7.5);
    \draw [short] (2,7.5) -- (2.5,7.5);
    \draw [short] (10.75,7.75) -- (11,7.5);
    \draw [short] (10.75,7.75) -- (10.5,7.5);
    \draw [short] (10.5,7.5) -- (11,7.5);
    \node [font=\LARGE] at (4,9.75) {P};
    \node [font=\LARGE] at (5,10) {Q};
    \node [font=\LARGE] at (4,7.5) {R};
    \node [font=\LARGE] at (5,7.5) {S};
    \end{circuitikz}
    }%
    
    
     % Specify the path to your TikZ file
        \caption{option1}
        %\label{fig2}
    \end{figure}
    \begin{figure}[!ht]
        \centering
        
    \resizebox{0.4\textwidth}{!}{%
    \begin{circuitikz}
    \tikzstyle{every node}=[font=\normalsize]
    \draw [short] (8,10) -- (9,10);
    \draw [short] (8.25,10) -- (8.5,9.5);
    \draw [short] (8.5,9.5) -- (8.75,10);
    \draw [short] (8.25,9.5) -- (18,9.5);
    \draw [short] (17.25,9.5) -- (17,9);
    \draw [short] (17.25,9.5) -- (17.75,9);
    \draw [short] (16.5,9) -- (18,9);
    \draw [short] (8.5,9) -- (8.5,8.25);
    \draw [short] (11.25,9) -- (11.25,8.25);
    \draw [short] (14,9) -- (14,8.25);
    \draw [short] (17.25,9) -- (17.25,8.25);
    \draw [<->, >=Stealth] (8.5,8.5) -- (11.25,8.5)node[pos=0.5, fill=white]{$\frac{L}{3}$};
    \draw [<->, >=Stealth] (11.25,8.75) -- (14,8.75)node[pos=0.5, fill=white]{$\frac{L}{3}$};
    \draw [<->, >=Stealth] (14,8.75) -- (17.25,8.75)node[pos=0.5, fill=white]{$\frac{L}{3}$};
    \draw [->, >=Stealth] (10.75,10) .. controls (11.75,10) and (12,10) .. (11.25,9.25) ;
    \draw [->, >=Stealth] (13.75,10) .. controls (14.75,10) and (14.75,10) .. (14.5,9) ;
    \node [font=\normalsize] at (7.75,9.75) {P};
    \node [font=\normalsize] at (18.25,9.25) {Q};
    \end{circuitikz}
    }%
    
    % Specify the path to your TikZ file
        \caption{option2}
        %\label{fig3}
    \end{figure}
    \begin{figure}[!ht]
        \centering
        
    \resizebox{0.3\textwidth}{!}{%
    \begin{circuitikz}
    \tikzstyle{every node}=[font=\normalsize]
    \draw (2.5,9.5) to[R] (4.5,9.5);
    \draw (4.5,9.5) to[R] (7.75,9.5);
    \draw (4.75,9.5) to[R] (4.75,6.25);
    \draw (4.75,9.5) to[R] (6.5,7.75);
    \draw (4.75,6.25) to[R] (6,7.5);
    \draw (7.75,9.5) to[R] (7.75,6.25);
    \draw (7.75,9.5) to[R] (11,9.5);
    \draw (11,9.5) to[R] (11,6.25);
    \draw (4.75,6.25) to[R] (2.5,6.25);
    \draw [short] (6,7.5) .. controls (6.5,7.5) and (6.5,7.5) .. (6.5,8);
    \draw [short] (6.5,8) -- (7.75,9.5);
    \draw [short] (4.75,6.25) -- (6.75,6.25);
    \draw [short] (4.75,6.25) -- (7.75,6.25);
    \draw [short] (6.5,7.75) -- (7.75,6.25);
    \draw [short] (7.75,6.25) -- (11,6.25);
    \draw (2.5,9.5) to[short, -o] (1.75,9.5) ;
    \draw (2.5,6.25) to[short, -o] (1.75,6.25) ;
    \node [font=\normalsize] at (3.5,10) {1$\Omega$};
    \node [font=\normalsize] at (6,9.75) {2$\Omega$};
    \node [font=\normalsize] at (9.25,9.75) {1$\Omega$};
    \node [font=\normalsize] at (10.5,8) {1$\Omega$};
    \node [font=\normalsize] at (8.25,8) {6$\Omega$};
    \node [font=\normalsize] at (6,9) {6$\Omega$};
    \node [font=\normalsize] at (5.75,6.5) {3$\Omega$};
    \node [font=\normalsize] at (4.25,8) {3$\Omega$};
    \node [font=\normalsize] at (3.5,5.75) {0.8$\Omega$};
    \end{circuitikz}
    }%
   % Specify the path to your TikZ file
        \caption{option3}
       % \label{fig4}
    \end{figure}
    \begin{figure}[!ht]
        \centering
        
    \resizebox{0.4\textwidth}{!}{%
    \begin{circuitikz}
    \tikzstyle{every node}=[font=\normalsize]
    \draw  (12.75,11) rectangle (18.75,10.75);
    \draw [<->, >=Stealth] (12.75,11.25) -- (18.75,11.25)node[pos=0.5, fill=white]{L};
    \draw  (23.25,11.75) circle (1cm);
    \draw  (23.25,11.75) circle (1.75cm);
    \draw [ dashed] (23.25,11.75) circle (1.5cm);
    \draw [->, >=Stealth] (23.25,11.75) -- (24,13);
    \draw [->, >=Stealth] (23,10.25) -- (20.5,10.25);
    \draw [short] (23.25,10.75) -- (23.25,10);
    \node [font=\normalsize] at (23.5,12) {R};
    \node [font=\normalsize] at (19.25,10.25) {ends of the beam};
    \end{circuitikz}
    }%
    
      % Specify the path to your TikZ file
        \caption{option4}
       % \label{fig5}
    \end{figure}

\end{document}