\let\negmedspace\undefined
\let\negthickspace\undefined
\documentclass[journal]{IEEEtran}
\usepackage[a5paper, margin=10mm, onecolumn]{geometry}
\usepackage{lmodern} % Ensure lmodern is loaded for pdflatex
 % Include tfrupee package
\setlength{\headheight}{1cm} % Set the height of the header box
\setlength{\headsep}{0mm}     % Set the distance between the header box and the top of the text
\usepackage{enumitem}
\usepackage{gvv-book}
\usepackage{gvv}
\usepackage{cite}
\usepackage{amsmath,amssymb,amsfonts,amsthm}
\usepackage{algorithmic}
\usepackage{graphicx}
\usepackage{textcomp}
\usepackage{xcolor}
\usepackage{txfonts}
\usepackage{listings}
\usepackage{enumitem}
\usepackage{mathtools}
\usepackage{gensymb}
\usepackage{comment}
\usepackage[breaklinks=true]{hyperref}
\usepackage{tkz-euclide} 
\usepackage{listings}
% \usepackage{gvv}                                        
\def\inputGnumericTable{}                                 
\usepackage[latin1]{inputenc}                                
\usepackage{color}                                            
\usepackage{array}                                            
\usepackage{longtable}                                       
\usepackage{calc}                                             
\usepackage{multirow}                                         
\usepackage{hhline}                                           
\usepackage{ifthen}                                           
\usepackage{lscape}
\begin{document}

\bibliographystyle{IEEEtran}
\vspace{3cm}

\title{1-1.2-18}
\author{AI24BTECH11008- Sarvajith
}
% \maketitle
% \newpage
% \bigskip
{\let\newpage\relax\maketitle}

\renewcommand{\thefigure}{\theenumi}
\renewcommand{\thetable}{\theenumi}
\setlength{\intextsep}{10pt} % Space between text and floats
\numberwithin{equation}{enumi}
\numberwithin{figure}{enumi}
\renewcommand{\thetable}{\theenumi}
\textbf{Question: }\\
If the origin is the centroid of the triangle PQR with vertices
\textbf{P}\brak{2a, 2, 6}, \textbf{Q}\brak{-4, 3b, -10} and \textbf{R}\brak{8, 14, 2c}, then find the values of a, b and
c.\\
\textbf{Solution: }\\
\renewcommand{\tablename}{TABLE 1}
\begin{table}[h!]    
  \centering
  \begin{tabular}{|c|c|}
\hline
\textbf{points} & \textbf{values}\\
\hline
\textbf{A} & $\myvec{-5\\1}$\\
\hline
\textbf{B} & $\myvec{1\\p}$\\
\hline
\textbf{C} & $\myvec{4\\-2}$\\
\hline
\end{tabular}

  \caption{values of the geometrical points in given question}
  \label{tab1-1.2-18-1}
\end{table}
\textbf{proof: }\\


\section*{Proof of the Centroid Formula using Matrix Notation}

Given a triangle with vertices A\brak{x_1, y_1,z_1}, \brak{x_2, y_2,z_2}, and \brak{x_3, y_3,z_3}, the centroid \brak{x_g, y_g,z_g} of the triangle is defined as the point where the three medians of the triangle intersect. The centroid can also be found as the average of the coordinates of the vertices.

\subsection*{Step 1: Representing the Vertices as Column Vectors}

The coordinates of the vertices A ,  B , and  C  can be represented as column vectors:
\begin{align}


\subsection*{Step 2: Centroid as a Column Vector}

The centroid G\brak{x_g, y_g,z_g } can be represented as a column vector:
\begin{align*}
    \mathbf{G} = \begin{bmatrix} x_g \\ y_g \\ z_g \end{bmatrix}
\end{align*}

\subsection*{Step 3: Centroid Formula in Matrix Notation}

The centroid \brak{G} is the average of the coordinates of the vertices \brak{A} , \brak{B} , and \brak{C}:

\begin{align*}
    \mathbf{G} = \frac{1}{3} \left(\mathbf{A} + \mathbf{B} + \mathbf{C}\right)
\end{align*}
\subsection*{Step 4: Matrix Addition and Scalar Multiplication}
First, add the vectors \( \mathbf{A} \), \( \mathbf{B} \), and \( \mathbf{C} \):
\begin{align*}
    \mathbf{A} + \mathbf{B} + \mathbf{C} = \begin{bmatrix} x_1 \\ y_1 \\z_1\end{bmatrix} + \begin{bmatrix} x_2 \\ y_2 \\z_2 \end{bmatrix} + \begin{bmatrix} x_3 \\ y_3 \\z_3 \end{bmatrix} = \begin{bmatrix} x_1 + x_2 + x_3 \\ y_1 + y_2 + y_3 \\z_1 + z_2 +z_3\end{bmatrix}
\end{align*}
Next, multiply by the scalar \( \frac{1}{3} \):
\begin{align*}
    \mathbf{G} = \frac{1}{3} \begin{bmatrix} x_1 + x_2 + x_3 \\ y_1 + y_2 + y_3\\z_1+z_2+z_3 \end{bmatrix} = \begin{bmatrix} \frac{x_1 + x_2 + x_3}{3} \\ \frac{y_1 + y_2 + y_3}{3} \\ \frac{z_1+z_2+z_3 }{3} \end{bmatrix}
\end{align*}
\subsection*{Conclusion}
Thus, the coordinates of the centroid G\brak{x_g, y_g,z_g} are given by:
\begin{align*}
    x_g = \frac{x_1 + x_2 + x_3}{3}, \quad y_g = \frac{y_1 + y_2 + y_3}{3}, \quad z_g= \frac{z_1+z_2+z_3}{3}
\end{align*}



This proves the centroid formula using matrix notation.
\begin{equation}
\textbf{P}\brak{x_1,y_1,z_1} = \brak{2a,4,6}\label{eq1.12.18.1}
\end{equation}
\begin{equation}
    \textbf{Q}\brak{x_2,y_2,z_2} = \brak{-4,3b,10}\label{eq1.12.18.2}
\end{equation}
\begin{equation}
    \textbf{R}\brak{x_3,y_3,z_3} = \brak{8,14,2c}\label{eq1.12.18.3}\\
\end{equation}

Given that, the centroid of the triangle \textbf{PQR} is origin\brak{0,0,0}. \\
Centroid\brak{G}.
$$let\,the\,matrix\;S =
\begin{bmatrix}
2a&4&6\\
-4&3b&10\\
8&14&2c
\end{bmatrix}$$\\
$$The\,matrix\;G = \frac{1}{3}
\begin{bmatrix}
    1&1&1
\end{bmatrix}S$$\\
after the matrix multiplication
$$G =\frac{1}{3}
\begin{bmatrix}
2a-4+8&4+3b+14&6+10+2c
\end{bmatrix}$$
and given that
$$G =\begin{bmatrix}
      0&0&0
      \end{bmatrix}$$\\
on comparing we get that
$$2a -4 + 8 = 0\\$$
\begin{equation}
    a = -2	\label{eq : 1-1.2-18-4}\\
\end{equation}
$$4 + 3b + 14 = 0$$
\begin{equation}
    b = -6 \label{eq : 1-1.2-18-5}
\end{equation}
$$6 + 10 + 2c = 0$$
\begin{equation}
    c = -8\label{eq : 1-1.2-18-6}
\end{equation}\\
$\therefore $ the values of a,b,c are -2,-6,-8 respectively.
\begin{figure}[h!]
   \centering
   \includegraphics[width=0.7\linewidth]{figs/Figure_1.png}
   \caption{plot for triangle}
   \label{plot}
\end{figure}
\end{document}
