\let\negmedspace\undefined
\let\negthickspace\undefined
\documentclass[journal]{IEEEtran}
\usepackage[a5paper, margin=10mm, onecolumn]{geometry}
%\usepackage{lmodern} % Ensure lmodern is loaded for pdflatex
\usepackage{tfrupee} % Include tfrupee package
\setlength{\headheight}{1cm} % Set the height of the header box
\setlength{\headsep}{0mm}     % Set the distance between the header box and the top of the text
\usepackage{enumitem}
\usepackage{gvv-book}
\usepackage{gvv}
\usepackage{cite}
\usepackage{amsmath,amssymb,amsfonts,amsthm}
\usepackage{algorithmic}
\usepackage{graphicx}
\usepackage{textcomp}
\usepackage{xcolor}
\usepackage{txfonts}
\usepackage{listings}
\usepackage{enumitem}
\usepackage{mathtools}
\usepackage{gensymb}
\usepackage{comment}
\usepackage[breaklinks=true]{hyperref}
\usepackage{tkz-euclide}
\usepackage{listings}
% \usepackage{gvv}                                        
\def\inputGnumericTable{}                                
\usepackage[latin1]{inputenc}                                
\usepackage{color}                                            
\usepackage{array}                                            
\usepackage{longtable}                                      
\usepackage{calc}                                            
\usepackage{multirow}                                        
\usepackage{hhline}                                          
\usepackage{ifthen}                                          
\usepackage{lscape}
\begin{document}

\bibliographystyle{IEEEtran}
\vspace{3cm}

\title{1-1.2-18}
\author{AI24BTECH11008- Sarvajith
}
% \maketitle
% \newpage
% \bigskip
{\let\newpage\relax\maketitle}

\renewcommand{\thefigure}{\theenumi}
\renewcommand{\thetable}{\theenumi}
\setlength{\intextsep}{10pt} % Space between text and floats


\numberwithin{equation}{enumi}
\numberwithin{figure}{enumi}
\renewcommand{\thetable}{\theenumi}
\textbf{Question: }\\
If the origin is the centroid of the triangle PQR with vertices
\textbf{P}\brak{2a, 2, 6}, \textbf{Q}\brak{-4, 3b, -10} and \textbf{R}\brak{8, 14, 2c}, then find the values of a, b and
c.\\
\textbf{Solution: }\\
\begin{table}[h!]    
  \centering
    \begin{tabular}{|c|c|}
\hline
\textbf{lengths} & \textbf{values}\\
\hline
\textbf{BC} & 5.5cm\\
\hline
\textbf{AL} & 5.3cm\\
\hline
\end{tabular}

      \caption{values of the geometrical points in given question}
        \label{tab1-1.2-18-1}
	\end{table}

	\textbf{proof: }\\
	if there are 3 points which form a triangle in 3 dimensional axis then, their centroid can be calculated using matrices by : \\
	\begin{enumerate} [1.]
	    \item First form a 3 x 3 matrix with the values of points as each row and name it as S\\
	             $$ S = \begin{bmatrix}
		                   x_1&y_1&z_1\\
				                 x_2&y_2&z_2\\
						               x_3&y_3&z_3
							                 \end{bmatrix}$$
									     \item then take another 3 x 1 matrix I which is filled with  $\frac{1}{3}$\\
									         $$ I = \begin{bmatrix}
										         \frac{1}{3}&\frac{1}{3}&\frac{1}{3}\\
											     \end{bmatrix}$$
											         \item now perform matrix multiplication I X S, which results in an 3 x 1 matrix G, which will have a the elements corresponding to centroid elements \\
												     $$G= \begin{bmatrix}
												             \frac{x_1+x_2+x_3}{3}&\frac{y_1+y_2+y_3}{3}&\frac{z_1+z_2+z_3}{3}
													         \end{bmatrix}$$
														     \item this indicates that centroid can be calculated using this method.
														        
															\end{enumerate}
															\begin{equation}
															\textbf{P}\brak{x_1,y_1,z_1} = \brak{2a,4,6}\label{eq1.12.18.1}
															\end{equation}
															\begin{equation}
															    \textbf{Q}\brak{x_2,y_2,z_2} = \brak{-4,3b,10}\label{eq1.12.18.2}
															    \end{equation}
															    \begin{equation}
															        \textbf{R}\brak{x_3,y_3,z_3} = \brak{8,14,2c}\label{eq1.12.18.3}\\
																\end{equation}

																Given that, the centroid of the triangle \textbf{PQR} is origin\brak{0,0,0}. \\
																Centroid\brak{G}.
																$$let\,the\,matrix\;S =
																\begin{bmatrix}
																2a&4&6\\
																-4&3b&10\\
																8&14&2c
																\end{bmatrix}$$\\
																$$The\,matrix\;G = \frac{1}{3}
																\begin{bmatrix}
																    1&1&1
																    \end{bmatrix}S$$\\
																    after the matrix multiplication
																    $$G =\frac{1}{3}
																    \begin{bmatrix}
																    2a-4+8&4+3b+14&6+10+2c
																    \end{bmatrix}$$
																    and given that
																    $$G =\begin{bmatrix}
																          0&0&0
																	        \end{bmatrix}$$\\
																		on comparing we get that
																		$$2a -4 + 8 = 0\\$$
																		\begin{equation}
																		    a = -2 \label{eq : 1-1.2-18-4}\\
																		    \end{equation}
																		    $$4 + 3b + 14 = 0$$
																		    \begin{equation}
																		        b = -6 \label{eq : 1-1.2-18-5}
																			\end{equation}
																			$$6 + 10 + 2c = 0$$
																			\begin{equation}
																			    c = -8\label{eq : 1-1.2-18-6}
																			    \end{equation}\\
																			    $\therefore $ the values of a,b,c are -2,-6,-8 respectively.
																			    \end{document}

